%%
% Copyright (c) 2017 - 2025, Pascal Wagler;
% Copyright (c) 2014 - 2025, John MacFarlane
%
% All rights reserved.
%
% Redistribution and use in source and binary forms, with or without
% modification, are permitted provided that the following conditions
% are met:
%
% - Redistributions of source code must retain the above copyright
% notice, this list of conditions and the following disclaimer.
%
% - Redistributions in binary form must reproduce the above copyright
% notice, this list of conditions and the following disclaimer in the
% documentation and/or other materials provided with the distribution.
%
% - Neither the name of John MacFarlane nor the names of other
% contributors may be used to endorse or promote products derived
% from this software without specific prior written permission.
%
% THIS SOFTWARE IS PROVIDED BY THE COPYRIGHT HOLDERS AND CONTRIBUTORS
% "AS IS" AND ANY EXPRESS OR IMPLIED WARRANTIES, INCLUDING, BUT NOT
% LIMITED TO, THE IMPLIED WARRANTIES OF MERCHANTABILITY AND FITNESS
% FOR A PARTICULAR PURPOSE ARE DISCLAIMED. IN NO EVENT SHALL THE
% COPYRIGHT OWNER OR CONTRIBUTORS BE LIABLE FOR ANY DIRECT, INDIRECT,
% INCIDENTAL, SPECIAL, EXEMPLARY, OR CONSEQUENTIAL DAMAGES (INCLUDING,
% BUT NOT LIMITED TO, PROCUREMENT OF SUBSTITUTE GOODS OR SERVICES;
% LOSS OF USE, DATA, OR PROFITS; OR BUSINESS INTERRUPTION) HOWEVER
% CAUSED AND ON ANY THEORY OF LIABILITY, WHETHER IN CONTRACT, STRICT
% LIABILITY, OR TORT (INCLUDING NEGLIGENCE OR OTHERWISE) ARISING IN
% ANY WAY OUT OF THE USE OF THIS SOFTWARE, EVEN IF ADVISED OF THE
% POSSIBILITY OF SUCH DAMAGE.
%%

%%
% This is the Eisvogel pandoc LaTeX template.
%
% For usage information and examples visit the official GitHub page:
% https://github.com/Wandmalfarbe/pandoc-latex-template
%%
% Options for packages loaded elsewhere
\PassOptionsToPackage{unicode}{hyperref}
\PassOptionsToPackage{hyphens}{url}
\PassOptionsToPackage{dvipsnames,svgnames,x11names,table}{xcolor}
\documentclass[
  a4paper,
  ,captions=tableheading
]{scrartcl}
\usepackage{xcolor}
\usepackage[left=1cm,right=1cm,top=2cm,bottom=2cm]{geometry}
\usepackage{amsmath,amssymb}

\usepackage[export]{adjustbox}
\usepackage{graphicx}

% add backlinks to footnote references, cf. https://tex.stackexchange.com/questions/302266/make-footnote-clickable-both-ways
\usepackage{footnotebackref}
\setcounter{secnumdepth}{5}
\usepackage{iftex}
\ifPDFTeX
  \usepackage[T1]{fontenc}
  \usepackage[utf8]{inputenc}
  \usepackage{textcomp} % provide euro and other symbols
\else % if luatex or xetex
  \usepackage{unicode-math} % this also loads fontspec
  \defaultfontfeatures{Scale=MatchLowercase}
  \defaultfontfeatures[\rmfamily]{Ligatures=TeX,Scale=1}
\fi
\usepackage{lmodern}
\ifPDFTeX\else
  % xetex/luatex font selection
\fi
% Use upquote if available, for straight quotes in verbatim environments
\IfFileExists{upquote.sty}{\usepackage{upquote}}{}
\IfFileExists{microtype.sty}{% use microtype if available
  \usepackage[]{microtype}
  \UseMicrotypeSet[protrusion]{basicmath} % disable protrusion for tt fonts
}{}

% Use setspace anyway because we change the default line spacing.
% The spacing is changed early to affect the titlepage and the TOC.
\usepackage{setspace}
\setstretch{1.2}
\makeatletter
\@ifundefined{KOMAClassName}{% if non-KOMA class
  \IfFileExists{parskip.sty}{%
    \usepackage{parskip}
  }{% else
    \setlength{\parindent}{0pt}
    \setlength{\parskip}{6pt plus 2pt minus 1pt}}
}{% if KOMA class
  \KOMAoptions{parskip=half}}
\makeatother
\usepackage{listings}
\newcommand{\passthrough}[1]{#1}
\lstset{defaultdialect=[5.3]Lua}
\lstset{defaultdialect=[x86masm]Assembler}
\setlength{\emergencystretch}{3em} % prevent overfull lines
\providecommand{\tightlist}{%
  \setlength{\itemsep}{0pt}\setlength{\parskip}{0pt}}
\usepackage{bookmark}
\IfFileExists{xurl.sty}{\usepackage{xurl}}{} % add URL line breaks if available
\urlstyle{same}
\definecolor{default-linkcolor}{HTML}{A50000}
\definecolor{default-filecolor}{HTML}{A50000}
\definecolor{default-citecolor}{HTML}{4077C0}
\definecolor{default-urlcolor}{HTML}{4077C0}

\hypersetup{
  pdftitle={DSP Exam Questions},
  pdfauthor={Thomas Debelle AUTHOR Anonymous student},
  hidelinks,
  breaklinks=true,
  pdfcreator={LaTeX via pandoc with the Eisvogel template}}

\title{DSP Exam Questions}
\usepackage{etoolbox}
\makeatletter
\providecommand{\subtitle}[1]{% add subtitle to \maketitle
  \apptocmd{\@title}{\par {\large #1 \par}}{}{}
}
\makeatother
\subtitle{\href{https://github.com/Tfloow/Q8_KUL}{An Open-Source
Summary}}
\author{Thomas Debelle AUTHOR Anonymous student}
\date{\today}


%
% for the background color of the title page
%
\usepackage{pagecolor}
\usepackage{afterpage}

%
% break urls
%
\PassOptionsToPackage{hyphens}{url}

%
% When using babel or polyglossia with biblatex, loading csquotes is recommended
% to ensure that quoted texts are typeset according to the rules of your main language.
%
\usepackage{csquotes}

%
% captions
%
\definecolor{caption-color}{HTML}{777777}
\usepackage[font={stretch=1.2}, textfont={color=caption-color}, position=top, skip=4mm, labelfont=bf, singlelinecheck=false, justification=raggedright]{caption}
\setcapindent{0em}

%
% blockquote
%
\definecolor{blockquote-border}{RGB}{221,221,221}
\definecolor{blockquote-text}{RGB}{119,119,119}
\usepackage{mdframed}
\newmdenv[rightline=false,bottomline=false,topline=false,linewidth=3pt,linecolor=blockquote-border,skipabove=\parskip]{customblockquote}
\renewenvironment{quote}{\begin{customblockquote}\list{}{\rightmargin=0em\leftmargin=0em}%
\item\relax\color{blockquote-text}\ignorespaces}{\unskip\unskip\endlist\end{customblockquote}}

%
% Source Sans Pro as the default font family
% Source Code Pro for monospace text
%
% 'default' option sets the default
% font family to Source Sans Pro, not \sfdefault.
%
\ifnum 0\ifxetex 1\fi\ifluatex 1\fi=0 % if pdftex
    \usepackage[default]{sourcesanspro}
  \usepackage{sourcecodepro}
  \else % if not pdftex
    \usepackage[default]{sourcesanspro}
  \usepackage{sourcecodepro}

  % XeLaTeX specific adjustments for straight quotes: https://tex.stackexchange.com/a/354887
  % This issue is already fixed (see https://github.com/silkeh/latex-sourcecodepro/pull/5) but the
  % fix is still unreleased.
  % TODO: Remove this workaround when the new version of sourcecodepro is released on CTAN.
  \ifxetex
    \makeatletter
    \defaultfontfeatures[\ttfamily]
      { Numbers   = \sourcecodepro@figurestyle,
        Scale     = \SourceCodePro@scale,
        Extension = .otf }
    \setmonofont
      [ UprightFont    = *-\sourcecodepro@regstyle,
        ItalicFont     = *-\sourcecodepro@regstyle It,
        BoldFont       = *-\sourcecodepro@boldstyle,
        BoldItalicFont = *-\sourcecodepro@boldstyle It ]
      {SourceCodePro}
    \makeatother
  \fi
  \fi

%
% heading color
%
\definecolor{heading-color}{RGB}{40,40,40}
\addtokomafont{section}{\color{heading-color}}
% When using the classes report, scrreprt, book,
% scrbook or memoir, uncomment the following line.
%\addtokomafont{chapter}{\color{heading-color}}

%
% variables for title, author and date
%
\usepackage{titling}
\title{DSP Exam Questions}
\author{Thomas Debelle AUTHOR Anonymous student}
\date{\today}

%
% tables
%

%
% remove paragraph indentation
%
\setlength{\parindent}{0pt}
\setlength{\parskip}{6pt plus 2pt minus 1pt}
\setlength{\emergencystretch}{3em}  % prevent overfull lines

%
%
% Listings
%
%


%
% general listing colors
%
\definecolor{listing-background}{HTML}{F7F7F7}
\definecolor{listing-rule}{HTML}{B3B2B3}
\definecolor{listing-numbers}{HTML}{B3B2B3}
\definecolor{listing-text-color}{HTML}{000000}
\definecolor{listing-keyword}{HTML}{435489}
\definecolor{listing-keyword-2}{HTML}{1284CA} % additional keywords
\definecolor{listing-keyword-3}{HTML}{9137CB} % additional keywords
\definecolor{listing-identifier}{HTML}{435489}
\definecolor{listing-string}{HTML}{00999A}
\definecolor{listing-comment}{HTML}{8E8E8E}

\lstdefinestyle{eisvogel_listing_style}{
  language         = java,
  numbers          = left,
  xleftmargin      = 2.7em,
  framexleftmargin = 2.5em,
  backgroundcolor  = \color{listing-background},
  basicstyle       = \color{listing-text-color}\linespread{1.0}%
                      \lst@ifdisplaystyle%
                      \small%
                      \fi\ttfamily{},
  breaklines       = true,
  frame            = single,
  framesep         = 0.19em,
  rulecolor        = \color{listing-rule},
  frameround       = ffff,
  tabsize          = 4,
  numberstyle      = \color{listing-numbers},
  aboveskip        = 1.0em,
  belowskip        = 0.1em,
  abovecaptionskip = 0em,
  belowcaptionskip = 1.0em,
  keywordstyle     = {\color{listing-keyword}\bfseries},
  keywordstyle     = {[2]\color{listing-keyword-2}\bfseries},
  keywordstyle     = {[3]\color{listing-keyword-3}\bfseries\itshape},
  sensitive        = true,
  identifierstyle  = \color{listing-identifier},
  commentstyle     = \color{listing-comment},
  stringstyle      = \color{listing-string},
  showstringspaces = false,
  escapeinside     = {/*@}{@*/}, % Allow LaTeX inside these special comments
  literate         =
  {á}{{\'a}}1 {é}{{\'e}}1 {í}{{\'i}}1 {ó}{{\'o}}1 {ú}{{\'u}}1
  {Á}{{\'A}}1 {É}{{\'E}}1 {Í}{{\'I}}1 {Ó}{{\'O}}1 {Ú}{{\'U}}1
  {à}{{\`a}}1 {è}{{\`e}}1 {ì}{{\`i}}1 {ò}{{\`o}}1 {ù}{{\`u}}1
  {À}{{\`A}}1 {È}{{\`E}}1 {Ì}{{\`I}}1 {Ò}{{\`O}}1 {Ù}{{\`U}}1
  {ä}{{\"a}}1 {ë}{{\"e}}1 {ï}{{\"i}}1 {ö}{{\"o}}1 {ü}{{\"u}}1
  {Ä}{{\"A}}1 {Ë}{{\"E}}1 {Ï}{{\"I}}1 {Ö}{{\"O}}1 {Ü}{{\"U}}1
  {â}{{\^a}}1 {ê}{{\^e}}1 {î}{{\^i}}1 {ô}{{\^o}}1 {û}{{\^u}}1
  {Â}{{\^A}}1 {Ê}{{\^E}}1 {Î}{{\^I}}1 {Ô}{{\^O}}1 {Û}{{\^U}}1
  {œ}{{\oe}}1 {Œ}{{\OE}}1 {æ}{{\ae}}1 {Æ}{{\AE}}1 {ß}{{\ss}}1
  {ç}{{\c c}}1 {Ç}{{\c C}}1 {ø}{{\o}}1 {å}{{\r a}}1 {Å}{{\r A}}1
  {€}{{\EUR}}1 {£}{{\pounds}}1 {«}{{\guillemotleft}}1
  {»}{{\guillemotright}}1 {ñ}{{\~n}}1 {Ñ}{{\~N}}1 {¿}{{?`}}1
  {…}{{\ldots}}1 {≥}{{>=}}1 {≤}{{<=}}1 {„}{{\glqq}}1 {“}{{\grqq}}1
  {”}{{''}}1
}
\lstset{style=eisvogel_listing_style}

%
% Java (Java SE 12, 2019-06-22)
%
\lstdefinelanguage{Java}{
  morekeywords={
    % normal keywords (without data types)
    abstract,assert,break,case,catch,class,continue,default,
    do,else,enum,exports,extends,final,finally,for,if,implements,
    import,instanceof,interface,module,native,new,package,private,
    protected,public,requires,return,static,strictfp,super,switch,
    synchronized,this,throw,throws,transient,try,volatile,while,
    % var is an identifier
    var
  },
  morekeywords={[2] % data types
    % primitive data types
    boolean,byte,char,double,float,int,long,short,
    % String
    String,
    % primitive wrapper types
    Boolean,Byte,Character,Double,Float,Integer,Long,Short
    % number types
    Number,AtomicInteger,AtomicLong,BigDecimal,BigInteger,DoubleAccumulator,DoubleAdder,LongAccumulator,LongAdder,Short,
    % other
    Object,Void,void
  },
  morekeywords={[3] % literals
    % reserved words for literal values
    null,true,false,
  },
  sensitive,
  morecomment  = [l]//,
  morecomment  = [s]{/*}{*/},
  morecomment  = [s]{/**}{*/},
  morestring   = [b]",
  morestring   = [b]',
}

\lstdefinelanguage{XML}{
  morestring      = [b]",
  moredelim       = [s][\bfseries\color{listing-keyword}]{<}{\ },
  moredelim       = [s][\bfseries\color{listing-keyword}]{</}{>},
  moredelim       = [l][\bfseries\color{listing-keyword}]{/>},
  moredelim       = [l][\bfseries\color{listing-keyword}]{>},
  morecomment     = [s]{<?}{?>},
  morecomment     = [s]{<!--}{-->},
  commentstyle    = \color{listing-comment},
  stringstyle     = \color{listing-string},
  identifierstyle = \color{listing-identifier}
}

%
% header and footer
%
\usepackage[headsepline,footsepline]{scrlayer-scrpage}

\newpairofpagestyles{eisvogel-header-footer}{
  \clearpairofpagestyles
  \ihead*{DSP Exam Questions}
  \chead*{}
  \ohead*{\today}
  \ifoot*{Thomas Debelle AUTHOR Anonymous student}
  \cfoot*{}
  \ofoot*{\thepage}
  \addtokomafont{pageheadfoot}{\upshape}
}
\pagestyle{eisvogel-header-footer}



%
% Define watermark
%

\begin{document}

\begin{titlepage}
\newgeometry{left=6cm}
\newcommand{\colorRule}[3][black]{\textcolor[HTML]{#1}{\rule{#2}{#3}}}
\begin{flushleft}
\noindent
\\[-1em]
\color[HTML]{5F5F5F}
\makebox[0pt][l]{\colorRule[435488]{1.3\textwidth}{4pt}}
\par
\noindent

{
  \setstretch{1.4}
  \vfill
  \noindent {\huge \textbf{\textsf{DSP Exam Questions}}}
    \vskip 1em
  {\Large \textsf{\href{https://github.com/Tfloow/Q8_KUL}{An Open-Source
Summary}}}
    \vskip 2em
  \noindent {\Large \textsf{Thomas Debelle AUTHOR Anonymous student}}
  \vfill
}

\noindent
\includegraphics[width=35mm, left]{KULlogo.png}

\textsf{\today}
\end{flushleft}
\end{titlepage}
\restoregeometry
\pagenumbering{arabic}

% don't generate the default title
% \maketitle


{
\setcounter{tocdepth}{3}
\tableofcontents
}
\section{\texorpdfstring{\textbf{January 30
2024}}{January 30 2024}}\label{january-30-2024}

\textbf{Question 1}\\
1. Chapter-2 p.40 (``Polyphase decomposition: Example\ldots''): Explain
how the presented\\
equations are exploited in general oversampled filter banks (=Chapters
12-13).\\
2. Chapter-3 p.19 (``3. Least squares estimation\ldots''). Explain the
meaning of the parameters\\
K and L, and how these have to be set. What procedure does Matlab use to
solve such a\\
least squares estimation problem?\\
3. Chapter-4 p.10 (``This is a `Quadratic Optimization'\ldots{}''):
Provide similar formulas for Type-2\\
linear phase FIR filter design (p.8), and explain all the symbols used
in these formulas.\\
4. Chapter-5 p.16 (``From (*) (p.12), it follows that\ldots''):
Generalize the given formulas for the\\
case with 3 power complementary filters (as on p.20).\\
5. Chapter-6 p.33: Provide a justification for the `lumping' of noise
sources (as illustrated in\\
subsequent pages) that is explicitly based on the given formulas (with
`DC-gain' and\\
`noise-gain').\\
\textbf{Question 2}\\
1. Chapter-7 p.24 (``MMSE cost function can be expanded as\ldots{}''):
How does the Wiener filter\\
formula ( wWF=(Xuu)-1Xdu ) and/or its components (Xuu and Xdu) change in
the case of a\\
`linear combiner' problem (as on p.20)? What is the computational
complexity (to solve the\\
resulting set of equations) in this case (as on p.27)?\\
2. Chapter-8: Explain in your own words how the characteristics of the
filter input signal (uk)\\
influence the behavior of the LMS algorithm? What is then the `ideal'
filter input signal in\\
this respect?\\
3. Chapter-9 p.35 (``4-by-4 example\ldots{}''): If the adaptive filter
input signal and the desired\\
output signal have different `dynamics' (for instance if the
characteristics of one are very\\
stationary, while the characteristics of the other are very
non-stationary), would it be\\
possible/useful to apply two different exponential weighting factors (a
first one in the part\\
corresponding to the input signal and a second one in the part
corresponding to the\\
desired output signal of the signal flow graph)?\\
4. Chapter-10 p.19 (``Theorem\ldots''): The graph also shows an equality
for the accumulated\\
product of the cosines of the rotation angles. Provide an explanation
for this equality.\\
5. In Chapter-11 p.24 (``relevant sub-problem is\ldots''): Explain why
the framed triangular matrix\\
and corresponding right-hand side vector in the first equation
correspond to the formulas\\
that are added in the second equation.\\
\textbf{Question 3}\\
1. Chapter-12 p.36 (``A solution is as follows\ldots''): Prove that the
conditions Fo(z)=H1(-z) and\\
F1(z)=-Ho(-z) indeed lead to alias-free operation. How is the frequency
response of Fo(z)\\
and F1(z) then related to the frequency response of Ho(z) and H1(z)?\\
2. Chapter-13 p.12 (``ii) Necessary \& sufficient condition\ldots''):
Explain in your own words the\\
statement ``hence \(pr(z)=\text{pure delay}=z-\delta\), and all other
\(pn(z)=0\)''. Explain how this then leads\\
to the next formula.\\
3. Chapter-14 p.13 (``Conclusion: economy in\ldots''): Explain in your
own words the statement\\
``N filters for the price of 1''.\\
4. Chapter-14 p.27 (``Example-1: Define B(z4)\ldots''):\\
a) Specify B(z) for the case where N=5 and D=2.\\
b) Derive the conditions for perfect reconstruction and specify when the
resulting set of\\
equations can be solved.\\
cfr. slide 31 for an example

\section{\texorpdfstring{\textbf{January 11
2024}}{January 11 2024}}\label{january-11-2024}

\textbf{Question 1}\\
1. Chapter-2 p36 (``Example: HP anti-aliasing\ldots''): Explain the
relevance of the presented operations in the filter bank context
(=Chapters 12-13-14).\\
2. Chapter-3 p.29 (``At the Rx, throw away L samples
corresponding\ldots''). Rewrite the formula for the case where the order
of the FIR channel is larger than the cyclic prefix length.\\
3. Chapter-4 (``Filter Design''): Explain how the filter phase response
is controlled in IIR filter design, and compare this to linear-phase FIR
filter design.\\
4. Chapter-5 p.30 (``Derivation similar to p.22\ldots''): Explain why
the highlighted element in the first formula has to be a zero (unlike in
p.21 and p.22).\\
5. Chapter-6: Draw a ``parallel realization'' of
\(H(z)=(1+az-1)-1 + (1+bz-1)-1\). Insert all relevant quantization noise
sources and compute the corresponding noise transfer functions. Can some
of these noise sources be lumped into an equivalent noise source? Why
(not)?

\textbf{Question 2}\\
1. Chapter-7 p.24 (``MMSE cost function can be expanded as\ldots{}''):
How does the Wiener filter formula (\(wWF=(Xuu)-1Xdu\) ) and/or its
components (Xuu and Xdu) change in the case of a multi-channel FIR
problem (as on p.19)?\\
2. Chapter-8: Explain how the LMS algorithm can be viewed as an RLS
algorithm with a specific substitution for the input signal correlation
matrix. Based on this link with RLS, provide an intuitive explanation
for the statement that the convergence of the LMS depends on the
correlation matrix eigenvalue spread.\\
3. Chapter-9 p.38 (``Residual extraction\ldots''): Consider the case
where uk is an all-zero vector and dk is non-zero (and R{[}k{]} is
full-rank). What would be the corresponding rotation angles and epsilon,
and hence the a posteriori and a priori residual.\\
4. Chapter-10 p.20 (``The main trick\ldots''): Redraw the signal flow
graph when the ``main trick'' is used to remove the column with R15,
R25, \ldots{} Define the relevant epsilon-signals in the signal flow
graph (with subscripts \& superscripts).\\
5. In Chapter-11 p.25 (``Recursive Square-Root\ldots''): Could residual
extraction (cfr. Chapter 9) be added to this algorithm (would it also
require only the ``lower-right/lower part'' as stated on p.23)? What
exactly would be the meaning of the extracted residuals?

\textbf{Question 3}\\
1. Chapter-12 p.12 (``Noise reduction\ldots''): If D=4 (instead of D=3)
and if the Gi's are not all equal to 1, what would be a condition for
alias-free operation (a general formula is sufficient here) and what
would be the resulting (linear) ``distortion function''?\\
2. Chapter-13 p.11 (``This can be verified\ldots''). Provide an
equivalent verification where in the first step R(z)E(z) is swapped with
the upsampling (instead of the downsampling).\\
3. Chapter-13 p.31 (``Given E(z)\ldots''). Consider the case with N=4,
D=1 and LE=3. Construct a (simple) example with transfer functions Hi(z)
and Fi(z) that provide perfect reconstruction.\\
4. Chapter-14 p.27 (``Example-1: Define B(z4)\ldots''):\\
a) Specify B(z) for the case where N=7 and D=4.\\
b) Provide the corresponding proof (similar to the proof on p.27) (with
explanation in words for non-trivial steps) that a 7-channel
DFT-modulated filter bank is indeed obtained with this B(z).

\section{\texorpdfstring{\textbf{January 31 2023 (example
exam)}}{January 31 2023 (example exam)}}\label{january-31-2023-example-exam}

\textbf{Question 1}

1. \textbf{Chapter-3}: Specify the computational complexity (number of
multiplications per second) of the DMT receiver operations (only those
discussed in Chapter-3), provide (approximate)\\
formulas that contain the main DMT parameters (N, L, symbol rate,
\ldots).

2. \textbf{Chapter-4 p.8} (``Linear Phase FIR Filters -- Type
1\ldots Type 2\ldots''). Explain why `modulating' a type-3 filter
results in a Type-3 filter. Illustrate the statement
``BP\textless-\textgreater BP'' by sketching the\\
filter frequency responses. Are the two `BPs' exactly the same?

3. \textbf{Chapter-5 p.18} (``Repeated application results in `lossless
lattice' \ldots''): Explain the\\
appearance of the multiplication by `cosq4' and `sinq4' in this lossless
lattice realization.

4. \textbf{Chapter-5 p.30} (``Derivation similar to p.22 (overlap-save,
similar for overlap-add) \ldots''):\\
Explain why the highlighted element in the first formula has to be a
zero (unlike in p.21\\
and p.22).

5. \textbf{Chapter-6 p.32}

\textbf{Question 2}

1. \textbf{Chapter-7 p.8} (``example: channel identification'') and p.14
(``example: channel equalization (training mode)''): Explain and compare
these two example applications.

2. \textbf{Chapter-8 p.21} (``LMS analysis in a nutshell: Noisy
gradients\ldots.''): Explain the `noisy\\
gradients' effect in an acoustic echo cancellation set-up, with a
near-end speaker that is\\
sometimes active and sometimes not active.

3. \textbf{Chapter-9 p.35} (``4-by-4 example''): Give an intuitive
explanation for the fact that\\
exponential weighting can be implemented here only by adding
multiplications with l after\\
the delay operations.

4. \textbf{Chapter-10 p.13} (``Preliminaries / LS residuals are not
changed\ldots{}''): Explain in your own\\
words how this property is exploited in p.19 (``Theorem'').

5. In \textbf{Chapter-11 p.24} (``Relevant sub-problem is\ldots''): If
the Kalman Filter would also have to\\
compute xk-1\textbar k (next to xk\textbar k and xk+1\textbar k ) how
would the formulas on the slide have to be\\
adapted?

\subsection{\texorpdfstring{\textbf{Question
3}}{Question 3}}\label{question-3}

\textbf{Chapter-12 p.4 (``What we have in mind is this\ldots''):} In the
given block scheme (without any upsampling/downsampling), what would be
the condition for perfect reconstruction? Would a set of power
complementary filters provide perfect reconstruction?

Does someone have an idea for this question? Especially the last part if
a set of power complementary functions provides perfect reconstruction,
my intuition says yes, but I don't have a good explanation for it.
Doesn't seem logical to me, we have to invert the damage done by the
H(z)'s, only possible by filter with transfer function equal to 1/H(z)
(which is not power complementary)

2. \textbf{Chapter-13 p.18} (``Design Procedure:\ldots''): Explain in
your own words how the stability

issue mentioned in the last sentence is overcome by having D\textless N
instead of D=N?

3. \textbf{Chapter-13 p.33} (``Example N=4\ldots''): Indicate how the
formula changes (number of

unknowns versus number of equations) if the order of all the synthesis
filters is increased

by one (=copy the formula and then highlight where in the matrices
additional entries

appear).

4. \textbf{Chapter-14 p.27} (``Example-1: Define B(z4)\ldots''): Specify
B(z) for the case where N=4 and

D=3 and provide the corresponding proof that a 4-channel DFT-modulated
filter bank is

obtained with this B(z) (similar to the proof on p.27 for N=8 and D=4).

5. \textbf{Chapter-15 p.12} (``If maximally decimated\ldots''). Provide
an interpretation of the last formula

by comparison with the DTFT on p.4. (What are the basis functions now?
How many basis

functions?)

\section{\texorpdfstring{\textbf{January 12 2023 (example
exam)}}{January 12 2023 (example exam)}}\label{january-12-2023-example-exam}

Instructions:

\begin{itemize}
\tightlist
\item
  Give structured, concise but complete answers, max 1 page per
  sub-question.\\
\item
  Start a new page for each sub-question.\\
\item
  Turn in exactly 8 sheets (for 15 sub-questions) + this sheet (to be
  checked and stapled).
\end{itemize}

\subsection{\texorpdfstring{\textbf{Question
1}}{Question 1}}\label{question-1}

\begin{enumerate}
\def\labelenumi{\arabic{enumi}.}
\item
  \textbf{Chapter-4 (``Filter Design''):} Explain how the filter phase
  response is controlled in IIR filter design, and compare this to FIR
  filter design.

  Design of analog filter -\textgreater{} digital, difficult to control
  phase response. IIR can never have linear phase response, compared to
  FIR, easy to control and have certain linear phase response.

  For a linear phase filter H(w) should be symmetric or antisimetric.
  From other course Bessel thomson approximation has a maximally flat
  group delay.

  The phase characteristic is taken into account by including it in the
  optimisation problem. This is not the case for FIR filters where only
  the magnute is included in the optimization function.
\item
  \textbf{Chapter-5 p.10 (``Repeated application results in `lattice
  form' \ldots''):} Explain the appearance of the multiplication by `b0'
  in the lattice realization.

  B0 is preserved through the transformations to lattice structure
\item
  \textbf{Chapter-5 p.28 (``Derivation similar to p.22 (overlap-save,
  similar for overlap-add) \ldots''):} Explain why the highlighted
  element in the first formula has to be a zero (unlike in p.21 and
  p.22).

  Does anyone have an idea? Is it because the coefficient b4 is in the
  polyphase components? Why don't we leave out the entire row of 0's
  here and the u{[}k-7{]} component, seems like it is not
  contributing?\\
  Maybe something to do with that FFT is used\ldots{}

  Thanks, seems logical!

  The extra zeros are required to make sure the resulting circulant
  matrix is square (FFT structure only valid for circulant matrices).
\item
  \textbf{Chapter-6 p.13 (``Coefficient quantization effect on pole
  locations / Higher-order systems\ldots''):} Explain in your own words
  the meaning of the (approximate) equation, and how it leads to the
  given conclusions.

  Gives relation between difference in coefficient value and the
  quantized coefficient and the difference of the root value and root
  value after quantization. Closely spaced roots result in an
  amplification of the error made on the coefficient to the error made
  on the location of the root, resulting in poor behavior.
\item
  \textbf{Chapter-6 p.35 (``PS: In a direct form realization\ldots''):}
  Explain in detail why it is that all quantization noises can be lumped
  into e1 and e2. What are the corresponding noise transfer functions?

  Transfer functions of separate noise sources are equivalent. Transfer
  functions are both 1? The upper part is referred to the input and the
  lower part to the output.

  Linearity makes it possible to just sum the noise sources (only +
  operations in path where noise is generated). Transfer function of
  e2{[}k{]} = 1 and transfer function of e1{[}k{]} = IIR filter itself?
\end{enumerate}

\subsection{\texorpdfstring{\textbf{Question
2}}{Question 2}}\label{question-2}

\begin{enumerate}
\def\labelenumi{\arabic{enumi}.}
\item
  \textbf{Chapter-7 p.19 (``PS: Can generalize FIR filter to
  `multi-channel FIR filter'\ldots''):} For this specific (3-input)
  example, provide formulas for the (input auto-)correlation matrix, the
  cross-correlation vector, and for the Wiener filter.

  Combine 3 input vectors into u and three coefficient vectors into w,
  reduces to the same problem as before. What happens to input
  correlation matrix and cross correlation vector? Stay also the same?
\item
  \textbf{Chapter-8 p.37 (``compare to p.32-33\ldots.''):} Explain in
  your own words the appearance of the summation, the first formula, and
  the statement (in p.36) that ``D takes the place of L+1''.

  Replacing the original order of the filter L (thus L+1) coefficients
  by D polyphase components, which will all have L\_D = L+1/D
  coefficients per polyphase component. D takes the place of L+1 means
  that originally we would have a block length of L+1, but by writing in
  polyphase components this is replaced by a block length of D, which
  can be lower, limiting the introduced delay.
\item
  \textbf{Chapter-9 p.26 (``QRD for LS estimation\ldots''):} Explain why
  the ` * ` in the second formula does not play a role (in the third
  formula).

  These rows in the vector are independent of w and do not contribute to
  minimizing the w norm of the vector w.r.t. w.
\item
  \textbf{Chapter-10 p.20 (``The main trick\ldots''):} Redraw (sketch)
  the signal flow graph when the ``main trick'' is used to remove the
  column with R12. Define the relevant epsilon-signals in the signal
  flow graph (with subscripts \& superscripts).
\item
  \textbf{Chapter-11 p.10 (``PPS: Note that if in p.9 matrix U has only
  1 row\ldots''):} Explain in your own words, based on the given
  formulas, how the standard RLS algorithm formulas can be related to a
  specific linear regression parameter estimation problem with a
  specific `initial estimate'.

  Someone an idea please?

  For a regressor and an error that has a unit covariance matrix, the
  least squares parameter estimation of a linear MMSE estimator with an
  initial estimate can be seen as the same estimate of the standard RLS
  algorithm where the correlation matrix is replaced with the error
  covariance matrix of the initial estimate.
\end{enumerate}

\subsection{\texorpdfstring{\textbf{Question
3}}{Question 3}}\label{question-3-1}

\begin{enumerate}
\def\labelenumi{\arabic{enumi}.}
\item
  \textbf{Chapter-12 p.29 (``Now insert DFT-matrix\ldots''):} Matrix F
  is said to be a DFT matrix, but more generally can be any square
  invertible matrix. In general, would it also be possible and
  meaningful to have a non-square matrix F (with some alternative for
  the inversion)?

  Yes, this is what we are doing in the case of an oversampled filter
  bank, with E(z) and R(z) rectangular matrices
\item
  \textbf{Chapter-13 p.12 (``Necessary \& sufficient condition for PR is
  then\ldots''):} Based on the first )formula, provide an expression for
  the distortion function (function of delta and r).

  T(z) = z\textsuperscript{-r*z}-delta ???

  I think T(z) = z\^{}-r * pr(z\^{}4), so T(z) = z\^{}-r * z\^{}-4*delta

  Red is correct \textless3

  But isnt R and E after downsample? How is it z\^{}-4? -\textgreater{}
  R(z)E(z) = z\^{}-delta I\_N after upsampling to fourth power. This is
  for r = 0 here, so extra z\^{}-r required for general case.
\item
  \textbf{Chapter-14 p.13 (``Conclusion: economy in\ldots``):} Explain
  in your own words the statement ``N filters for the price of 1''.

  Filter is implemented only once, then modulated to higher frequencies
  by taking the IFFT.
\item
  \textbf{Chapter-14 p.27 (``Example-1: Define B(z4)\ldots''):} Specify
  B(z) for the case where N=8 and D=6 and provide the corresponding
  proof that an 8-channel DFT-modulated filter bank is obtained with
  this B(z) (similar to the proof on p.27 for N=8 and D=4).
\item
  \textbf{Chapter-15 p.15 (``Example: N=4, d=2\ldots'\,''):}
  Explain/derive the ``minimum norm solution'' formulas. What is the
  relevance of such a ``minimum norm solution''?
\end{enumerate}

\subsubsection{}\label{section}

\subsubsection{\texorpdfstring{\textbf{26 January
2021}}{26 January 2021}}\label{january-2021}

\textbf{Question 1}\\
1. Chapter-3 p.26 (`To ensure the time-domain signal is real-valued,
have to\\
choose\ldots{}'): Why is a real-valued time-domain signal needed? How
does this\\
choice modify the receiver structure?\\
\strut \\
At receiver: only half of values used, redundancy

2. Chapter-4 p.12: Explain by means of a few formulas how the
minimization problem\\
can be solved using QR-decomposition, instead computing Q\^{}-1.p.\\
See 7 january 2021

3. Chapter-5: Consider a linear phase FIR filter. Is it possible to use
a (LPC) lattice\\
realization for this filter? A lossless lattice realization?\\
No, 26 jan 2017 Q1.4\\
What is the reason that lossless lattice realization is not possible?

4. Chapter-5 p.13: Explain the relevance of the Schur-Cohn stability
test for the\\
derivation of the lattice-ladder realization.\\
IIR Lattice ladder realization: Ki's are computed from the coefficients
of the denominator\(\rightarrow\) this time the stability test can be
used to see if the transfer function is stable

5. Chapter-6 p.35: Explain why it is that all quantization noises can be
lumped into e1\\
and e2. What are the corresponding noise transfer functions?\\
11 january 2019, Q1.2

\textbf{Question 2}\\
1. Chapter-7 p.40: Provide an intuitive explanation for the unrealizable
Wiener filter\\
formula for the two considered cases.

2. Chapter-8 p.19: Explain the `noisy gradients' effect in an acoustic
echo cancellation\\
set-up, with a near-end speaker that is sometimes active and sometimes
not active.\\
F\\
3. Chapter-9 p.26: In this structure with `residual extraction', where
are the actual FIR\\
filter coefficients, i.e.~the estimated model for the echo path?

4. Chapter-9 p.35: Redraw (sketch!) the (relevant parts of the) signal
flow graph when\\
the `main trick' is used to remove the column with R15,\ldots R45.
Define the relevant\\
epsilon-signals to the signal flow graph (with subscripts \&
superscripts).

5. Chapter-10 p.19: Explain the statement ``is seen to require only the
lower right/lower part''.

\textbf{Question 3}\\
1. Chapter-11 p.39: Explain why the R(zD) is a D-by-N matrix (and not an
N-by-N or\\
D-by-D or N-by-D matrix).\\
Otherwise dimensions not compatible and PR criterium not satisfied

2. Chapter-12 p.7: Explain (in +/- half a page) the design procedure
mentioned at the\\
bottom of the slide.

3. Chapter-12 p.31: Specify B(z) for the case where N=5 and D=4.

4. Chapter-13 p.18: Explain the statement `let B(z) take the place of
distortion\\
function T(z)'. Will there be any distortion in this case?

5. In Chapter-14 p.12, explain the meaning of the reconstruction formula
(at the\\
bottom of the slide) and compared to the reconstruction formula of
p.4.\\
Compare with 7 january 2021, Q3.5

\subsubsection{}\label{section-1}

\subsubsection{\texorpdfstring{\textbf{7 January 2021} (Written exam due
to
Covid-19)}{7 January 2021 (Written exam due to Covid-19)}}\label{january-2021-written-exam-due-to-covid-19}

``

\textbf{Question 1}

\begin{enumerate}
\def\labelenumi{\arabic{enumi}.}
\item
  Chapter-3 slide 19: What is the meaning of this formula? Provide a
  suitable solution strategy to solve the minimization problem.
  Determine a good approximation by minimizing the distance between the
  exact and approximate solution. Strategy? Iterative?\\
  In this slide we see the least squares estimation that we can use for
  the channel estimation. We can for example send a training sequence
  every once in a while and thus we know what the expected signal would
  be. This expected signal, the training sequence that has been
  transmitted, form the matrix with the x-components, further referred
  to as the X-matrix

  The signal that is actually received is represented by the y-vector,
  containing the y-components. We then want to find the h-vector that
  minimizes the difference y-Xh. This h-vector will contain the
  estimated channel coefficients. We find the best estimation for the
  actual channel coefficients by calculating the square of the two-norm
  of the difference y-Xh and then find the values in the h-vector for
  which this is minimal. To solve this we can write this in a matrix
  form of the given values and convert it into the following least
  squares problem: Xh=y. We can now use the least squares method to find
  a solution:\\
  1) Compute the matrix X\^{}TX and the vector X\^{}Ty\\
  2) Form the augmented matrix for the matrix equation
  X\textsuperscript{TXh=X}Ty and row reduce\\
  3)This equation is always consistent and any solution ĥ is a least
  squares solution. In Matlab there are built in functions that can
  calculate this least squares problem for us.

  I think in practice done with the QR decomposition and
  backsubstitution.
\item
  Chapter-4 slide 12: Give an alternative strategy to Q\^{}-1*p to solve
  the minimization problem. Provide formulas. I was thinking maybe if
  you take c\^{}T = A, d = x and Gd=y than you have an similar problem
  as question 1? See chapter 9 QRD for LS

  Based on the explanation above (do notice the matlab notation):
  \(Gd = [Gd(\omega_1), Gd(\omega_2), ...], A = [c^t(\omega_1); c^t(\omega_2); ...]\),
  x remains x. Now: Ax = Gd can be solved using leased .squares. I don't
  know how to add the weighted function tho. Note that this can only be
  done with a `discretized' quadratic optimization problem. This is not
  compatible with the problem of slide 10!
\item
  Compare the complexity of the FIR realizations based on the
  multiplications that we have seen in the course.

  \begin{enumerate}
  \def\labelenumii{\arabic{enumii}.}
  \tightlist
  \item
    Direct -\textgreater{} L+1 multiplications and L additions

    \begin{enumerate}
    \def\labelenumiii{\arabic{enumiii}.}
    \setcounter{enumiii}{1}
    \tightlist
    \item
      Transposed direct -\textgreater{} L+1 multiplications and L
      additions\\
    \item
      Lattice (lpc lattice) -\textgreater{} 2L+1 multiplications and 2L
      additions\\
    \item
      Lossless lattice (2 outputs) -\textgreater{} 4L+2 multiplications
      and 2L additions\\
    \item
      Lossless lattice (N outputs) -\textgreater{} 2NL+N multiplications
      and 2(N-1)L additions\\
    \item
      Frequency domain implementation O(log(L))\\
    \item
      Frequency domain implementation using D-fold polyphase
      decomposition O(log(D))\\
    \end{enumerate}
  \end{enumerate}
\item
  Chapter-5 slide 14: Why do we need scaling? Is the scaling unique? ?
  \(\rightarrow\) has to do with the sum being one on the next slide and
  both number will be positive.

  Scaling needed if the magnitude of the H(z) itself is already larger
  than 1, to make this power complementarity possible?

  I don't think the scaling is unique as there are many scaling factors
  to let the magnitude of H(z) be below 1.
\item
  Chapter-6 slide 34: Explain why the noise sources are equivalent up to
  a delay. Why is this delay not a problem to combine them together?
  Every component (multiplier/adder) has a noise source, the difference
  between a noise source at adder 1 (most left) and adder 2 is 1 delay
  element. This will also be visible in the noise transfer functions.
  You can rewrite/adapt the transfer functions to make it like the
  source is located at another place. If you do this for all the noise
  sources to 1 specific point, you can add the transfer functions. (I
  would add that a delay is a linear operation, allowing each linearly
  shifted noise signal to be added to the others in the leftmost node
  and keeping and LTI filter) ← the delay identity has the extra
  requirement for the noise sources to be \emph{ergodic}. Linearity is
  required for the additions and multiplications, though.

  Quantization noise is assumed to be truly random, so a delay will not
  influence this noise contribution? There is no correlation between
  successive errors, so delay won't influence these error contributions.
  Correct??

  I think you also need to mention that the calculation of the effect on
  the output signal of a noise source does not change by lumping them
  together if the mean and covariance are correctly adapted.
\end{enumerate}

Question 2

\begin{enumerate}
\def\labelenumi{\arabic{enumi}.}
\item
  Chapter-7 slide 38: Explain the statement and the condition for the
  AEC. (Irr. error = variance of near signal) Even the unrealizable
  Wiener coefficients/transfer function cannot make the desired
  signal-input signal cross correlation equal to the desired signal
  autocorrelation due to the noise being added after the compensated
  transfer function in the model. Thus the noise autocorrelation is not
  accounted for in the model and cannot be compensated =\textgreater{}
  forms the irreducible error.\\
\item
  Chapter-8 slide 34: What is a rank-1 update? How does it define the
  complexity?

  All vectors can be written as a scaling of the first one
  =\textgreater{} matrix product in the second term of the weight update
  equation simplified to a concatenation of scaled vectors = O(L\^{}2)
  instead of a standard matrix product which would be O(L\^{}3)
\item
  Chapter-9 slide 26: What is the relevance of error residuals for
  AEC(acoustic error correction)? If there is no error remaining, this
  means that the filter properly represents the ``channel'' (acoustics,
  echo's, \ldots). So how lower the error, how better. Isn't the
  residual the voice of the speaker on the other side, instead of your
  echo (in an AEC scenario)? (i.e.~the part you actually want to
  \emph{keep}) -\textgreater{} yes, I agree -\textgreater{} no, I don't
  agree I agree up to a certain part. I think it will be your voice of
  the speaker on the other side so the part you want to keep + the error
  which you can't reduce down. (course 2023: chapter 9: slide 40 it says
  it's the near-end signal + residual echo) i agree with this\\
\item
  Chapter-9 slide 32: Explain the property and explain why it's useful
  for the derivation. Changing the input does not change the residual
  output. Useful? Check slide 35 and the way the notations to the
  epsilon signals are done.\\
\item
  Chapter-10 slide 20: How are the indicated parts of the first formula
  identified?\\
  Course text on adaptive filters, page 136:\\
\end{enumerate}

Question 3

\begin{enumerate}
\def\labelenumi{\arabic{enumi}.}
\tightlist
\item
  Chapter-11 slide 34: Why is the polyphase decomposition of E(z) and
  R(z) per-row and per-column?\\
\item
  Chapter-12 slide 11: Explain DD(L\_E + L\_R + 1) and show equations.\\
  Every entry in R(z)*E(z) is of order LE+LR, thus has LE+LR+1
  coefficients. The dimension of the identity matrix is D, so there D*D
  equations for the polynomials with each LE+LR+1 coefficients, thus
  there are in total D*D(LE+LR+1) equations.\\
  Equations are: each element in \(R(z)*E(z) = z-\delta\) if on the main
  diagonal, and 0 otherwise.\\
\item
  Chapter-12 slide 25: Why does every En(z) need to be unimodular and
  FIR to have E(z) modular and FIR?\\
  \(RN-1-n(z)= z-\delta / En(z)\) (from slide 23). In order for
  RN-1-n(z) to be FIR, En(z) can only have 1 term in z, thus every En(z)
  has to be unimodular\\
\item
  Chapter-13 slide 7: Rewrite the selection matrix {[}0\_4x4 I\_4x4{]}
  into {[}I\_4x4 0\_4x4{]}. What does this mean for the Bi's?\\
  The bottom 4 rows and the top 4 rows in the circulant matrix on slide
  7 will be swapped. The Bi's are then calculated as {[}B0, B1, \ldots,
  B7{]}T = F {[}b0, 0, 0, 0, 0, b3, b2, b1{]}T (from the property on
  slide 5)\\
\item
  Chapter-14 slide 24: Explain the reconstruction formula and compare it
  to the formula on p4. Why is x0 a separate term?\\
  Slide 25: Reconstruction formula may be viewed as an expansion of
  u{[}k{]}, using a set of basis functions (infinitely many) (they are
  weighted + shifted versions of IR synthesis filters)\\
  x0 is a different term because f0 has the same upsampling factor as
  f1\\
  Additionally: if the x0 part wasn't separate, the summation of the
  separate subbands wouldn't cover the entire spectrum.
\end{enumerate}

\subsubsection{}\label{section-2}

\subsubsection{}\label{section-3}

\subsubsection{}\label{section-4}

\subsubsection{}\label{section-5}

\subsubsection{}\label{section-6}

\subsubsection{}\label{section-7}

\subsubsection{}\label{section-8}

\subsubsection{}\label{section-9}

\subsubsection{}\label{section-10}

\subsubsection{}\label{section-11}

\subsubsection{}\label{section-12}

\subsubsection{}\label{section-13}

\subsubsection{}\label{section-14}

\subsubsection{}\label{section-15}

\subsubsection{}\label{section-16}

\subsubsection{\texorpdfstring{\textbf{10 August
2020}}{10 August 2020}}\label{august-2020}

Question 1

\begin{enumerate}
\def\labelenumi{\arabic{enumi}.}
\item
  Explain how WLS for FIR filter design reduces the degrees of freedom
  when imposing a linear phase requirement.\\
  Due to the linear phase requirement we know the H(w) is symmetric. The
  d-coefficients are related to the IR and thanks to the symmetry, there
  are roughly half as many d-coefficients as H's (see also slide 6).
  When we take a look at the optimization criterion, we see it's an
  overdetermined system. Is Symmetry a requirement for linear phase?\\
  Either symmetry or antisymmetric, but both can be used to reduce the
  degrees of freedom\\
  \textless f\\
\item
  Can QRD be used in WLS FIR filter design? How?\\
  Chapter 9, QRD for LS estimation? The solution to the discretised
  optimisation problem (ch4, slide 12) is xopt=A\^{}-1 * p (xopt =
  Q\^{}-1 * p on the slide, here replaced with A to avoid confusion),
  where xopt contains the coefficients of the IR such that the distance
  to the desired IR is minimised. The set of equations to this problem
  can be numerically solved more efficiently with QRD on matrix A:
  A=Q*R, where Q is orthogonal so Q\^{}T*Q=I and
  Q\textsuperscript{-1=Q}T and R is upper triangle so R\^{}-1 is easy to
  compute. The relevant problem can be written as min\_x
  \textbar\textbar A*x-p\textbar\textbar\^{}2. Since multiplying with
  Q\^{}T is an orthogonal transformation (which preserves norm) min\_x
  \textbar\textbar Q\^{}T *( A*x-p)\textbar\textbar\^{}2 is an
  equivalent problem (same as ch9, slide 12). So
  Q\textsuperscript{T*A*xopt=Q}T*.Q*R*xopt=R*xopt=phat -\textgreater{}
  xopt=R\^{}-1*phat (Q\^{}T*p=phat), which is very easy to compute What
  is the point? Q*x\_opt = p is a determined system. There is no error
  because the error is left in mu.\\
\item
  When WLS FIR filter design is used for filters with an arbitrary
  required phase response, how does the design procedure change? Give
  relevant formulas.\\
  Is this a question related to question 2 or the first one?

  Fir filters can also have arbitrary phase responses, if they are not
  symmetric, thus it is not necessary to change it into an IIR filter.
  When designing the filter, we can no longer use the symmetric
  property, thus all coefficients have to be determined by the LS
  calculations, rather than only half of them. The phase response will
  have to be included in the objective function, as on slide 4-9. The
  procedure does not change by a lot, except that the transfer functions
  within the squared operation will now be complex.

  If the impulse response is not symmetric, the phase response will not
  be linear. Causality only means that h{[}k\textless0{]} = 0.

  Alternatively, use windowed design
\end{enumerate}

Question 2

\begin{enumerate}
\def\labelenumi{\arabic{enumi}.}
\item
  For overdetermined Ax=b, LS is x\_LS=(A\textsuperscript{T*A)}(-1)
  A\^{}T*b, manipulate this to justify back substitution in QRD RLS.\\
  The given formula is the original equation of the LS solution. To
  achieve better numerical results another representation is used to
  calculate the solution.\\
  Algebraic substitution: A = Q\_tilde*R =\textgreater{} AR\^{}(-1) =
  Q\_tilde\\
  (A\textsuperscript{T*A)}(-1) A\^{}T*b\\
  = (A\textsuperscript{T*A)}(-1) (Q\_tilde*R)\^{}T*b =
  (A\textsuperscript{T*A)}(-1) R\textsuperscript{T*Q\_tilde}T*b\\
  Bring the R\^{}T = (R\textsuperscript{T)}(-1)\^{}(-1) =
  (R\textsuperscript{(-1))}T\^{}(-1) in the inverse\\
  = (R\textsuperscript{(-1))}T*A\textsuperscript{T*A)}(-1)
  *Q\_tilde\^{}T*b\\
  Merge the two transposes together\\
  = ((A*R\textsuperscript{(-1))}T*A)\textsuperscript{(-1)*Q\_tilde}T*b\\
  Knowing AR\^{}(-1) = Q\_tilde\\
  = (Q\_tilde\textsuperscript{T*A)}(-1)*Q\_tilde\^{}T*b\\
  This is the same form as with the triangular backsubstitution on slide
  12.\\
\item
  How is Chapter-9, slide 32 used in derivation of QRD LSL? See january
  24 2020: Q2.1\\
\item
  Chapter-9, slide 33, the set of input signals {[} u{[}k{]} u{[}k-1{]}
  u{[}k-2{]} u{[}k-3{]} u{[}k-4{]} {]}\\
  is extended to {[} u{[}k{]} u{[}k-1{]} u{[}k-2{]} u{[}k-3{]}
  u{[}k-4{]} v{[}k{]} {]} with v{[}k{]} independent of u{[}k{]}, is QRD
  LSL still possible? Sketch a block scheme, omit details.\\
  I think this is possible. Why the `independent'? How can it be added?
  As the epsilons are a prediction based on the input signals\\
  What do you think? In chapter 10, slide 22 he says that ``The square
  root RLS algorithm form before is a special case of the square root
  Kalman filter algorithm (for Vk=0), so I guess if Vk isn't equal to
  zero like in this question, you'd end up with the square-root-Kalman
  filter? Does anyone think that's indeed what he's referring to, I'm
  really not sure\ldots{}

  I think is is more a case like on the screenshots added under ``24
  januari 2020'', Q2.3
\end{enumerate}

Question 3

\begin{enumerate}
\def\labelenumi{\arabic{enumi}.}
\item
  For maximally decimated filter banks, provide an intuitive explanation
  for perfect reconstruction despite aliasing in each channel.\\
  In a maximally decimated filter bank is D=N. The input is split into
  several frequency bands using the analysis filters. These are all
  separated and decimated which results in a stretched spectrum. When
  subband processing is done, these channels are compressed using
  upsampling. The copies are also present and these can be filtered out
  afterwards using a synthesis filter (interpolation filter, same band
  as the original). When the separate bands are combined this results in
  the original signal. PR is possible because of the synthesis bank,
  which is designed to remove aliasing effects.\\
  \textbf{Is this what you expect from this question or should we
  explain the polyphase decomposition and the alias transfer function +
  distortion function?}

  I think just explain chapter 11, slide 23 to 28?

\begin{lstlisting}
       I think that the explanation is correct and you can complete it with the slides 23 to 28
\end{lstlisting}
\item
  Compare maximally decimated filter banks with oversampled filter
  banks. What are the advantages/disadvantages of oversampling?\\
  Maximally decimated (D=N):\\
  - Not much design freedom\\
  - R(z) has to be IIR (FIR no ?) (No it is IIR see slide 11 of Ch 12,
  L\_R would need to be infinite aka IIR, this is also stated on slide
  12 at the bottom in case D = N)\\
  Oversampled (D\textless N):\\
  + Easier to design\\
  + FIR R(z) can always be found
\item
  Design a DFT modulated filter bank with 4 channels and 3 fold
  decimation. Give relevant formulas.\\
  This is an oversampled DFT-modulated FB. Modifying the example of the
  slides should suffice.

\begin{lstlisting}
     B(z) \= \[E0(z^4)    z^-1\*E4(Z^4)      z^-2 E8\]

               \[z^-3\*E9        E1                  z^-1\*E5\]

               \[z^-2\* E6       z-3\*E10             E3    \]

               \[z^-1\*E3        z^-2 E7         z^-3\*E11\]
\end{lstlisting}
\end{enumerate}

Question 4 (about acoustic modem)

\begin{enumerate}
\def\labelenumi{\arabic{enumi}.}
\item
  What is the relevance of channel equalisation in the acoustic modem?

  The signal Xk will be modified by a channel H(z) and if you do the FFT
  at the receiver you won't have the same signals as at the transmitter
  due to this channel H(z). To recover the perfect signal (with the
  noise of course), one has to do the Freq-domain EQualization (FEQ)
  after having removed the CP and having done the FFT. slide 25
  Chapter-3.

  The channel equalization is easy due to the fact that you use a prefix
  -\textgreater{} makes the matrix H diagonal.
\item
  What is the relevance of the prefix in OFDM? What are the requirements
  for this prefix?
\end{enumerate}

The OFDM frame together with the prefix allows for a circular
convolution. In the receiver, the L samples corresponding with the
prefix are thrown away which results in a circulant matrix. This
circulant matrix can then be factorized using IDFT/DFT( H= F\^{}-1 *
diag(H0,..,Hn-1) * F). With this, a diagonal matrix is obtained. The
resulting output Y is then simply the input X multiplied with some
scalar H(all in frequency domain), so the X can be estimated using
component wise division. The prefix length needs to be longer than the
impulse response of the channel. (slide 31-35`)

\begin{enumerate}
\def\labelenumi{\arabic{enumi}.}
\setcounter{enumi}{2}
\tightlist
\item
  How does one derive a channel model for the acoustic modem? What if
  one wants to equalise without modelling the channel?
\end{enumerate}

You can send training frames that are known at both sides. You know X
and Y, by using the OFDM modulation with CP, it is very easy to derive a
channel model. Y\_k\^{}(n) = Hn * X\_k \^{}(n) (slide 35,ch 3) which
leads to H = Y\_k * X\_k\^{}(-1) -\textgreater{} Least Square estimation

For the second part of the question, I understand it like this. In order
to avoid losing channel resources for training frames, a adaptive filter
can be used and you find a channel for each time.

\subsubsection{\texorpdfstring{\textbf{24 januari
2020}}{24 januari 2020}}\label{januari-2020}

Question 1

\begin{enumerate}
\def\labelenumi{\arabic{enumi}.}
\tightlist
\item
  The derivation of the FIR Lossless Lattice Realization (from one given
  filter transfer function) may involve `scaling' (Chapter-5, s14). Why
  is this needed? is the scaling unique?\\
  In one of the questions from previous years someone mentioned that it
  is bad to have a big/small H(z). Personally I think this is more the
  interpretation of the formula. The equation must hold so the sum of
  both magnitudes should equal one. A square is always positive (unless
  i\^{}2, but shouldn't matter here as we're taking the magnitude?) and
  when this value exceeds 1 the equation cannot hold anymore. So we
  should scale it to make sure it holds (normalize)?\\
  \^{} This sounds logical indeed. I also think the scaling is unique:
  assume that we have all the coefficients (b0, \ldots{} , bL) given. We
  are looking for a scaling factor. The only equation to solve for is
  the one on ch5, s14:\\
  1 equation, 1 unknown -\textgreater{} no degrees of freedom\\
  I don't think the scaling in unique. If you scale H smaller then H
  tildetilde will just become bigger and vice versa no? (CH5 s15)\\
  I think that for ch5, s14, the magnitude of H and Htildetilde should
  be complementary over the whole \(\omega\)-axis after scaling (so here
  ``\(\omega\)'' is a variable and not a constant)\\
  On second thought I think the scaling is not unique: As long as
  \(H(z)*H(z^-1) < 1\), the formula (*) (ch5, s14) will ensure the
  coefficients of Htildetilde to be complementary.
\end{enumerate}

So if H(z)*H(z\^{}-1) \textless{} 1, scaling suffices

\begin{enumerate}
\def\labelenumi{\arabic{enumi}.}
\item
  Explain how an FIR lossless lattice realization is derived for 3 given
  FIR filters that are power complimentary (Chapter-5 s22). Give
  formulas and signal flow graphs and explain how the required sequence
  of orthogonal rotations is defined.
\item
  If the resulting realization (Chapter-5 s22) represents an analysis
  filter bank, how can a synthesis filter bank be constructed such that
  perfect reconstruction is achieved for (decimation) D=1?\\
  Should we execute the rotations in reverse? Again no power loss\\
  I think they want you to refer to CH12 s9 here, but there maximum
  decimation is used so i'm not sure\\
  That's more likely indeed.\\
  PR: R(z)*E(z)=z\^{}-delta*I\_D = 1 if D=1 and delta = 0. So if
  R(z)={[}H\_tilde\_tilde(z) H\_tilde(z) H(z){]} (a row vector with all
  TFs), then E(z) = {[}H\_tilde\_tilde(z\^{}-1) H\_tilde(z\^{}-1)
  H(z\textsuperscript{-1){]}}T (a column vector). Then the PR criterion
  is the same as the lossless criterion. (=indeed slide 9 ch12,
  paraunitary FBs)\\
  Should not be this??:\\
\end{enumerate}

Question 2

\begin{enumerate}
\def\labelenumi{\arabic{enumi}.}
\item
  Explain how the QRD-LSL algorithm is derived from the QRD-RLS
  algorithm. Explain the meaning of the epsilon notation and the
  `main-trick' in the derivation. What is the relevance of the property
  illustrated on Chapter-9 s32? See previous years\\
  On this slide, you can observe the invariance of the residuals. It
  does not change under permutations. This applies to the a posteri and
  a priori error. This is the key for the derivation.\\
\item
  If the adaptive input signal and the desired output signal have
  different `dynamics' (for instance if the characteristics of one are
  very stationary, while the characteristics of the other are very
  non-stationary), would it be possible/useful to apply two different
  exponential weighting factors in the desired output signal of the
  (signal flow graph of) QRD-LSL?\\
  Two different weighting factors for the desired output signal? That
  sounds weird. In the SFG the exponential weighting factor is included
  in the small black box next to the rotation. How would it be possible
  to apply two different factors? Does this question suggest to adjust
  the weights for the input AND the output?\\
  The weighting factor is factored out, so I think this shouldn't be
  different.\\
  Cfr 16 augustus 2017\\
  I think they mean different weights for input (so in the rotation with
  R) and desired output (rotation with z), but no idea if that would
  work
\item
  How would you use the QRD-LSL algorithm for acoustic echo cancellation
  in a scenario with 1 loudspeaker and 2 microphones (i.e.~stereo
  recording)? Define all input/output signals.\\
  Cfr 13 augustus 2018 Q3.3 and have a look at fig 7.13 (equal channel
  length) and 7.17 (unequal) in the pdf on his website about adaptive
  filters (also available on burgiclan i think) \textbf{In the images
  below there is only 1 desired signal\ldots{} But below they mentioned
  there should be two desired signals. However I think it should be the
  same. Because you want to calculate the filter coefficients and you'd
  like to have an estimation for your channel. What do you think?}\\
  \textbf{But if you look at CH9 slide 26 you see the u comes from the
  loudspeaker and d from the microphone so isn't it logic that you have
  2 desired signals here? Then you have 2 different errors which are the
  near end signal of the two microphones ( with residual echo)}\\
  \textbf{I agree. How are the two residual errors combined to have one
  output at the end?}\\
  \textbf{I don't think they are combined together, they are both just
  the near end signal of one of the microphones}\\
  \textbf{Oh, so it's more like one mic in one corner of the room and
  another at the opposite corner? Both providing a good signal for that
  location? Not to have a better result globally?}\\
  \textbf{Yes I think this is what they mean. It's not that the residual
  is in this case an error but with echo cancellation it is the near end
  signal without echo}\\
  \textbf{Alright Thanks!}\\
  (ftp://\href{http://ftp.esat.kuleuven.ac.be/pub/SISTA/moonen/reports/chapter7.ps}{ftp.esat.kuleuven.ac.be/pub/SISTA/moonen/reports/chapter7.ps})\\
  Here are two screenshots:\\
\end{enumerate}

Question 3

\begin{enumerate}
\def\labelenumi{\arabic{enumi}.}
\tightlist
\item
  Explain why \& how polyphase decompositions are used to derive
  conditions for perfect reconstruction for oversampled
  analysis/synthesis filter banks.\\
  The polyphase decomposition is used to make things beautifully simple.
  Oversampled: D\textless N; Explain slides of ch11, s39 but with a bit
  of background from the previous slides.\\
\item
  Explain (briefly) how the conditions for alias-free and perfect
  reconstruction are derived for oversampled analysis/synthesis filter
  banks (using polyphase decomposition). Ch11, s40, with again
  background of previous slides\\
\item
  Explain (in detail) how these conditions for perfect reconstruction
  can be used to design oversampled filter banks (with FIR
  analysis/synthesis filters).\\
  Slides about the general PR-FB design? Ch12 s10-12
\end{enumerate}

\subsubsection{\texorpdfstring{\textbf{22 januari 2020
AM}}{22 januari 2020 AM}}\label{januari-2020-am}

\subsubsection{Question 1}\label{question-1-1}

\begin{enumerate}
\def\labelenumi{\arabic{enumi}.}
\item ~
  \subsubsection{Explain how `weighted least squares' optimization is
  used for FIR and IIR filter design. You have a set of correct values
  and you have a function to approximate these values. You want to
  minimize the distance between the correct and approximate values. This
  can be done by taking the best values. These best values are
  determined by a least square problem. The distances of certain correct
  values with his approximation are more important than at other values.
  These more important values need to be taken into account more by
  giving them a higher weight. The less important values will get a
  smaller
  weight.}\label{explain-how-weighted-least-squares-optimization-is-used-for-fir-and-iir-filter-design.-you-have-a-set-of-correct-values-and-you-have-a-function-to-approximate-these-values.-you-want-to-minimize-the-distance-between-the-correct-and-approximate-values.-this-can-be-done-by-taking-the-best-values.-these-best-values-are-determined-by-a-least-square-problem.-the-distances-of-certain-correct-values-with-his-approximation-are-more-important-than-at-other-values.-these-more-important-values-need-to-be-taken-into-account-more-by-giving-them-a-higher-weight.-the-less-important-values-will-get-a-smaller-weight.}
\item ~
  \subsubsection{When WLS optimization is used for IIR filter design,
  how can the filter's phase response be
  controlled/designed?}\label{when-wls-optimization-is-used-for-iir-filter-design-how-can-the-filters-phase-response-be-controlleddesigned}
\item ~
  \subsubsection{When WLS optimization is used for FIR filter design and
  the desired FIR filter has a desired phase response different from a
  linear phase response, how would the design procedure have to be
  adjusted? Provide formula's. (answer result is almost the same as with
  linear phase, in the end the cosines of c(w) are replaced by phase
  shifts.)}\label{when-wls-optimization-is-used-for-fir-filter-design-and-the-desired-fir-filter-has-a-desired-phase-response-different-from-a-linear-phase-response-how-would-the-design-procedure-have-to-be-adjusted-provide-formulas.-answer-result-is-almost-the-same-as-with-linear-phase-in-the-end-the-cosines-of-cw-are-replaced-by-phase-shifts.}
\end{enumerate}

\subsubsection{Question 2}\label{question-2-1}

\begin{enumerate}
\def\labelenumi{\arabic{enumi}.}
\item ~
  \subsubsection{Explain how QRD-LSL (least-squares-lattice) algorithm
  is derived from the QRD-RLS algorithm. Explain the meaning of the
  epsilon notation and the `main trick' in the derivation. What is the
  relevance of the property illustrated in Chapter-9 p 32? (the fact
  that epsilon doesn't change when
  permutations)}\label{explain-how-qrd-lsl-least-squares-lattice-algorithm-is-derived-from-the-qrd-rls-algorithm.-explain-the-meaning-of-the-epsilon-notation-and-the-main-trick-in-the-derivation.-what-is-the-relevance-of-the-property-illustrated-in-chapter-9-p-32-the-fact-that-epsilon-doesnt-change-when-permutations}
\item ~
  \subsubsection{A hexagon in the signal flow graph (SFG) represents a
  transformation with a 2-by-2 orthogonal matrix. Would it be possible
  (in QRD-RLS or QRD-LSL) to use non-orthogonal transformations to
  introduce the zero's in specific positions of the input vector (for
  example Chapter-9 p 17 replace {[}cos(theta) sin(theta); -sin(theta)
  cos(theta){]} by {[}1 tan(theta); -tan(theta) 1{]})? If ot why not?
  (answer chapter 9 p 12: orthogonal transformation preserves the
  norm)}\label{a-hexagon-in-the-signal-flow-graph-sfg-represents-a-transformation-with-a-2-by-2-orthogonal-matrix.-would-it-be-possible-in-qrd-rls-or-qrd-lsl-to-use-non-orthogonal-transformations-to-introduce-the-zeros-in-specific-positions-of-the-input-vector-for-example-chapter-9-p-17-replace-costheta-sintheta--sintheta-costheta-by-1-tantheta--tantheta-1-if-ot-why-not-answer-chapter-9-p-12-orthogonal-transformation-preserves-the-norm}

  No because if you use non orthogonal transformations you would
  introduce non-zero values in the positions where you previously
  created zeros and thus it would be impossible to create the triangular
  matrix and won't be able to use the QRD decomposition.

  \textbf{What do you think about that answer ?}\\
  \textbf{Sounds good for me\ldots{}}

  \textbf{I don't agree with previous answers. It is entirely possible
  to construct non-orthogonal transformations that introduce zeros in
  the same way as a Givens rotation would. For example
  left-multiplication with {[}1 0; 0 0{]} would do this.}

  \textbf{However, we are after all still trying to calculate a sosition
  (starting from an already nicely structured matrix) so the
  left-multiplicant in the triangularization needs to be orthogonal. In
  general, this happens if the transformations done on the matrix are
  orthogonal transformations since the composition of orthogonal
  transformations/matrices is again an orthogonal
  transformation/matrix.}
\item ~
  \subsubsection{How can the SFG for QRD\_LSL be extended to allow for
  `decision-directedchannel equalization' (as in Chapter-7, p15).
  (answer: choose first zero as target output). In the pdf on page 123
  there's a formula of W\_LS with decision directed mode and also a note
  about the performance: ``However, one can verify that
  decision-directed operation introduces an undesirable (inefficient)
  feedback path in the SFG.'' But I have no idea what should change in
  the
  SFG}\label{how-can-the-sfg-for-qrd_lsl-be-extended-to-allow-for-decision-directedchannel-equalization-as-in-chapter-7-p15.-answer-choose-first-zero-as-target-output.-in-the-pdf-on-page-123-theres-a-formula-of-w_ls-with-decision-directed-mode-and-also-a-note-about-the-performance-however-one-can-verify-that-decision-directed-operation-introduces-an-undesirable-inefficient-feedback-path-in-the-sfg.-but-i-have-no-idea-what-should-change-in-the-sfg}
\end{enumerate}

\subsubsection{Question 3}\label{question-3-2}

\begin{enumerate}
\def\labelenumi{\arabic{enumi}.}
\item ~
  \subsubsection{What are DFT-modulated filter banks? What are
  particular advantages of such filter
  banks.}\label{what-are-dft-modulated-filter-banks-what-are-particular-advantages-of-such-filter-banks.}
\item ~
  \subsubsection{What are oversampled filter banks? What are particular
  advantages of such filter
  banks?}\label{what-are-oversampled-filter-banks-what-are-particular-advantages-of-such-filter-banks}
\item ~
  \subsubsection{Consider a 5-channel DFT-modulated filter bank with
  4-fold downsampling. How can the analysis bank and synthesis bank be
  realized efficiently? And then how can perfect reconstruction be
  obtained in this case? Provide
  formulas.}\label{consider-a-5-channel-dft-modulated-filter-bank-with-4-fold-downsampling.-how-can-the-analysis-bank-and-synthesis-bank-be-realized-efficiently-and-then-how-can-perfect-reconstruction-be-obtained-in-this-case-provide-formulas.}
\end{enumerate}

\subsubsection{\texorpdfstring{\textbf{20 januari 2020
AM}}{20 januari 2020 AM}}\label{januari-2020-am-1}

Question 1

\begin{enumerate}
\def\labelenumi{\arabic{enumi}.}
\item
  Explain how an FIR (LPC) lattice realization is derived for a given
  FIR filter. Derivation of FIR 3 Lattice Realization slide 8-13\\
\item
  What is the relevance of the Schur-Cohn stability test appearing in
  this derivation?\\
  This is the procedure to compute the Ki's (reflection coefficients)\\
\item
  For a given FIR (LPC) lattice realization, how can the effect of
  quantization errors in the arithmetic operations be analyzed?\\
  Quantization noise is usually analyzed in a statistical manner. This
  is based on assumptions:\\
  - Each quantization error is random\\
  - Successive errors at output of given multiplier/adder are
  uncorrelated\\
  - Quantization errors at output of different multipliers/adders are
  uncorrelated\\
  A noise source is added after each multiplier/adder and since the
  filter is a linear filter this is added to the output.

  Independent sources can be summed and I think this results at the end
  in an equivalent noise added at y and y\_tilde. If you look at one
  block, there's a sum at the end with:\\
  - the error resulting from the b0 multiplication (mutual)\\
  - error from k3 multiplication (different/out)\\
  - error from sum at end (different/out)

  Although this is used a lot, it would be better to do an analysis in a
  deterministic manner as quantization is a nonlinear effect.\\
\item
  What are linear phase filters? What is the relevance of the linear
  phase characteristic in practice?\\
  Linear phase filters are those filters whose phase response varies
  linearly with frequency. Each of the frequency components of the
  signal is delayed by a constant amount. Hence such filters preserve
  the shape of the signal and introduce no distortions. This is an
  important characteristic required for some applications which cannot
  tolerate phase distortions at all.\\
\item
  Can linear phase filters be realized as an LPC lattice?\\
  No, 26 jan 2017 Q1.4
\end{enumerate}

Question 2

\begin{enumerate}
\def\labelenumi{\arabic{enumi}.}
\tightlist
\item
  Explain how the LMS algorithm is derived as a `stochastic gradient'
  algorithm. Explain all formulas used in the derivation.\\
\item
  In an acoustic echo cancellation (AEC) scenario with one loudspeaker
  and one microphone, if LMS is used for the adaptive filtering, how do
  the characteristics of the loudspeaker signal influence the behavior
  of the LMS algorithm? What is then the ideal loudspeaker signal in
  this respect?
\end{enumerate}

I'm not sure but with AEC, I think it is a fast time-varying system so
if we want to track it properly, we should choose a sufficiently large
\textbackslash mu because if it is too small the LMS solution will lag
behind. And instead of only having the noise due to the environment that
stays as in the MSE, you will also have some other noise because the
filter won't be perfect. And it leads to the formula of the pdf pg 72
that you can compare with slide 38 in chapter 7

The choice of \textbackslash mu must be based on a compromise between
fast tracking (large \textbackslash mu) and small J\_noise (small
\textbackslash mu). That is because we have stochastic parameters with
LMS and not statistical as with MSE.

\begin{enumerate}
\def\labelenumi{\arabic{enumi}.}
\setcounter{enumi}{2}
\item
  In an AEC scenario with one loudspeaker and one microphone, if LMS is
  used for the adaptive filtering, how do the characteristics (or
  presence) of the near-end speaker signal influence the behavior of the
  LMS algorithm? What is then the ideal near-end signal in this respect?
\item
  In a stereo AEC scenario with two loudspeakers (playing two different
  signals) and one microphone, can LMS still be applied? if yes, what
  would the LMS update formula look like?\\
  If one would like to have both signals separated afterwards, he needs
  two filters. One to filter out the first and another one for the
  second signal. If however both signals contribute to the same output
  then only the desired signals should be combined.\\
  In the first situation there are two FIR filters and in the second
  only one.\\
  I don't know how the desired signals can be combined\ldots{}

  In AEC you don't want the output from the loudspeakers, you need to
  filter this out of the received signal (see CH7 s 9). So in this case
  you have one desired response, the near end signal with echo, and 2
  inputs/ far end signals from the two loudspeakers. Since the 2
  loudspeakers are independent, wouldn't it be possible to just
  substract both of them? (something like on CH7 s19)
\end{enumerate}

Question 3

\begin{enumerate}
\def\labelenumi{\arabic{enumi}.}
\item
  Explain the context and general operation of the Short-time Fourier
  Transform.

\begin{lstlisting}
             STFT is for example used in a short lived sine wave, where the frequencies mustn't be viewed from an infinity time perspective but from a finite one. For example when making a spectogram, it's only the instantaneous frequencies that are important so you get a different frequency spectrogram for each time bin. 
\end{lstlisting}
\end{enumerate}

In practice, the procedure for computing STFTs is to divide a longer
time signal into shorter segments of equal length and then compute the
Fourier transform separately on each shorter segment.

\begin{enumerate}
\def\labelenumi{\arabic{enumi}.}
\setcounter{enumi}{1}
\tightlist
\item
  Explain how the STFT is realized by means of a (maximally decimated or
  oversampled) DFT-modulated filter bank.\\
\item
  Explain how the WOLA design and perfect reconstruction (PR) conditions
  relate to the filter bank design procedure and PR conditions in
  Chapter 12.
\end{enumerate}

\subsubsection{}\label{section-17}

\subsubsection{}\label{section-18}

\subsubsection{}\label{section-19}

\subsubsection{}\label{section-20}

\subsubsection{\texorpdfstring{\textbf{10 januari
2020}}{10 januari 2020}}\label{januari-2020-1}

\paragraph{Question 1}\label{question-1-2}

\begin{enumerate}
\def\labelenumi{\arabic{enumi}.}
\item
  What are noise transfer functions? Explain how such noise transfer
  functions are used in the analysis of quantisation effects in filter
  implementation.

  See previous years
\item
  What would change in the NTF analysis if there would be a MIMO system
  instead of a SISO system as described in Chapter-6?
\item
  Are there other/better ways of analysing the effect of quantisation
  errors in filter implementations and realisations? If yes, why? If
  yes, what other ways?
\end{enumerate}

See previous years + explain what deterministic analysis is (input is
not uncorrelated with error).

We suppose that the errors are uncorrelated, but that is not always the
case. If they were uncorrelated, we could just add it up but is is a bit
more complicated than that. If we do a deterministic simulation
(analysis is to complicated) then unwanted effects occur as for instance
zero input limit cycle oscillations up.

\paragraph{Question 2}\label{question-2-2}

\begin{enumerate}
\def\labelenumi{\arabic{enumi}.}
\item
  In which sense is the Kalman Filter a generalisation of the Least
  Squares parameter estimation.

  KF is dynamic, parameter estimation is static.
\item
  Explain how a square-root Kalman Filter algorithm is derived.

  See previous years (just explain what is going on in the slides of
  SRKF, why you can throw away the not relevant part to the problem
  (because the lhs is already triangular and you don't need the most
  part for zeroing the lhs using Givens rotations))
\item
  What can you say about the computational complexity of the QRD-RLS and
  the Square Root KF?

  Same in big O notation, but since you have to zero less (since Vk=0),
  the QRD-RLS is a bit faster than SRKF.\\
  O(L)?
\end{enumerate}

\paragraph{Question 3}\label{question-3-3}

\begin{enumerate}
\def\labelenumi{\arabic{enumi}.}
\item
  What are DFT-modulated filter banks? What are the advantages of such
  FB's?

  See previous years
\item
  Most analysis/synthesis filter banks have a complex impulse response
  (time domain). Why is that? How can this be avoided?
\end{enumerate}

Most filter banks are DFT modulated, so this means using a prototype
filter (easier to design). Prototype filter is shifted in the frequency
domain, this translated to a complex time domain impulse response (see
slide 12.14). This can be avoided by using cosine-modulated FB's (slide
12.27), since they have a symmetrical frequency response and thus real
in time domain. Or you could just design every Hn(z) separately.

\begin{enumerate}
\def\labelenumi{\arabic{enumi}.}
\setcounter{enumi}{2}
\tightlist
\item
  Explain how a 6-channel oversampled (D=5) DFT-modulated filter bank
  can be realised.
\end{enumerate}

Explain examples at the end of chapter 12.

\subsubsection{\texorpdfstring{\textbf{10 januari
2020}}{10 januari 2020}}\label{januari-2020-2}

Question 1

\begin{enumerate}
\def\labelenumi{\arabic{enumi}.}
\tightlist
\item
  Explain windowed FIR filters Take a design and convert the design to a
  predetermined range/ window.\\
\item
  Explain the relation between windowed FIR filters and WLS It will
  generate the same result.\\
\item
  Explain whether the windowed FIR filter will always have a linear
  phase or how we can enforce a linear phase Add extra delay to turn an
  non-linear, non-causal filter into an linear causal filter.
\end{enumerate}

Question 2

\begin{enumerate}
\def\labelenumi{\arabic{enumi}.}
\tightlist
\item
  Explain the SFG of QRD-LSR\\
\item
  Explain the main trick that is used in the derivation of the QRD-LSL
  algorithm.\\
\item
  The epsilon-signals are referred to as forward and backward prediction
  errors (up to a scaling with the `cosines product'). Specify the
  forward and backward prediction problems that are referred to here.
\end{enumerate}

Question 3

\begin{enumerate}
\def\labelenumi{\arabic{enumi}.}
\tightlist
\item
  Explain the difference in freedom of design between the maximally
  decimated and oversampled case in the \textbf{general
  analysis/synthesis case}.\\
\item
  Explain the difference in freedom of design between the maximally
  decimated and oversampled case in the \textbf{DFT-modulated FB}.\\
\item
  If E(z) contains only zero order polynomials, what order polynomials
  can R(z) contain?
\end{enumerate}

\subsubsection{\texorpdfstring{\textbf{11 januari
2019}}{11 januari 2019}}\label{januari-2019}

\subsubsection{Question 1}\label{question-1-3}

\begin{enumerate}
\def\labelenumi{\arabic{enumi}.}
\item ~
  \subsubsection{What are noise transfer functions? Explain how such
  noise transfer functions are used in the analysis of quantization
  effects in filter
  implementation.}\label{what-are-noise-transfer-functions-explain-how-such-noise-transfer-functions-are-used-in-the-analysis-of-quantization-effects-in-filter-implementation.}
\end{enumerate}

A noise transfer function describes how the noise acts on the output :
H\_n0 = Y(z)/E(z) where E(z) is the noise signal. It gives more
information about the influence of each noise source to the output. Each
adder and multiplier introduce noise. This means that given for a filter
realization and assuming a linear model we can lump all the noise
sources together in one equivalent source. We suppose that the
quantization errors at the output of given multipliers/adder are
uncorrelated.

As the quantizer introduces noise (after a multiplier or adder), if you
study the noise, you are studying the influence of the quantization on
the complete system.

\begin{enumerate}
\def\labelenumi{\arabic{enumi}.}
\setcounter{enumi}{1}
\item ~
  \subsubsection{Why is it that in such analysis different quantization
  noise sources can be lumped into one equivalent noise source, for
  specific filter realizations? (use
  formulas)}\label{why-is-it-that-in-such-analysis-different-quantization-noise-sources-can-be-lumped-into-one-equivalent-noise-source-for-specific-filter-realizations-use-formulas}

  Assume a direct form implementation of a FIR filter: y(k) = b0*u(k) +
  b1*u(k-1) + b2*u(k-2). Now assume that each addition and
  multiplication has a noise source (m = multiplier, a is adder): y(k) =
  b0*u(k) + em(k) + ea(k) + b1*u(k-1) + em(k-1) + ea(k-1) + b2*u(k-2) +
  em(k-2) + ea(k-2). All of these noise sources can be added because
  they are the same, upto a certain delay. This can be one equivalent
  source in a direct form representation (can be verified graphically)
  or two in the case of a transposed direct form representation.
\item ~
  \subsubsection{Are there other/better ways of analyzing the effect of
  quantization errors in filter implementations and realizations? If
  yes, why? If yes, what other
  ways?}\label{are-there-otherbetter-ways-of-analyzing-the-effect-of-quantization-errors-in-filter-implementations-and-realizations-if-yes-why-if-yes-what-other-ways}

\begin{lstlisting}
 Yes, quantization is actually a deterministic effect. However, analysing it as such is very difficult and is only helpful for finding limit cycles in the filter (which only occur in feedback systems).  
\end{lstlisting}
\end{enumerate}

\subsubsection{Question 2}\label{question-2-3}

\begin{enumerate}
\def\labelenumi{\arabic{enumi}.}
\item ~
  \subsubsection{Explain how general `overlap-add' and `overlap-save'
  based FIR filter realizations are
  derived.}\label{explain-how-general-overlap-add-and-overlap-save-based-fir-filter-realizations-are-derived.}

  Overlap-add and overlap-save filters are so called block filters, in
  these type of filters, a `block' of output samples is calculated at
  once. The coefficients of a normal FIR filter can be represented in
  matrix form with these filters (see slide 5 of chapter 13). This
  matrix can be extended and formed into a circulant matrix. These types
  of matrices can be decomposed using IDFT and DFT matrices. The
  resulting matrix will be a diagonal matrix with the entries being the
  frequency domain representation of the filter coefficients. This can
  then be implemented in a filterbank where now the FIR coefficients are
  2replaced with the distortion function(T(z)) coefficients. In an
  overlap-save implementation, the current block of samples is DFT
  modulated together with the previous samples in the analysis bank.
  This is then IDFT modulated where the first half is thrown away, while
  the second half is saved in the synthesis bank. In an overlap-add, the
  analysis bank does a DFT with the current block and the synthesis bank
  does the IDFT and adds the result to the previous IDFT block.
\item ~
  \subsubsection{Explain how the process delay (latency) and
  computational complexity of such realizations can be
  controlled.}\label{explain-how-the-process-delay-latency-and-computational-complexity-of-such-realizations-can-be-controlled.}

\begin{lstlisting}
 In overlap-save, the order of efficiency is O(log(L)), for large L, with L being the filter order. The latency however, is of order O(L). This can only be controlled by changing the filter order. In the overlap-add the efficiency and latency are different based on the amount of channels N and the filter order. If N\~L, then efficiency is O(log(L)) and latency nois O(L), but if N \<\< L, then efficiency is O(L) and the latency is O(N).
\end{lstlisting}
\item ~
  \subsubsection{Consider a `overlap-save' based realization based on a
  8-channel analysis and synthesis prototype filter. Derive an
  expression for the analysis and synthesis filter
  and}\label{consider-a-overlap-save-based-realization-based-on-a-8-channel-analysis-and-synthesis-prototype-filter.-derive-an-expression-for-the-analysis-and-synthesis-filter-and}
\end{enumerate}

\subsubsection{sketch the frequency response of this
filters.}\label{sketch-the-frequency-response-of-this-filters.}

See slide 11-12 of chapter 13 for derivation. Frequency responses are
All-pass (I think) since the entries are only ones, zeroes, or delays.

\subsubsection{Question 3}\label{question-3-4}

\begin{enumerate}
\def\labelenumi{\arabic{enumi}.}
\item ~
  \subsubsection{Explain how the QRD-LSL algorithm is derived from
  QDR-RLS. Also explain the epsilon-notation and the main trick that is
  used in this
  derivation.}\label{explain-how-the-qrd-lsl-algorithm-is-derived-from-qdr-rls.-also-explain-the-epsilon-notation-and-the-main-trick-that-is-used-in-this-derivation.}

  In the QRD-RLS, there is an asterisk(*) which was not being used.
  However, that star can be used to calculate least squares residuals.
  When a small part of the SFG (slide 33 in chapter 9) is taken, it can
  be viewed as a smaller subsystem. This subsystem also has its own
  residual terms. These epsilons are used as:
  \(\epsilon\)\^{}\{k-2\}\_\{k-1:k\}, where the subscript k-1:k refers
  to the subsystem, in other words, this residual is calculated with the
  signals going from time k-1 to time k. Or the subsystem is fed with
  the signals u(k) and u(k-1). The superscript refers to the the time
  index, so in words: ``\(\epsilon\)\^{}\{k-2\}\_\{k-1:k\} is the
  residual at time k-2 of the subsystem which is fed with the input
  signals u(k) and u(k-1).'' Now it can be shown that, two residuals at
  a different time, can create another residual which is equal to
  another residual in the system. This means that the calculation for
  the latter residual can be removed and as such some rotations and
  multiplications are removed. Take for example slide 35 of chapter 9.
  In this SFG it is shown that the residual fed with k-1:k at time k+1,
  together with the residual fed with k-1:k, at time k-2, can create a
  residual which is the same (upto a delay) as the residual fed with
  k-2:k at time k-3. So that circuitry can be removed (see slide 36).
\item ~
  \subsubsection{The epsilon-signals are referred to as forward and
  backward prediction errors (up to a scaling with the `cosines
  product'). Specify the forward and backward prediction problems that
  are referred to
  here.}\label{the-epsilon-signals-are-referred-to-as-forward-and-backward-prediction-errors-up-to-a-scaling-with-the-cosines-product.-specify-the-forward-and-backward-prediction-problems-that-are-referred-to-here.}
\item ~
  \subsubsection{In an acoustic echo cancellation scenario with one
  loudspeaker and two microphones (ie. where an echo contribution is to
  be cancelled in each of the two microphone signals), is it possible to
  apply a QRD-LSL-algorithm? Sketch an efficient realization (clearly
  defining input and output
  signals).}\label{in-an-acoustic-echo-cancellation-scenario-with-one-loudspeaker-and-two-microphones-ie.-where-an-echo-contribution-is-to-be-cancelled-in-each-of-the-two-microphone-signals-is-it-possible-to-apply-a-qrd-lsl-algorithm-sketch-an-efficient-realization-clearly-defining-input-and-output-signals.}
\end{enumerate}

\subsubsection{}\label{section-21}

\subsubsection{\texorpdfstring{\textbf{13 augustus
2018}}{13 augustus 2018}}\label{augustus-2018}

\subsubsection{Question 1}\label{question-1-4}

\begin{enumerate}
\def\labelenumi{\arabic{enumi}.}
\item ~
  \subsubsection{What are noise transfer functions? Explain how such
  noise transfer functions are used in the analysis of quantization
  effects in filter
  implementation.}\label{what-are-noise-transfer-functions-explain-how-such-noise-transfer-functions-are-used-in-the-analysis-of-quantization-effects-in-filter-implementation.-1}

  Chapter 6: each multiplier and adder introduces noise. The transfer
  function from e{[}k{]} to the output is the noise transfer function.
  It gives more info about the influence of each noise source to the
  output. This means that given a filter realisation and assuming a
  linear model, we can lump all the noise-sources together in one
  equivalent source.
\end{enumerate}

As the quantisation introduces noise (after a mult or add), if you study
the noise you are studying the influence of that quantisation on the
complete system.

\begin{enumerate}
\def\labelenumi{\arabic{enumi}.}
\setcounter{enumi}{1}
\item ~
  \subsubsection{Do the properties of the quantization mechanism play a
  role in such
  analysis?}\label{do-the-properties-of-the-quantization-mechanism-play-a-role-in-such-analysis}

  (slide 20) Different quantizations gives different errors
  =\textgreater{} different noise. This gives different noise mean and
  variation -\textgreater{} important for the analysis
\item ~
  \subsubsection{The analysis relies on linearity (explain) whereas
  quantization is a non-linear operation. Could this be a
  contradiction?}\label{the-analysis-relies-on-linearity-explain-whereas-quantization-is-a-non-linear-operation.-could-this-be-a-contradiction}

  The analysis we use is statistical it assumes all noise sources are
  independent (white) and the quantisation error is random. Quantization
  may be non- linear, the error is linear?????

  Quantization is non-linear and therefore produces some errors which
  are known as quantization errors. These errors result in what is known
  as quantization noise. The noise itself is viewed as additive white
  gaussian noise so it is equal for all frequencies.

\begin{lstlisting}
    The filter system which is being studied is assumed to be linear time invariant (LTI), it only uses additions, multiplications, and delays. What is assumed in the noise analysis is that the noise sources from quantization have a linear relation to the output (so are added). Therefore the noise can come from a non linear source, but the analysis relies on the circuit being linear.   
\end{lstlisting}

  m In the noise analysis we assume quantisation noise to be additive,
  but this is not actually the case. You can not just add it after a
  multiplication since it is dependent on which numbers are added, this
  makes it very unpredictable (see cycle errors)
\end{enumerate}

\subsubsection{Question 2}\label{question-2-4}

\begin{enumerate}
\def\labelenumi{\arabic{enumi}.}
\item ~
  \subsubsection{In a maximally decimated (MD) filter bank aliasing
  usually occurs because of the down-sampling operation. Provide an
  intuitive explanation why perfect reconstruction (PR) is still
  possible, in spite of the downsampling and
  aliasing.}\label{in-a-maximally-decimated-md-filter-bank-aliasing-usually-occurs-because-of-the-down-sampling-operation.-provide-an-intuitive-explanation-why-perfect-reconstruction-pr-is-still-possible-in-spite-of-the-downsampling-and-aliasing.}

  If we choose our synthesis filters in a way that they restore the
  aliasing distortion, PR is possible -\textgreater{} slides 24-28. If
  we look at all filters in total we still have all the information
  available despite the aliassing.
\item ~
  \subsubsection{What are DFT modulated filter banks? What are the
  advantages of such
  FB's?}\label{what-are-dft-modulated-filter-banks-what-are-the-advantages-of-such-fbs}

  DFT modulated filter banks are uniform filter banks that are based on
  a polyphase decomposition of a prototype filter H0 and followed by an
  inverse DFT, hence all the other Hi filters are derived from only one
  prototype filter. The main advantage is that this kind of filter banks
  is that are easier to design: yodo : you only design one!!! The rest
  of the filters are shifted versions of the prototype filter. Low
  computational cost also i think\ldots{}\\
  Low computational cost because only N polyphase components ( = 1
  filter) and DFT versus N\^{}2 polyphase components ( = N filters).
\item ~
  \subsubsection{How can DFT-modulated PR filter banks be
  designed?}\label{how-can-dft-modulated-pr-filter-banks-be-designed}

  Maximally decimated:\\
\end{enumerate}

\begin{enumerate}
\def\labelenumi{\arabic{enumi})}
\item
  Design prototype H\_0(z)\\
\item
  Calculate E\_n(z) = polyphase component\\
\item
  Assuming all E can be inverted -\textgreater{} choose R

  !!! not every E stable -\textgreater{} choose E unimodular or
  paraunitary!!!\\
  Not a lot of freedom\\
  Oversampled -\textgreater{} the same but more E and R so more freedom
  in design.
\end{enumerate}

\subsubsection{Question 3}\label{question-3-5}

\begin{enumerate}
\def\labelenumi{\arabic{enumi}.}
\item ~
  \subsubsection{Explain the main trick that is used in the derivation
  of the QRD-LSL
  algorithm.}\label{explain-the-main-trick-that-is-used-in-the-derivation-of-the-qrd-lsl-algorithm.}

  LSL = RLS?? No, LSL = H9 part 2, fast algorithm for FIR RLS LSL =
  Least square lattice, RLS = Recursive least squares\\
  As most of the input vector is reused in for the next input vector
  (h{[}k{]} -\textgreater{} h{[}k-1{]}; h{[}k-1{]} -\textgreater{}
  h{[}k-2{]};\ldots..\\
  On slide 32 is shown that when a random permutation is applied to the
  input, the residues remain the same.\\
  In slide 35 it is shown that the residual for a given problem is also
  the residual for other, related problems; for example
  \(\epsilon^{k−2}_{k−1:k+1}\) may be viewed as the epsilon-output of a
  1-dimensional RLS problem, with left-hand side
  \(\epsilon^{k+1}_{k−1:k}\) and right-hand side
  \(\epsilon^{k−2}_{k−1:k}\). After a delay element this becomes
  \(\epsilon^{k-3}_{k-2:k}\). Since we now have a way to generate this
  epsilon, the other way is redundant and thus can be removed. This
  process is then repeated until the SFG on slide 37 is reached.
\end{enumerate}

Can someone explain how we can calculate the filtercoefficeint with
this? ( i found a formula but it only uses the R matrix, and with this
there is no R\ldots{}


Filter coeff are not calculated (slide 22)

\begin{enumerate}
\def\labelenumi{\arabic{enumi}.}
\setcounter{enumi}{1}
\item ~
  \subsubsection{Compare the complexity of the QRD-LSL algorithm with
  the complexity of
  LMS.}\label{compare-the-complexity-of-the-qrd-lsl-algorithm-with-the-complexity-of-lms.}

  LMS \(\rightarrow\) O(6L)-\/-\textgreater{} moet dit niet 2L
  multiplications zijn? (slide 16) en kan big O-notatie hier gebruikt
  worden?\\
  QRD-LSL \(\rightarrow\) O(24L) \(\rightarrow\) en dit moet 24*L
  multiplications zijn, niet O(24L)
\item ~
  \subsubsection{In an acoustic echo cancellation scenario with one
  loudspeaker and two microphones (ie. where an echo contribution is to
  be cancelled in each of the two microphone signals), is it possible to
  apply a QRD-LSL-algorithm? Sketch an efficient realisation (clearly
  defining input and output
  signals).}\label{in-an-acoustic-echo-cancellation-scenario-with-one-loudspeaker-and-two-microphones-ie.-where-an-echo-contribution-is-to-be-cancelled-in-each-of-the-two-microphone-signals-is-it-possible-to-apply-a-qrd-lsl-algorithm-sketch-an-efficient-realisation-clearly-defining-input-and-output-signals.}

  The input is equal for both microphone signals, but the desired output
  is different. By simply adding an extra desired signal we can reuse
  the same procedure.\\
  Do we need to calculate an epsilon for both microphone signals? And
  can we use the same R matrix for both signals?\\
  Yes different epsilons are needed, remember that epsilon is: e = d -
  w*x. So if there is another desired signal d, then there is also
  another epsilon.
\end{enumerate}

\subsubsection{Question 4: Acoustic modem
project}\label{question-4-acoustic-modem-project}

\begin{enumerate}
\def\labelenumi{\arabic{enumi}.}
\item ~
  \subsubsection{Explain how the OFDM modulation format (including a
  cyclic prefix) provides an easy equalization of the transmission
  channel. What are conditions that the transmission channel has to
  satisfy for this `cyclic prefix trick' to work
  properly?}\label{explain-how-the-ofdm-modulation-format-including-a-cyclic-prefix-provides-an-easy-equalization-of-the-transmission-channel.-what-are-conditions-that-the-transmission-channel-has-to-satisfy-for-this-cyclic-prefix-trick-to-work-properly}
\end{enumerate}

The OFDM frame together with the prefix allows for a circular
convolution. In the receiver, the L samples corresponding with the
prefix are thrown away which results in a circulant matrix. This
circulant matrix can then be factorized using IDFT/DFT( H= F\^{}-1 *
diag(H0,..,Hn-1) * F). With this, a diagonal matrix is obtained. The
resulting output Y is then simply the input X multiplied with some
scalar H(all in frequency domain), so the X can be estimated using
component wise division. The prefix length needs to be longer than the
impulse response of the channel. (slide 31-35`)

\begin{enumerate}
\def\labelenumi{\arabic{enumi}.}
\setcounter{enumi}{1}
\item ~
  \subsubsection{If a channel model is required for the equalization,
  explain how it can be derived. Is it possible to run the equalization
  without first deriving a channel
  model?}\label{if-a-channel-model-is-required-for-the-equalization-explain-how-it-can-be-derived.-is-it-possible-to-run-the-equalization-without-first-deriving-a-channel-model}

  A channel model can be acquired by using trainingframes?? The data in
  these frames is known at the receiver, so the channel model can be
  derived with a least squares estimation.\\
  By measurement, and ise slide 19
\end{enumerate}

\subsubsection{}\label{section-22}

\subsubsection{\texorpdfstring{\textbf{26 januari
2018}}{26 januari 2018}}\label{januari-2018}

\subsubsection{Question 1}\label{question-1-5}

\begin{enumerate}
\def\labelenumi{\arabic{enumi}.}
\item ~
  \subsubsection{Explain the lossless lattice realization for 4 FIR
  filters. Explain the orthogonal matrix
  transformations.}\label{explain-the-lossless-lattice-realization-for-4-fir-filters.-explain-the-orthogonal-matrix-transformations.}

  P18: we want it lossless so we want power-complementary:\\
  1= H(z)H(z\textsuperscript{-1)+Ĥ(z)Ĥ(z}-1) =\textgreater{} we only
  want rotations to conserve the energy!!!
\item ~
  \subsubsection{Sometimes scaling is necessary,
  explain.}\label{sometimes-scaling-is-necessary-explain.}

  Stability: if H(z) is very big/ small -\textgreater{} H(z\^{}-1) very
  small/big -\textgreater{} not good
\item ~
  \subsubsection{Are there any \ldots{} matrices present (matrices with
  determinant =
  cte*z\^{}x)}\label{are-there-any-matrices-present-matrices-with-determinant-ctezx}

  \ldots{} = unimodulary:\\
  The orthogonal transformation matrix and the order reduction matrix.
\end{enumerate}

\subsubsection{Question 2}\label{question-2-5}

\begin{enumerate}
\def\labelenumi{\arabic{enumi}.}
\item ~
  \subsubsection{Explain the context and need of the short time fourier
  transform
  (STFT).}\label{explain-the-context-and-need-of-the-short-time-fourier-transform-stft.}

  STFT is for example used in a short lived sine wave, where the
  frequencies mustn't be viewed from an infinite time perspective but
  from a finite one. For example when making a spectrogram, it's only
  the instantaneous frequencies that are important, so you get a
  different frequency spectrogram for each time bin.
\item ~
  \subsubsection{Explain the link to DFT modulated filter
  banks}\label{explain-the-link-to-dft-modulated-filter-banks}

  In DFT modulated FBs we shift one LPF over the frequency axis and
  obtain a set of N filters the spans all frequencies. In this context,
  I think we design a short time prototype filter and the inverse
  fourier transform operation shifts it over all time samples?? Anyone
  know if this is correct?
\end{enumerate}

\subsubsection{Question 3}\label{question-3-6}

\begin{enumerate}
\def\labelenumi{\arabic{enumi}.}
\item ~
  \subsubsection{Explain the main trick in the fast RLS
  algorithm}\label{explain-the-main-trick-in-the-fast-rls-algorithm}
\item ~
  \subsubsection{You have 2 microphones and one speaker, can the
  algorithm be used?
  How?}\label{you-have-2-microphones-and-one-speaker-can-the-algorithm-be-used-how}
\end{enumerate}

\subsubsection{\texorpdfstring{\textbf{15 januari
2018}}{15 januari 2018}}\label{januari-2018-1}

\subsubsection{Question 1}\label{question-1-6}

\begin{enumerate}
\def\labelenumi{\arabic{enumi}.}
\item ~
  \subsubsection{Formulate in your own words the Schur-Cohn stability
  criterion to test the `stability' of a
  polynomial.}\label{formulate-in-your-own-words-the-schur-cohn-stability-criterion-to-test-the-stability-of-a-polynomial.}
\end{enumerate}

For a polynomial to be stable, all the zeros have to be within the unity
circle. Instead of calculating every single root of a polynomial (which
takes a lot of effort), you can simply use the Schur-Cohn criterion.
This dictates that the polynomial is stable if your reflection
coëfficients \textbar Ki\textbar{} \textless{} 1. The reflection
coëfficients can be calculated based on your polynomial coëfficients: K0
= bL/b0 etc\ldots{}

26 jan 2017

\begin{enumerate}
\def\labelenumi{\arabic{enumi}.}
\setcounter{enumi}{1}
\item ~
  \subsubsection{Where does the Schur-Cohn test appear in the context of
  FIR filter realisation, and what is its relevance
  there?}\label{where-does-the-schur-cohn-test-appear-in-the-context-of-fir-filter-realisation-and-what-is-its-relevance-there}

  Appears in the lattice realization. Used for computing kappais from
  bis so we can be sure that all zeros lie inside the unit circle and
  are therefore stable.\\
  The zeros of a FIR filter do not need to lie inside the unit circle, a
  FIR filter is always guaranteed to be stable, so I don't think there
  is any relevance of this test for FIR filters (only for IIR filters as
  in question 1.4).

  No, different structures can be unstable even it's FIR
\end{enumerate}

26 jan 2017

\begin{enumerate}
\def\labelenumi{\arabic{enumi}.}
\setcounter{enumi}{2}
\item ~
  \subsubsection{If the Schur-Cohn test is applied to a linear-phase FIR
  filter, what will be the outcome? What does this mean in terms of
  linear-phase FIR
  realization?}\label{if-the-schur-cohn-test-is-applied-to-a-linear-phase-fir-filter-what-will-be-the-outcome-what-does-this-mean-in-terms-of-linear-phase-fir-realization}

  abs(ki) is altijd 1, dus lattice realization van FIR linear fase
  filter is onmogelijk?\\
  Yes
\end{enumerate}

26 jan 2017

\begin{enumerate}
\def\labelenumi{\arabic{enumi}.}
\setcounter{enumi}{3}
\item ~
  \subsubsection{Where does the Schur-Cohn test appear in the context of
  IIR filter realisation, and what is its relevance
  there?}\label{where-does-the-schur-cohn-test-appear-in-the-context-of-iir-filter-realisation-and-what-is-its-relevance-there}

  IIR Lattice ladder realization: Ki's are computed from the
  coefficients of the denominator\(\rightarrow\) this time the stability
  test can be used to see if the transfer function is stable
\end{enumerate}

26 jan 2017

\subsubsection{Question 2}\label{question-2-6}

\begin{enumerate}
\def\labelenumi{\arabic{enumi}.}
\item ~
  \subsubsection{What are DFT modulated filter banks? What are the
  advantages of such
  FB's?}\label{what-are-dft-modulated-filter-banks-what-are-the-advantages-of-such-fbs-1}
\item ~
  \subsubsection{How can oversampled DFT filter banks be designed? What
  are the
  advantages?}\label{how-can-oversampled-dft-filter-banks-be-designed-what-are-the-advantages}
\item ~
  \subsubsection{Explain how an 8-channel oversampled (D=4) filter bank
  can be
  realised.}\label{explain-how-an-8-channel-oversampled-d4-filter-bank-can-be-realised.}

  Example 1 in slides 29-34, chapter 12
\item ~
  \subsubsection{How can the analysis and synthesis filter of such a
  filter bank be
  derived?}\label{how-can-the-analysis-and-synthesis-filter-of-such-a-filter-bank-be-derived}

  Analysis: {[}H\_0 H\_1 H\_2 \ldots{} H\_D-1{]}\^{}T = IFFT * B(z\^{}D)
  * {[}1 z\^{}-1 z\^{}-2 \ldots{} z\textsuperscript{-(D-1){]}}T\\
  Synthesis: {[}F\_0 F\_1 F\_2 \ldots{} F\_D-1{]} = {[}z\^{}-(D-1)
  \ldots{} z\^{}-2 z\^{}-1 1{]} * C(z\^{}D) * FFT
\end{enumerate}

\subsubsection{Question 3}\label{question-3-7}

\begin{enumerate}
\def\labelenumi{\arabic{enumi}.}
\item ~
  \subsubsection{Explain the operation of the signal flow graph (SFG) of
  QRD-RLS.}\label{explain-the-operation-of-the-signal-flow-graph-sfg-of-qrd-rls.}
\item ~
  \subsubsection{Each hexagon in the signal flow graph represents an
  orthogonal transformation (2 input, 2 output). Could non orthogonal
  transformations be used in the rotational cells? If so: how? If not:
  why
  not?}\label{each-hexagon-in-the-signal-flow-graph-represents-an-orthogonal-transformation-2-input-2-output.-could-non-orthogonal-transformations-be-used-in-the-rotational-cells-if-so-how-if-not-why-not}

  No, because if you use non orthogonal transformations you would
  introduce non-zero values in the positions where you previously
  created zeros and thus it would be impossible to create the triangular
  matrix.
\item ~
  \subsubsection{Consider the linear system y{[}k{]} = a*u{[}k{]} + b*u
  {[}k-1{]}. Could QRD-RLS be used in a `system identification'
  application, where `a' and `b' are determined from observations of
  y{[}k{]} and u{[}k{]}? If so: how? If not: why
  not?}\label{consider-the-linear-system-yk-auk-bu-k-1.-could-qrd-rls-be-used-in-a-system-identification-application-where-a-and-b-are-determined-from-observations-of-yk-and-uk-if-so-how-if-not-why-not}
\item ~
  \subsubsection{Consider the non linear system y{[}k{]} =
  a*u{[}k{]}*u{[}k-1{]} + b*u{[}k-2{]}*u{[}k-2{]}. Could QRD-RLS be used
  in a `system identification' application, where `a' and `b' are
  determined from observations of y{[}k{]} and u{[}k{]}? If so: how? If
  not: why
  not?}\label{consider-the-non-linear-system-yk-aukuk-1-buk-2uk-2.-could-qrd-rls-be-used-in-a-system-identification-application-where-a-and-b-are-determined-from-observations-of-yk-and-uk-if-so-how-if-not-why-not}
\end{enumerate}

\subsubsection{}\label{section-23}

\subsubsection{\texorpdfstring{\textbf{16 augustus
2017}}{16 augustus 2017}}\label{augustus-2017}

\subsubsection{Question 1}\label{question-1-7}

\begin{enumerate}
\def\labelenumi{\arabic{enumi}.}
\item ~
  \subsubsection{What are noise transfer functions? Explain how such
  noise transfer functions are used in the analysis of quantization
  effects in filter
  implementation.}\label{what-are-noise-transfer-functions-explain-how-such-noise-transfer-functions-are-used-in-the-analysis-of-quantization-effects-in-filter-implementation.-2}
\item ~
  \subsubsection{Do the properties of the quantization mechanism play a
  role in such
  analysis?}\label{do-the-properties-of-the-quantization-mechanism-play-a-role-in-such-analysis-1}
\item ~
  \subsubsection{The analysis relies on linearity (explain) whereas
  quantization is a non-linear operation. Could this be a
  contradiction?}\label{the-analysis-relies-on-linearity-explain-whereas-quantization-is-a-non-linear-operation.-could-this-be-a-contradiction-1}
\end{enumerate}

\subsubsection{Question 2}\label{question-2-7}

\begin{enumerate}
\def\labelenumi{\arabic{enumi}.}
\item ~
  \subsubsection{What are DFT modulated filter banks? What are the
  advantages of such
  FB's?}\label{what-are-dft-modulated-filter-banks-what-are-the-advantages-of-such-fbs-2}
\item ~
  \subsubsection{How can maximally decimated DFT filter banks be
  designed?}\label{how-can-maximally-decimated-dft-filter-banks-be-designed}
\item ~
  \subsubsection{How can oversampled DFT filter banks be
  designed?}\label{how-can-oversampled-dft-filter-banks-be-designed}
\end{enumerate}

\subsubsection{Question 3}\label{question-3-8}

\begin{enumerate}
\def\labelenumi{\arabic{enumi}.}
\item
  What is a rank-1 update? How does it define the complexity?
\item ~
  \subsubsection{Residuals: explain a priori and a posteriori residuals
  of
  SFG}\label{residuals-explain-a-priori-and-a-posteriori-residuals-of-sfg}

  The a priori residual is the difference between the desired signal and
  the filtered input, using the filter coefficients of the previous
  iteration.\\
  The a posteriori residual is the difference between the desired signal
  and the filtered input, using the filter coefficients of the current
  iteration.
\item ~
  \subsubsection{Is it useful to use two exponential weighting factors
  if the output and input are very different: static input and non
  static output (for example)? (something like
  this)}\label{is-it-useful-to-use-two-exponential-weighting-factors-if-the-output-and-input-are-very-different-static-input-and-non-static-output-for-example-something-like-this}

  \subsubsection{I don't think so: you're supposed to use the same
  weighting factor for both the desired (input) and the received
  (output) signals (ch.8 slide 37) else the LS solution won't calculate
  the filter coëfficient
  correctly}\label{i-dont-think-so-youre-supposed-to-use-the-same-weighting-factor-for-both-the-desired-input-and-the-received-output-signals-ch.8-slide-37-else-the-ls-solution-wont-calculate-the-filter-couxebfficient-correctly}
\end{enumerate}

\subsubsection{Question 4 (Acoustic modem
project)}\label{question-4-acoustic-modem-project-1}

\begin{enumerate}
\def\labelenumi{\arabic{enumi}.}
\item ~
  \subsubsection{Explain how the OFDM modulation format (include cyclic
  prefix) provides easy equalization of the transmission channel. What
  conditions are to be met for the cyclic
  prefix?}\label{explain-how-the-ofdm-modulation-format-include-cyclic-prefix-provides-easy-equalization-of-the-transmission-channel.-what-conditions-are-to-be-met-for-the-cyclic-prefix}
\item ~
  \subsubsection{Explain how RLS is used for channel equalization in the
  acoustic modem project. What input for the adaptive filter? What is
  the desired output signal? What is the order of the filter? (awnser:
  1, for every
  carrier/subband)}\label{explain-how-rls-is-used-for-channel-equalization-in-the-acoustic-modem-project.-what-input-for-the-adaptive-filter-what-is-the-desired-output-signal-what-is-the-order-of-the-filter-awnser-1-for-every-carriersubband}
\end{enumerate}

Does anyone have an idea?\\
I think we used NLMS in week 7 (and not RLS) . And the input of the
adaptive filter is the signal received at te microphone
(x{[}k{]}*h{[}k{]}+n{[}k{]}). The desired output is the output of the
decision device which takes the output of the adaptive filter as
input.\\
The order of the filter is 1, you try to find an estimation for every
subband carrier channel.

\subsubsection{}\label{section-24}

\subsubsection{\texorpdfstring{\textbf{16 jan
2017}}{16 jan 2017}}\label{jan-2017}

Question 1

\begin{enumerate}
\def\labelenumi{\arabic{enumi}.}
\tightlist
\item
  Formulate in your own words the Schur-Cohn stability criterion to test
  the `stability' of a polynomial.
\end{enumerate}

We have a polynomial B(z) and want to know whether all zeros of B(z) lie
inside the unit circle. If B(z) = sum\{i=0..L\}(b\_i*z\^{}-i), then
calculate kappa\_0 as b\_L/b\_0. Calculate all new coefficients
b'\_0..b'\_\{L-1\}. Now you are ready to calculate kappa\_1 as
b'\_\{L-1\}/b'0. If all L Kappa\_i satisfy \textbar K\_i\textbar{}
\textless{} 1, the zeros lie inside the unit circle, and thus, the
polynomial is stable.\\
(+ see p87 sl 173)

\begin{enumerate}
\def\labelenumi{\arabic{enumi}.}
\setcounter{enumi}{1}
\tightlist
\item
  Where does the Schur-Cohn test appear in the context of FIR filter
  realisation, and what is its relevance there?
\end{enumerate}

It appears in lattice ladder realisation. Its relevance is mainly to
calculate the different kappa\_i to realise the filter.\\
(should be lattice realization instead of lattice ladder I think)

\begin{enumerate}
\def\labelenumi{\arabic{enumi}.}
\setcounter{enumi}{2}
\tightlist
\item
  Where does the x test appear in the context of IIR filter realisation,
  and what is its relevance there?\\
  Schur-cohn is used to compute the reflection coefficients of a
  lattice-ladder realization\\
\end{enumerate}

\begin{itemize}
\tightlist
\item
  the reflection coefficients are calculated from the ai of the IIR
  filter (the denominator), because all Ki \textless{} 1 (sine of
  theta), this ensures the stability of the IIR filter
\end{itemize}

-\textgreater{} correct?? I think so Yes, this is why you cannot make a
lattice realisation of an unstable IIR

\begin{enumerate}
\def\labelenumi{\arabic{enumi}.}
\setcounter{enumi}{3}
\tightlist
\item
  If the Schur-Cohn test is applied to a linear-phase FIR filter, what
  will be the outcome? What does this mean in terms of linear-phase FIR
  realization?
\end{enumerate}

Linear phase filters have symmetrical impulse responses, so the first
kappa will be bL/b0, which will be 1 or -1, depending on even or odd
symmetry. K\_0 = +-1. Then, when calculating the ``next-generation''
coefficients b1' to bL',\\
A zero will appear in the denominator, because of kappa being 1.\\
Is it possible to build it in lattice realization if kappa = 1? No\\
This means it is - at least with this design procedure - impossible to
design a linear phase lattice FIR filter. \(\rightarrow\) better? And
the next generation coefficients are infinity I think? Divided by 0
\(\rightarrow\)it will be 0/0 for even symmetry and inf for odd symmetry
I think. Indeed, but what does that mean? Just you can't realize with
this realization? Marcske remained vague in his slides, but I guess we
don't know how to or if it is realisible. I think other realizations are
possible, but not the lattice, like direct form. Exactly, no Lattice,
yes direct form, transposed form or lossless lattice!

\begin{enumerate}
\def\labelenumi{\arabic{enumi}.}
\setcounter{enumi}{4}
\tightlist
\item
  Use the Schur-Cohn stability test to derive the stability conditions
  for a 2nd order polynomial A(z) = (1 + a1*z\^{}-1 + a2*z\^{}-2)
\end{enumerate}

Kappa\_0 = a2/a0 = a2\\
A0' = a0;\\
a1' = (a1-kappa0*a1)/(1-kappa0\^{}2) = (a1-a1a2)/(1-a2\^{}2)\\
Kappa\_1 = a'1/a'0 = a'1 = (a1-a1a2)/(1-a2)\^{}2\\
For the system to be stable, all kappa should have a modulus smaller
than 1:\\
\textbar k0\textbar{} \textless{} 1 ⇔ \textbar a2\textbar{} \textless{}
1\\
\textbar k1\textbar{} \textless{} 1 ⇔
\textbar(a1-a1a2)/(1-a2)\^{}2\textbar{} \textless{} 1 ⇔
\textbar a1(1-a2)\textbar{} \textless{} \textbar(1-a2)\^{}2\textbar{} ⇔
\textbar a1\textbar{} \textless{} \textbar1-a2\textbar{}\\
If those conditions are met, every zero of the polynomial will lie
inside the unit circle.\\
Note that there is no criterion for a0, as this is already set to 1 in
the given polynomial.\\
Also note that this is a strict criterion in that when it's not met, the
poles will not lie strictly inside the unit circle!\\
I think there was a mistake in the denominator above: (1-a2²) changed to
(1-a2)². If I keep the first option then my result is
\textbar a1\textbar{} \textless{} \textbar1+a2\textbar{}\\
See also chapter 6 slide 12

Question 2\\
Consider an oversampled (analysis + synthesis) filter bank.

\begin{enumerate}
\def\labelenumi{\arabic{enumi}.}
\tightlist
\item
  Explain how a condition for alias-free operation is derived, based on
  polyphase representations of the analysis and synthesis filters.
\end{enumerate}

Chapter-11 slides 34-36

\begin{enumerate}
\def\labelenumi{\arabic{enumi}.}
\setcounter{enumi}{1}
\item
  By using this condition, derive an expression for the `distortion'
  function as a function of the analysis filters H(z) and synthesis
  filters F(z).
\item
  Explain how a condition for perfect reconstruction is derived, again
  based on polyphase representations of the analysis and synthesis
  filters.
\end{enumerate}

Question 3

\begin{enumerate}
\def\labelenumi{\arabic{enumi}.}
\item
  Explain the operation of the signal flow graph (SFG) of QRD-RLS.\\
\item
  Explain how the QRD-LSL (least-squares-lattice) algorithm is derived.
  What is the meaning of the epsilon-notation?\\
  Meaning of the epsilon-notation: The SFG can be seen as a collection
  of nested RLS problems. (e.g.~when you remove the last 3 rows on the
  figure of the SFG from the slides, you can view the remaining 2 rows
  as a SFG that solves 4 RLS problems simultaneously. They all have the
  same U-matrix as left hand side (u{[}k{]} \& u{[}k-1{]}). The right
  hand side for each RLS problem is u{[}k-2{]}, u{[}k-3{]} and
  u{[}k-4{]}. The signals between row 2 and row 3 can thus be seen as
  residues of an RLS problem (= epsilons). The same is true for all the
  other rows.

  On slide 32 is shown that when a random permutation is applied to the
  input, the residues remain the same.\\
  In slide 35 it is shown that the residual for a given problem is also
  the residual for other, related problems; for example
  \(\epsilon^{k−2}_{k−1:k+1}\) may be viewed as the epsilon-output of a
  1-dimensional RLS problem, with left-hand side
  \(\epsilon^{k+1}_{k−1:k}\) and right-hand side
  \(\epsilon^{k−2}_{k−1:k}\). After a delay element this becomes
  \(\epsilon^{k-3}_{k-2:k}\). Since we now have a way to generate this
  epsilon, the other way is redundant and thus can be removed. This
  process is then repeated until the SFG on slide 37 is reached.
\item
  The epsilon-signals are referred to as forward and backward prediction
  errors (up to a scaling with the `cosines product'). Specify the
  forward and backward prediction problems that are referred to here.

  A posteriori residual = d\_k - (u\_k)\^{}T*w\_LS{[}k{]}\\
  = epsilon*prod\_\{i=1\}\^{}\{L+1\}(cos(teta\_i)) t

  A priori residual = d\_k - (u\_k)\^{}T*w\_LS{[}k-1{]}\\
  = epsilon/prod\_\{i=1\}\^{}\{L+1\}(cos(teta\_i))
\end{enumerate}

\subsubsection{\texorpdfstring{\textbf{28 jan
2016}}{28 jan 2016}}\label{jan-2016}

Question 1

\begin{enumerate}
\def\labelenumi{\arabic{enumi}.}
\tightlist
\item
  how is the IIR lattice related to direct form FIR\\
  A lattice-ladder with all theta's 0 (this means H(z) is FIR) becomes a
  direct-form FIR\\
\item
  why do you need to scale the TF for Lossless Lattice realisation?\\
  If H(z)*H(z\^{}-1) \textgreater{} 1 the magic equation isn't possible
  I think?\\
  Anyone a further explanation on this? Do we have to prove that the
  second term of the magic equation can never be negative?\\
\item
  how is the schur-kohn stability test related to the implementation of
  lattice FIR filters?\textbackslash{}
\end{enumerate}

Question 2

\begin{enumerate}
\def\labelenumi{\arabic{enumi}.}
\tightlist
\item
  explain how PR is achieved given the AA condition of a DFT modulated
  FB
\end{enumerate}

The requirements for perfect reconstruction are:

\begin{enumerate}
\def\labelenumi{\arabic{enumi})}
\tightlist
\item
  Alias-free \(\rightarrow\) alias transfer function A(z) = 0\\
\item
  DIstortion-free \(\rightarrow\) T(z) = z\^{}(-delta) i.e.~T(z) is a
  pure delay
\end{enumerate}

Based on the R(z)*E(z) product, a necessary and sufficient condition for
perfect reconstruction is that R(z)*E(z) = z\^{}(-delta)*I\_N (see page
192).\\
Is AA standing for anti-aliasing? I think so :)

\begin{enumerate}
\def\labelenumi{\arabic{enumi}.}
\setcounter{enumi}{1}
\tightlist
\item
  what are unimodulary matrices? how are they used in DFT modulated FB?
\end{enumerate}

A unimodulary matrix is a matrix of which the determinant = constant *
z\^{}(d). We use them in filter banks because we need to make sure that
if we design a FIR analysis filter, that the synthesis filter (which is
\textasciitilde{} the inverse of the analysis filter) is also a FIR
filter.

\begin{enumerate}
\def\labelenumi{\arabic{enumi}.}
\setcounter{enumi}{2}
\tightlist
\item
  what are paraunitary matrices? how are they used in DFT modulated FB?
\end{enumerate}

Paraunitary matrices are unimodulary matrices that are built from
unitary E\_l matrices. The definition according to google is that if U
is paraunitary, then U(z)*U(z\^{}-1) = I. They are cool because they
have some properties listed at p.~197 (sl. 394).\\
Question 3

\begin{enumerate}
\def\labelenumi{\arabic{enumi}.}
\tightlist
\item
  what is the difference between optimal filters and RLS filters?
\end{enumerate}

Optimal filtering is based on given statistical information. In RLS
filtering this information is not given and is estimated based on real
samples.

\begin{enumerate}
\def\labelenumi{\arabic{enumi}.}
\setcounter{enumi}{1}
\tightlist
\item
  explain how standard RLS is a special case of standard KF
\end{enumerate}

Slide 336

\begin{enumerate}
\def\labelenumi{\arabic{enumi}.}
\setcounter{enumi}{2}
\tightlist
\item
  explain how the algorithm of RLS can be seen in the algorithm of KF
\end{enumerate}

\subsubsection{\texorpdfstring{\textbf{12 Jan
2016}}{12 Jan 2016}}\label{jan-2016-1}

Question 1

\begin{enumerate}
\def\labelenumi{\arabic{enumi}.}
\tightlist
\item
  What are 'zero-input limit cycle oscillations? Where and how do these
  appear?\\
  Oscillations in the absence of input, they are unwanted and only
  appear in IIR filters. Solution: MAGNITUDE truncation + good choice of
  filter realization\\
\item
  Explain how from a given frequency domain specification (e.g.~low-pass
  filter characteristic) a filter can be designed and realized that is
  guaranteed to be free of limit cycle oscillations. Explain (briefly)
  different steps and options in the design process.
\end{enumerate}

First choice: FIR/IIR \(\rightarrow\) FIR is always free of limit cycle
oscillations.\\
Second choice: filter realization, FIR choice doesn't matter for limit
cycle oscillations\\
IIR choice of lossless lattice realization or latice-ladder realization
with ¶¶¶magnitude truncation guarantees no limit cycle oscillations.
With other realizations/quantization limit cycle oscillations are
possible.\\
Question 2

\begin{enumerate}
\def\labelenumi{\arabic{enumi}.}
\tightlist
\item
  What are oversampled DFT-modulated filter banks? What are particular
  advantages of such filter banks?
\end{enumerate}

Oversampled DFT-modulated filter banks are DFT-modulated filter banks in
which the \#channels \textgreater{} decimation. This leads to more
design freedom in the polyphase components. In maximally decimated there
is 0 design freedom in the polyphase components.

\begin{enumerate}
\def\labelenumi{\arabic{enumi}.}
\setcounter{enumi}{1}
\tightlist
\item
  Consider an 8-channel DFT-modulated filter bank with 4-fold
  down-sampling. How can the analysis bank and synthesis bank be
  realized efficiently?
\end{enumerate}

Explain slide 414-417

\begin{enumerate}
\def\labelenumi{\arabic{enumi}.}
\setcounter{enumi}{2}
\tightlist
\item
  \ldots And then how can PR be obtained in this case (N=8,D=4)?
\end{enumerate}

Explain slide 418-421\\
Question 3

\begin{enumerate}
\def\labelenumi{\arabic{enumi}.}
\tightlist
\item
  Explain the operation of the SFG (signal-flow graph) of QRD-RLS.
\end{enumerate}

Does anyone know a good video/explanation on the internet? You can find
some explanation on dropbox, or on the DSP site (in ghostview format), I
think it was used in the past years, some kind of course written by
Marcy

\begin{enumerate}
\def\labelenumi{\arabic{enumi}.}
\setcounter{enumi}{1}
\tightlist
\item
  A hexagon in the SFG represents a transformation with a 2-by-2
  orthogonal matrix. Would it be possible to use non-orthogonal
  transformations (to introduce the zeros in specific positions of the
  input vector)? How/why not?\\
  No, because if you use non orthogonal transformations you would
  introduce non-zero values in the positions where you previously
  created zeros and thus it would be impossible to create the triangular
  matrix.\\
\item
  What is residual extraction? How is it realized in the SFG of QRD-RLS?
  For which application(s) is it relevant?\\
  Sometimes the actual filter coefficients aren't important, but only
  the residual is.
\end{enumerate}

A posteriori residual = d\_k - (u\_k)\^{}T*w\_LS{[}k{]}\\
= epsilon*prod\_\{i=1\}\^{}\{L+1\}(cos(teta\_i))\\
By simply multiplying all the cos(teta\_i)'s and the final epsilon at
the end of the RLS problem we can easily extract the error signal. This
is mainly used in applications such as acoustic echo cancellation where
only the error is important. Because the actual filter coefficients
aren't important, we can skip the final step of calculating them through
back substitution.

\subsubsection{\texorpdfstring{\textbf{11 Jan
2016}}{11 Jan 2016}}\label{jan-2016-2}

Question 1e

\begin{enumerate}
\def\labelenumi{\arabic{enumi}.}
\tightlist
\item
  Formulate in your own words the Shur-Cohn stability criterion to test
  the `stability' of a polynomial.\\
\item
  Where does the Shur-Cohn test appear in the context of FIR filter
  realisation, and what is its relevance there?\\
\item
  If the Schur-Cohn test is applied to a linear-phase FIR filter, what
  will be the outcome? What does this mean in terms of linear-phase FIR
  realization?\\
\item
  Where does the Shur-Cohn test appear in the context of IIR filter
  realization and what is its relevance there.
\end{enumerate}

Question 2

\begin{enumerate}
\def\labelenumi{\arabic{enumi}.}
\tightlist
\item
  What are oversampled DFT-modulated filter banks? What are particular
  advantages of such filter banks?\\
\item
  Consider an 8-channel DFT-modulated filter bank with 4-fold
  downsampling. How can the analysis bank and the synthesis bank be
  realized efficiently?\\
\item
  \ldots{} And then how can perfect reconstruction be obtained in this
  case. (N=8, D=4)
\end{enumerate}

Question 3

\begin{enumerate}
\def\labelenumi{\arabic{enumi}.}
\tightlist
\item
  Explain how a square-root Kalman Filter algorithm is derived. Estimate
  (as `O(xx)') the resulting computational complexity per sampling
  interval.\\
\item
  Explain how the Kalman Filter extends Recursive Least Squares
  parameter oversaFIRestimation.
\end{enumerate}

\subsubsection{\texorpdfstring{\textbf{13 Jan 2015
(am)}}{13 Jan 2015 (am)}}\label{jan-2015-am}

\begin{enumerate}
\def\labelenumi{\arabic{enumi}.}
\tightlist
\item
  Question 1:

  \begin{enumerate}
  \def\labelenumii{\arabic{enumii}.}
  \tightlist
  \item
    What are noise transfer functions? Explain how such noise transfer
    functions are used in the analysis of quantisation effects in filter
    implementations\\
  \item
    Do the properties of the quantisation mechanism play a role in such
    analysis?\\
  \item
    The analysis relies on linearity, whereas quantisation is a
    non-linear operation. Could this be a contradiction?\\
  \end{enumerate}
\item
  Question 2:

  \begin{enumerate}
  \def\labelenumii{\arabic{enumii}.}
  \tightlist
  \item
    Explain how polyphase decompositions are used for a convenient
    representation of maximally decimated analysis and synthesis banks.
    Explain in detail how the polyphase matrices are constructed.\\
  \item
    Explain how polyphase decompositions are used for a convenient
    representation of oversampled analysis and synthesis banks. Explain
    in detail how the polyphase matrices are constructed.\\
  \item
    Develop a similar polyphase decomposition based on representation
    for transmultiplexers, where the number of channels is equal to the
    up-/downsampling factor. We haven't seen transmultiplexers!\\
  \end{enumerate}
\item
  Question 3:

  \begin{enumerate}
  \def\labelenumii{\arabic{enumii}.}
  \tightlist
  \item
    Explain the main trick that is used in the derivation of the QRD-LSL
    algorithm.\\
  \item
    Compare the complexity of the QRD-LSL algorithm with the complexity
    of LMS.\\
  \item
    In an acoustic echo cancellation scenario with one loudspeaker and
    two microphones (ie. where an echo contribution is to be cancelled
    in each of the two microphone signals), is it possible to apply a
    QRD-LSL-algorithm? Sketch an efficient realisation (clearly defining
    input and output signals).
  \end{enumerate}
\end{enumerate}

\subsubsection{\texorpdfstring{\textbf{13 Jan 2015
(pm)}}{13 Jan 2015 (pm)}}\label{jan-2015-pm}

\begin{lstlisting}
See wiki
\end{lstlisting}

\subsubsection{\texorpdfstring{\textbf{12 januari
2015}}{12 januari 2015}}\label{januari-2015}

\begin{enumerate}
\def\labelenumi{\arabic{enumi}.}
\tightlist
\item
  Ruis

  \begin{enumerate}
  \def\labelenumii{\arabic{enumii}.}
  \tightlist
  \item
    Wat zijn noise transfer functions en waarvoor worden ze gebruikt?\\
  \item
    Heeft het quantisatieproces bij ruisanalyse effect op die analyse?\\
  \item
    quantisatie is een niet-lineair proces, de analyse veronderstelt
    lineariteit, dit lijkt een tegenspraak. Leg uit.\\
  \end{enumerate}
\item
  Filterbanken

  \begin{enumerate}
  \def\labelenumii{\arabic{enumii}.}
  \tightlist
  \item
    Leg in detail de vereisten uit voor de afwezigheid van aliasing bij
    polyfase gedecomposeerde filterbanken.\\
  \item
    Hetzelfde maar nu voor PR in plaats van AA.\\
  \item
    Leid equivalente eisen af voor een transmux.\\
  \end{enumerate}
\item
  Adaptieve filters (QRD updating)

  \begin{enumerate}
  \def\labelenumii{\arabic{enumii}.}
  \tightlist
  \item
    Leg de signal flow graph voor QRD RLS updating.\\
  \item
    We gebruiken hiervoor orthogonale transformaties, mag dit ook met
    niet-orthogonale transformaties? Waarom (niet)?
  \end{enumerate}
\end{enumerate}

\subsubsection{\texorpdfstring{\textbf{14 januari
2014}}{14 januari 2014}}\label{januari-2014}

\begin{enumerate}
\def\labelenumi{\arabic{enumi}.}
\tightlist
\item
  We willen een filter ontwerpen zonder limit cycles. Hoe doen we dit?
  Overloop het ontwerpproces en licht de keuzes toe die gemaakt moeten
  worden.

  \begin{enumerate}
  \def\labelenumii{\arabic{enumii}.}
  \tightlist
  \item
    Wat is een oversampled DFT-modulated filter bank? Wat zijn de
    voordelen hiervan?\\
  \item
    Gegeven een DFT-modulated FB met 8 kanalen en 4-voudige decimatie.
    Hoe kunnen we op een efficiënte manier de analyse- en synthesebank
    realiseren? Hoe bekomen we een FIR unimodular PR FB? Hoe bekomen we
    een FIR paraunitary PR FB?\\
  \end{enumerate}
\item
  Acoustic echo cancellation met (N)LMS

  \begin{enumerate}
  \def\labelenumii{\arabic{enumii}.}
  \tightlist
  \item
    Wat is de invloed van het far-endsignaal op de convergentie? Wat zou
    een optimale keuze voor het far-endsignaal zijn?\\
  \item
    Wat is de invloed van het near-endsignaal op de convergentie? Is een
    on-offsignaal te verkiezen, of een continu signaal (bv. muziek)?\\
  \item
    Vergelijk de complexiteit van LMS en NLMS.
  \end{enumerate}
\end{enumerate}

\subsubsection{\texorpdfstring{\textbf{Januari
2013}}{Januari 2013}}\label{januari-2013}

\begin{enumerate}
\def\labelenumi{\arabic{enumi}.}
\tightlist
\item
  Question 1

  \begin{enumerate}
  \def\labelenumii{\arabic{enumii}.}
  \tightlist
  \item
    What are noise transfer functions and how are these used to analyse
    the effect of quantization errors in filter realizations? Why is it
    that in such analysis filters, quantization noise sources can
    sometimes be lumped into a single noise source, for specific filter
    realizations? (use formulas)\\
  \item
    Are the assumptions under which such analysis is performed fully
    justified? (provide (counter-) examples)\\
  \item
    What are noise transfer functions like in a FIR lossless lattice
    realization?\\
  \end{enumerate}
\item
  Question 2

  \begin{enumerate}
  \def\labelenumii{\arabic{enumii}.}
  \tightlist
  \item
    Explain why and how polyphase decompositions are used to derive
    conditions for perfect reconstruction in maximally decimated filter
    banks.\\
  \item
    Consider a 6-channel filter bank with 3-fold up- and downsampling?.
    How can polyphase decompositions be used to derive conditions for
    perfect reconstruction? What is the condition for alias free
    operation? What is the condition for perfect reconstruction?\\
  \item
    Consider a 6-channel filter bank with 6-fold up- and downsampling,
    which provides perfect reconstruction. Does the same filter bank
    also provide perfect reconstruction under 3-fold up- and
    downsampling?\\
  \end{enumerate}
\item
  Question 3

  \begin{enumerate}
  \def\labelenumii{\arabic{enumii}.}
  \tightlist
  \item
    What is exponential weighting and why is it used in adaptive
    filtering? Explain how the LS cost function is modified by the
    exponential weighting.\\
  \item
    What is practical and advantageous about such exponential
    weighting?\\
  \item
    Explain how exponential weighting is included/appears in QRD-RLS.\\
  \item
    In an acoustic echo cancellation application, if the statistics of
    the far end signal are highly time-varying, would this require a
    larger/smaller exponential weighting factor?\\
  \item
    In an acoustic echo cancellation application, if the microphone
    records (lots of) background noise, next to echo and near-end
    speech, would this require a larger/smaller weighting factor?
  \end{enumerate}
\end{enumerate}

\subsubsection{\texorpdfstring{\textbf{Januari
2013}}{Januari 2013}}\label{januari-2013-1}

\begin{enumerate}
\def\labelenumi{\arabic{enumi}.}
\tightlist
\item
  derive lossless lattice FIR,

  \begin{enumerate}
  \def\labelenumii{\arabic{enumii}.}
  \tightlist
  \item
    what is paraunitic transfer function?\\
  \item
    Is lossless lattice paraunitair?\\
  \item
    What is unimodular? lossless lattice unimodular?\\
  \end{enumerate}
\item
  derivation of `overlap-save'. and relate to DFT oversampled
  filterbank\\
\item
  Discuss SFG QRD-RLS. deduce complexity and compare it with NLMS.
  extraction of residue, (additional question: why important with
  adaptive filter?). QRD RLS with or without residue extraction useful
  or not with adaptive filter?
\end{enumerate}

\emph{De vragen hieronder werden toegevoegd op het oude VTK
vakwikisysteem en werden later overgezet naar het huidige systeem}

\subsubsection{\texorpdfstring{\textbf{Januari
2012}}{Januari 2012}}\label{januari-2012}

\begin{enumerate}
\def\labelenumi{\arabic{enumi}.}
\tightlist
\item
  Leg het gebruik van minimum least square optimisation uit bij het
  ontwerp van een FIR filters.

  \begin{enumerate}
  \def\labelenumii{\arabic{enumii}.}
  \setcounter{enumii}{1}
  \tightlist
  \item
    Leg het ontwerp op basis van window-functies uit. Welke window zijn
    `goede' windows die hiervoor in aanmerking komen? Waarom?\\
  \item
    Leg uit hoe het ontwerp op basis van windows gelinkt is aan het
    ontwerp op basis van minimum least square optimisation. Welk window?
    Welke gewichtsfunctie?\\
  \item
    Leg uit hoe OFDM (met cyclische prefix) een handig mechanisme heeft
    voor channel equalisation. Opdat dit zou werken moet het kanaal aan
    verschillende voorwaarden voldoen. Dewelke?\\
  \item
    OFDM kan ook voorgesteld worden als een transmuliplexer. Wat zijn
    dan de analyse/synthese filters (formules)? Aantal kanalen? Wat zijn
    de upsample/downsample factoren en wat is hun betekenis?\\
  \item
    Hoe wordt het QRD-LSL algoritme afgeleid van het QRD-RSL
    algoritme?\\
  \item
    Wat is residu-extractie? Hoe wordt dit gebruikt in acoustic echo
    cancellation?\\
  \item
    Hoe zou je dit implementeren voor het geval van 2 micro's en 1
    luidspreker?
  \end{enumerate}
\end{enumerate}

\subsubsection{\texorpdfstring{\textbf{Januari
2011}}{Januari 2011}}\label{januari-2011}

\begin{enumerate}
\def\labelenumi{\arabic{enumi}.}
\tightlist
\item
  Lattice ladder van IIR (dus ook die prove it)

  \begin{enumerate}
  \def\labelenumii{\arabic{enumii}.}
  \setcounter{enumii}{1}
  \item
    hoe weet je zeker dat \textbar a4\textbar{} \textless{} 1\\
  \item
    wat als alle a-coeff gelijk zijn aan nul. Is H\textasciitilde{} dan
    nog altijd een allpass?\\
  \item
    Leg uit SFTF\\
  \item
    Leg de link met DFT-gemod filterbanken

    \begin{enumerate}
    \def\labelenumiii{\arabic{enumiii}.}
    \tightlist
    \item
      Wat is de betekenis van het aantal kanalen?\\
    \item
      Wat is de betekenis van de decimatiefactor?\\
    \item
      Wat als de vensterlengte groter is dan het aantal kanalen?\\
    \end{enumerate}
  \item
    Welke invloed heeft perfecte reconstrueerbaarheid op het ontwerp van
    de vensterfunctie?\\
  \item
    Wat is MMSE? Teken de kostenfunctie voor een 2-tapsfilter\\
  \item
    Wat betekent die kostenfunctie bij acoustic echo cancellation?

    \begin{enumerate}
    \def\labelenumiii{\arabic{enumiii}.}
    \tightlist
    \item
      Hoe verandert de kostenfunctie als de statistische eigenschappen
      van het far-end signaal veranderen?\\
    \item
      Welke invloed heeft het near-end signaal?\\
    \end{enumerate}
  \item
    Waarom zet men in de praktijk soms de adaptatie af als het near-end
    signaal actief wordt?
  \item
    Leg uit: Lossless Lattice FIR filter realisatie\\
  \item
    Hoe wordt hierbij H\textasciitilde\textasciitilde{} bepaald?\\
  \item
    Bepaal H\textasciitilde\textasciitilde{} en teken de realisatie
    voor: H(z) = 1/sqrt(2)*cos(theta) + z\^{}(-1) * sin (theta)\\
  \item
    Leg uit MD-PR filterbanken en wat is de vereiste voor Perfecte
    reconstructie\\
  \item
    Bespreek in detail de voorwaarde voor anti - aliasing\\
  \item
    Bespreek in detail de voorwaarde voor Perfecte reconstructie\\
  \item
    Bespreek QRD-LSL\\
  \item
    Hoe ziet de implementatie van QRD-LSL voor accoustische cancellatie
    eruit met 1 luidspreker en 1 microfoon\\
  \item
    Hoe ziet de implementatie van QRD-LSL voor accoustische cancellatie
    eruit met 2 luidsprekers en 2 microfoons\\
  \end{enumerate}
\item
  Leg uit hoe men aan de lattice ladder realisatie van een IIR filter
  komt. Wat gebeurt er wanneer \textbar a4\textbar{} \textgreater{} 1?
  Wanneer a4 gelijk is aan nul is Theta0 ook gelijk aan nul, wat als
  alle a-coëfficiënten gelijk zijn aan nul, hoe ziet H\textasciitilde{}
  er dan uit? Is H\textasciitilde{} nog steeds een APF?\\
\item
  Bijvragen: Wat voor filter is H\textasciitilde? (APF) Hoe zie je dat?
  (coëfficiënten in omgekeerde volgorde)

  \begin{enumerate}
  \def\labelenumii{\arabic{enumii}.}
  \tightlist
  \item
    Wanneer op het einde de hele procedure moet herhaald worden op het
    overgebleven stuk, gaat dit zomaar? (Structuur is hetzelfde maar wat
    ook belangrijk is dat het overgebleven deel ook een APF is)\\
  \item
    Wat is de betekenis van H\textasciitilde? (Ik heb gezegd dat die
    niet meteen voor iets dient maar dat die gewoon gebruikt wordt om
    tot die uiteindelijke structuur te komen)\\
  \item
    Wanneer alle a-coëfficiënten nul zijn, welke structuur staat er dan
    nog?\\
  \item
    Wat moet je doen wanneer \textbar a4\textbar{} \textgreater{} 1?\\
  \end{enumerate}
\item
  Leg uit wat Short Time Fourier Transform (STFT) is. Wat is het verband
  tussen STFT en DFT gemoduleerde filterbanken? Wat is de betekenis van
  de decimatiefactor? Wat betekent het aantal kanalen? Wat gebeurt er
  wanneer de window length groter is dan het aantal kanalen? Legt de eis
  om PR te hebben beperkingen op aan de keuze van de vensterfunctie?\\
\item
  Bijvragen: Vertel eens wat meer over filterbanken. (Ik was maar
  begonnen over die polyfase decompositie)

  \begin{enumerate}
  \def\labelenumii{\arabic{enumii}.}
  \tightlist
  \item
    Wat betekent frequentieresolutie?\\
  \item
    Bij die beperking op de vensterfunctie, hoe zit het met de inverse
    die je nodig hebt?\\
  \end{enumerate}
\item
  Leg de MSE kostfunctie uit bij optimale/adaptieve filtering. Schets de
  kostfunctie voor een filter met twee coëfficiënten. Wat is de
  betekenis van het minimum van de kostfunctie? Voor een echo
  cancellatie filter, wat gebeurt er met de vorm van de kostfunctie
  wanneer de statistische eigenschappen van het far-end signaal
  veranderen? Wat gebeurt er met de vorm wanneer het near-end signaal
  actief wordt? Waarom wordt praktisch adaptieve LMS uitgeschakeld
  wanneer er geen near-end signaal aanwezig is?\\
\item
  Bijvragen: Waarom hangt het optimale punt van de kostfunctie op een
  bepaalde afstand van het xy-vlak?

  \begin{enumerate}
  \def\labelenumii{\arabic{enumii}.}
  \tightlist
  \item
    Wat heeft de naam MSE te maken met de kostfunctie?\\
  \item
    Wat is /Xuu?\\
  \item
    Wat weet je over excess MSE? Is dat van toepassing wanneer de
    adaptieve LMS wordt uitgeschakeld wanneer er geen near-end signaal
    aanwezig is?\\
  \end{enumerate}
\item
  Leg uit hoe men aan de lattice ladder realisatie van een IIR filter
  komt. Wat gebeurt er wanneer \textbar a4\textbar{} \textgreater{} 1?
  Wanneer a4 gelijk is aan nul is Theta0 ook gelijk aan nul, wat als
  alle a-coëfficiënten gelijk zijn aan nul, hoe ziet H\textasciitilde{}
  er dan uit? Is H\textasciitilde{} nog steeds een APF?\\
\item
  Bijvragen: Wat voor filter is H\textasciitilde? (APF) Hoe zie je dat?
  (coëfficiënten in omgekeerde volgorde)

  \begin{enumerate}
  \def\labelenumii{\arabic{enumii}.}
  \tightlist
  \item
    Wanneer op het einde de hele procedure moet herhaald worden op het
    overgebleven stuk, gaat dit zomaar? (Structuur is hetzelfde maar wat
    ook belangrijk is dat het overgebleven deel ook een APF is)\\
  \item
    Wat is de betekenis van H\textasciitilde? (Ik heb gezegd dat die
    niet meteen voor iets dient maar dat die gewoon gebruikt wordt om
    tot die uiteindelijke structuur te komen)\\
  \item
    Wanneer alle a-coëfficiënten nul zijn, welke structuur staat er dan
    nog?\\
  \item
    Wat moet je doen wanneer \textbar a4\textbar{} \textgreater{} 1?\\
  \end{enumerate}
\item
  Leg uit wat Short Time Fourier Transform (STFT) is. Wat is het verband
  tussen STFT en DFT gemoduleerde filterbanken? Wat is de betekenis van
  de decimatiefactor? Wat betekent het aantal kanalen? Wat gebeurt er
  wanneer de window length groter is dan het aantal kanalen? Legt de eis
  om PR te hebben beperkingen op aan de keuze van de vensterfunctie?\\
\item
  Bijvragen: Vertel eens wat meer over filterbanken. (Ik was maar
  begonnen over die polyfase decompositie)

  \begin{enumerate}
  \def\labelenumii{\arabic{enumii}.}
  \tightlist
  \item
    Wat betekent frequentieresolutie?\\
  \item
    Bij die beperking op de vensterfunctie, hoe zit het met de inverse
    die je nodig hebt?\\
  \end{enumerate}
\item
  Leg de MSE kostfunctie uit bij optimale/adaptieve filtering. Schets de
  kostfunctie voor een filter met twee coëfficiënten. Wat is de
  betekenis van het minimum van de kostfunctie? Voor een echo
  cancellatie filter, wat gebeurt er met de vorm van de kostfunctie
  wanneer de statistische eigenschappen van het far-end signaal
  veranderen? Wat gebeurt er met de vorm wanneer het near-end signaal
  actief wordt? Waarom wordt praktisch adaptieve LMS uitgeschakeld
  wanneer er geen near-end signaal aanwezig is?\\
\item
  Bijvragen: Waarom hangt het optimale punt van de kostfunctie op een
  bepaalde afstand van het xy-vlak?

  \begin{enumerate}
  \def\labelenumii{\arabic{enumii}.}
  \tightlist
  \item
    Wat heeft de naam MSE te maken met de kostfunctie?\\
  \item
    Wat is /Xuu?\\
  \item
    Wat weet je over excess MSE? Is dat van toepassing wanneer de
    adaptieve LMS wordt uitgeschakeld wanneer er geen near-end signaal
    aanwezig is?
  \end{enumerate}
\end{enumerate}

\begin{center}\rule{0.5\linewidth}{0.5pt}\end{center}

\subsection{!!! Note: please correct or add more to the
answers}\label{note-please-correct-or-add-more-to-the-answers}

Last update: August 2024

\subsection{August 13 2024}\label{august-13-2024}

\textbf{Question 1}

1. Chapter-4 p.26,27,28: Describe briefly how these frequency responses
have been generated. Explain how the comparison of these different
responses leads to conclusions.

Ans: h{[}k{]} = h\_d{[}k{]}*w{[}k{]}. Side lobes vs.~smearing of main
lobe

2. Chapter-5 p.~27: This can be seen as a filter bank without subband
processing. Give E(z) and R(z).

Ans: \textasciitilde Ch. 14 slide 19: E(z) = F*diag(1,1,1,1,0,0,0,0),
R(z) = diag(1,1,1,1,1,1,1,1)*F\^{}-1 = F-1

3. Chapter-5: Counting computational complexity of the different filter
realisations as the number of multiplications per sample, which FIR
filter realisation offers the lowest cost?

Ans: Direct form and transposed direct form

4. Chapter-6 p.13: Explain in your own words the meaning of the last
equation, and how it leads to the given conclusions.

Same as Jan 12 exam

\textbf{Question 2}

1. Chapter-8 p.18 : Draw the SFG for the case of a multi-channelinput
(as on chapter-7 p.19), but when 2 input signals instead of 3 are used
with filter coefficients w0,\ldots,w3 and w4,\ldots,w7. Also give the
LMS formula for this case.

2. Chapter-9 p.16: What is a `rank-1 update' and how does it define the
computational complexity of the standard RLS algorithm?

3. Chapter-10 p.20 (``The main trick\ldots''): Redraw (sketch) the
signal flow graph when the ``main trick'' is used to remove the column
with R13. Define the relevant epsilon-signals in the signal flow graph
(with subscripts \& superscripts). Also indicate A, B, C, D \& E.

4. Chapter-11 p.25: In the last formula, the matrices appear to cancel
each other, so it appears that this notation is redundant. Why is the
formula given like this?

The formula was given like this because the two matrices are obtained as
a result of a QR decomposition. Basically, at time step k, we have a
left hand side vector and a right hand side vector to which we are
applying a QR decomposition. It was proven on p.24 that the left hand
side vector is P-½ k/k-1 with the state space parameters at time k and
that the right hand side vector is P-½ k/k-1 xk/k-1. By applying the
same QR decomposition to both sides we will see that we get P-½ k+1/k
and P-½ k+1/k xk+1/k respectively, which are then used to get xk+1/k.
So, the formula was written like this since its two components originate
from a QR decomposition.

\textbf{Question 3}

1. Chapter-12 p.14 : Explain the statement ``PR guarantees
distortion-free desired near-end speech signal''. Where does the
near-end speech signal appear in the signal flow graph?

2. Chapter-13 p.8 : Explain why, in the formula for E(z), the vector
containing the delay elements appears as a row vector, while in the
formula for R(z), it appears as a column vector

This is because in the R(z), it is multiplied from the left while in
E(z), tit is multiplied from the right, this becomes more clear when the
filterbank is oversampled, and tha matrix dimensions need to be correct
to allow multiplication.

3. Chapter-14 p.27 (``Example-1: Define B(z4)\ldots''): Specify B(z) for
the case where N=5 and D=3

4. Chapter-14 p.27 ( `Proof is simple'): Modify this proof for N=5 and
D=3 and explain the non-trivial steps.

\textbf{Question 4 (Acoustic modem project)}

1. Chapter-3: How is the cyclic prefix removed in practice?

2. Chapter-3 p.36: What is the computational complexity (in general) of
the steps `S/P', `FFT' and `FEQ'?

3. Chapter-3 p.27 (`So this will be\ldots'): A higher-order
QAM-constellation (for instance 16-\\
QAM instead of 4-QAM) can be used to increase the number of transmitted
bits per\\
second. In a practical channel, is there a limit to the order of the
QAM-constellation?

4. Chapter-3 p.35 (`Note that\ldots'): The frequency domain channel
equalization relies on the\\
inverse (H n ) -1 . How should the channel equalization be performed
when Hn =0?

\subsection{January 30 2024}\label{january-30-2024-1}

\textbf{Question 1}\\
1. Chapter-2 p.40 (``Polyphase decomposition: Example\ldots''): Explain
how the presented\\
equations are exploited in general oversampled filter banks (=Chapters
12-13).

The polyphase decomposition splits the filter into two polyphase
components E0(z2) and E1(z2). Then, using the noble identities we can
operate the filtering process at a lower sampling rate. This is very
helpful especially when it comes to filter banks since it would allow us
to reduce complexity and increase efficiency since this would require
less computation time and fewer samples.

2. Chapter-3 p.19 (``3. Least squares estimation\ldots''). Explain the
meaning of the parameters\\
K and L, and how these have to be set. What procedure does Matlab use to
solve such a\\
least squares estimation problem?

K: number of considered samples\\
L: number of filter coefficients (filter order)

How to set them: K\textgreater L since what we want to have is an
overdetermined set of linear equations (K equations L unknowns) which we
solve using least squares estimation. In MATLAB we just do y/x.\\
3. Chapter-4 p.10 (``This is a `Quadratic Optimization'\ldots{}''):
Provide similar formulas for Type-2\\
linear phase FIR filter design (p.8), and explain all the symbols used
in these formulas.\\
4. Chapter-5 p.16 (``From (*) (p.12), it follows that\ldots''):
Generalize the given formulas for the\\
case with 3 power complementary filters (as on p.20).

5. Chapter-6 p.33: Provide a justification for the `lumping' of noise
sources (as illustrated in\\
subsequent pages) that is explicitly based on the given formulas (with
`DC-gain' and\\
`noise-gain').

They have the same gain transfer function going to the output so they
are lumped together into one noise source and their means are just added
together. (same with their variances)\\
\textbf{Question 2}\\
1. Chapter-7 p.24 (``MMSE cost function can be expanded as\ldots{}''):
How does the Wiener filter\\
formula ( wWF=(Xuu)-1Xdu ) and/or its components (Xuu and Xdu) change in
the case of a\\
`linear combiner' problem (as on p.20)? What is the computational
complexity (to solve the\\
resulting set of equations) in this case (as on p.27)?

Xuu is not necessarily Toeplitz anymore since the inputs are distinct
instead of being delayed versions of each other. Xdu also won't have any
special structure. The equation wWF = Xuu-1Xdu stays the same since it
doesn't depend on the special structures and it just what we get when we
set the gradient equal to zero.

2. Chapter-8: Explain in your own words how the characteristics of the
filter input signal (uk)\\
influence the behavior of the LMS algorithm? What is then the `ideal'
filter input signal in\\
this respect?

3. Chapter-9 p.35 (``4-by-4 example\ldots{}''): If the adaptive filter
input signal and the desired\\
output signal have different `dynamics' (for instance if the
characteristics of one are very\\
stationary, while the characteristics of the other are very
non-stationary), would it be\\
possible/useful to apply two different exponential weighting factors (a
first one in the part\\
corresponding to the input signal and a second one in the part
corresponding to the\\
desired output signal of the signal flow graph)?

4. Chapter-10 p.19 (``Theorem\ldots''): The graph also shows an equality
for the accumulated\\
product of the cosines of the rotation angles. Provide an explanation
for this equality.

The rotation cells compute bcos(theta)-asin(theta). At first the b is 1
meaning that the first rotation cell outputs cos(theta1), which is the b
for the second rotation cell, thus the output of the 2nd rotation cell
is cos(theta1)cos(theta2), etc.

5. In Chapter-11 p.24 (``relevant sub-problem is\ldots''): Explain why
the framed triangular matrix\\
and corresponding right-hand side vector in the first equation
correspond to the formulas\\
that are added in the second equation.\\
\textbf{Question 3}\\
1. Chapter-12 p.36 (``A solution is as follows\ldots''): Prove that the
conditions Fo(z)=H1(-z) and\\
F1(z)=-Ho(-z) indeed lead to alias-free operation. How is the frequency
response of Fo(z)\\
and F1(z) then related to the frequency response of Ho(z) and H1(z)?

Frequency response of F0(z) becomes the frequency response of H1(z) but
shifted by pi (same as F1 and H0 however the magnitude is inverted so F1
becomes a HP filter if Ho is LP).

If we fill in these F1 and F0 in the formula for A(z) on slide 34, we
can see that A(z) = 0. The alias transfer function is zero so we have an
alias-free filterbank\\
2. Chapter-13 p.12 (``ii) Necessary \& sufficient condition\ldots''):
Explain in your own words the\\
statement ``hence pr(z)=pure \(delay=z-\delta\), and all other
pn(z)=0''. Explain how this then leads\\
to the next formula.\\
3. Chapter-14 p.13 (``Conclusion: economy in\ldots''): Explain in your
own words the statement\\
``N filters for the price of 1''.\\
You only need to implement the polyphase components of the prototype
filter H0, and these will automatically be shited to the right
frequencies by the IDFT that comes after\\
4. Chapter-14 p.27 (``Example-1: Define B(z4)\ldots''):\\
a) Specify B(z) for the case where N=5 and D=2.\\
b) Derive the conditions for perfect reconstruction and specify when the
resulting set of equations can be solved.

\subsection{\&January 11 2024}\label{january-11-2024-1}

\textbf{Question 1}

1. Chapter-2 p36: Give the link with chapters 12, 13 and 14

2. Chapter-3 p 29: Rewrite the equation for the case that the filter
order of the FIR filter is greater than the cyclic prefix length.

3. Chapter-4 (``Filter Design''): Explain how the filter phase response
is controlled in IIR filter design, and compare this to FIR filter
design.

For IIR filters it is difficult to control the phase response because of
their non-linear behavior + the impulse response is infinitely long

For FIR filter linearity is easily achieved + impulse response is finite

4. Chapter-6: Draw a `parallel' realization of H(z) = (1+az-1)-1 +
(1+bz-1)-1 and compute the noise transfer functions. Can the noise
sources be lumped into equivalent noise sources?

5. Chapter-5, p.30: Explain why the highlighted element has to be a zero
(compared to p.~22)

Explanation on slide 31, the E interchange positions in the matrix

\textbf{Question 2}

1. Chapter-7 p24: Does the Wiener filter solution WWF = Xuu-1 * Xdu-1 or
its components change in the case of multi-channel?

It doesn't change because it is a generalized version in cascade with
the input signals to the filter being delayed versions of each other
(slide19-20)

2. Chapter-8: How can LMS be viewed as RLS, For a certain substitution
in correlation matrix (Not sure about last part)

From chapter9 slide 15-16: an LS solution at time k can be computed from
solution at time k-1. Matrix update using rank-1 matrix -\textgreater{}
computing the inverse -\textgreater{} computing the coefficients by
vector update. Basically Kalman gain vector x a priori residual, you
check the previous set of coefficients and apply that to the input to
get the next output. The value of the a priori determines the direction
of the gain vector + if it has to be updated (a=/0) or not (a=0)

\begin{lstlisting}
Textbook p63. “The original LMS algorithm is a simple one line updating algorithm, which may be derived from the RLS algorithm by ignoring the covariance matrix update and setting Xuu(k-1)^-1 to I in the updating formula for w leading to …”
\end{lstlisting}

3. Chapter-9 p38: u{[}k{]} is an all zero vector and R is of full rank.
Give the a posteriori and a priori residuals for this case.

A posteriori residual = multiplication with the product of the rotor
angles

A priori residual = division with the product of the rotor angles

If u{[}k{]} is all zero ⇨ a posteriori residual = a priori residual =
d{[}k{]}? Which makes sense since the filter can't adapt if no input is
provided

4. Chapter-10 p.20 (``The main trick\ldots''): Redraw (sketch) the
signal flow graph when the ``main trick'' is used to remove the column
with R15, R25, \ldots. Define the relevant epsilon-signals in the signal
flow graph (with subscripts \& superscripts).

5. Chapter-11 p25: Can residual extraction be added here? What do the
residuals then mean?

\textbf{Question 3}

1. Chapter-14 p.27 (``Example-1: Define B(z4)\ldots''): Specify B(z) for
the case where N=7 and D=4 and provide the corresponding proof that a
7-channel DFT-modulated filter bank is obtained with this B(z) (similar
to the proof on p.27 for N=8 and D=4).

2. ???

3. ???

4. ???

Unknown Qs:

chapter-5, p.30: Explain why the highlighted element has to be a zero
(compared to p.~22)

chapter-12, p.12: If D=4 (instead of D=3) and if the Gi's are not equal
to 1. What is the condition for alias-free operation (a general formula
is sufficient here) and what would be the resulting (linear)
``distortion function''?

Attenuation\textless1 =\textgreater{} subtracting the signals? Change in
Gi, SNR is decreased

chapter-13, p.~11: Verify for when R(z)E(z) is switched with the up
sampling instead of the down sampling

chapter-13, p.~32: For D=1, N=4, Le=3, design a simple H(z) and F(z).

\subsection{August 17 2023}\label{august-17-2023}

\textbf{Question 1}\\
1. Chapter-4 p.12 (`This leads to an equivalent (`discretized')..'):
Rewrite the quadratic optimization function as a least squares
estimation problem for an overdetermined set of equations of the form
minx \textbar\textbar A.x-b\textbar\textbar2 (i.e.~define A, x, and
b).\\
(Already in slides ?)\\
2. Chapter-5 p.8 (`FIR/3. Lattice Realization'): Explain why the special
case \textbar ki \textbar=1 with a rank-\\
deficient transformation matrix is problematic in the mathematical
derivation.\\
3. Chapter-5: Explain in your own words how a direct-form FIR is
generalized to an IIR lattice-ladder realization. Provide the expression
for H\^{}tilde (p.37) in the FIR-case.\\
4. Chapter-6 p.16 (`Example (continued)`): Explain in your own words the
relevance of the `coupled realization'.\\
\textbf{Question 2}\\
1. Chapter-8 p.27 (`Normalized LMS'): Explain in your own words the
meaning of the two\\
terms in the `specific optimization problem' that corresponds to NLMS
for the case where\\
the step size is equal to 1.\\
The problem here is represented as a cost function that needs to be
minimized. The main objective here is to minimize the a posteriori
error, which is the second term. However, assume the cost function is
only minimizing the a posteriori error with no regard to previous NLMS
solutions (aka alpha = 0). Here, we could witness high fluctuations in
NLMS solutions from one time step to the other, which results in a bad
performance in time varying environments. This is why the first term
exists: i tis here to penalize high fluctuations in NLMS solutions and
adds a dependency of the currents weights on the previous weights, thus
making sure we are getting smooth changes in weights while minimizing
the a posteriori error.\\
2. Chapter-9 p.35 (`4x4 example'): Consider the cell that has `R 24 `.
Specify (in words or by\\
means of a formula for the `epsilon-signals') all in- and outgoing
arrows, i.e.~all in- and\\
outgoing signals/parameters.\\
3. Chapter-10 p.21 (`Result = QRD Lattice..'): Are the least squares
residuals extracted with\\
the fast QRD-LSL algorithm exactly the same (or an approximation of) the
least squares\\
residuals extracted with the original (non-fast) QRD-RLS algorithm?\\
4. Chapter-11 p.23 (`A RECURSIVE implementation\ldots{}`): in your own
words the statement\\
``\ldots to require only the lower-right/lower part''.\\
To turn a triangular matrix with one extra row into a triangular matrix
again, we only need the last 2 rows.\\
\textbf{Question 3}\\
1. Chapter-12 p.40 (`Similarly, for\ldots{}'): For D\textless N (instead
of D=N) the dimension reduction of the polyphase matrices (from NxN to
DxN and NxD, as graphically illustrated in the slide) appears to reduce
the design degrees of freedom (i.e.~if D is made smaller than N, the
number of polyphase components is reduced). Does this give a
disadvantage in the perfect reconstruction design procedure? Does this
conflict with the general observation that oversampled filter bank
design is easier then maximally decimated filter bank design?\\
No, since we can adjust the orders of the FIR filters in R(z) and E(z)
such that we get a number of unknowns higher than the number of
equations. This will essentially mean that the set of equations will
have more solutions hence making the design easier. Also, in the
maximally decimated case, LR has to be very large, which corresponds to
saying that every filter in R(z) has to be an IIR filter, which could
introduce stability problems.\\
2. Chapter-13 p.19 (`PS: Inversion of\ldots{}`): Explain in your own
words the relevance of the observation that some FIR matrix transfer
functions can have an FIR inverse.\\
3. Chapter-14 p.27 (`Example1\ldots{}'): Specify B(z 3 ) for the case
where N=5 and D=3.\\
4. Chapter-14 p.27: Modify the `Proof is simple' part for the case N=5
and D=3.\\
\textbf{Question 4 (Acoustic Modem Project -- Only if this exam is a
2022-2023 retake exam)}\\
1. By including the cyclic prefix, the OFDM modulation format provides
an easy equalization\\
of the transmission channel. What are conditions that the transmission
channel has to\\
satisfy for this `cyclic prefix trick' to work properly? Are these
conditions satisfied in a\\
practical acoustic channel?\\
2. How does the distance between the transmitter (loudspeaker) and
receiver (microphone)\\
influence the transmission, or in other words what happens if
transmitter and receiver are\\
moved away from each other?\\
3. Chapter-3 p.27 (`So this will be\ldots{}`): A higher-order
QAM-constellation (for instance 16-\\
QAM instead of 4-QAM) can be used to increase the number of transmitted
bits per\\
second. In a practical channel, is there a limit to the order of the
QAM-constellation?\\
4. Chapter-3 p.35 (`Note that\ldots{}'): The frequency domain channel
equalization relies on the\\
inverse (H n ) -1 . How should the channel equalization be performed
when Hn =0?

\subsection{January 31 2023}\label{january-31-2023}

Question 1\\
1. Chapter-3: Specify the computational complexity (number of
multiplications per second) of the DMT receiver operations (only those
discussed in Chapter-3), provide (approximate) formulas that contain the
main DMT parameters (N, L, symbol rate, \ldots).\\
DMT = discrete multi-tone\\
Receiver: (see slide 36): FFT: N.log(N) calculations, Channel
equalization: N/2 calculations (see slide 26 (symbols are
duplicated(complex conj) to create real signals)), and every
N/(symbolrate) seconds this calculation needs to be done =\textgreater{}
number of multiplications per second = (N.log(N)+N/2).symbolrate/N

2. Chapter-4 p.8 (``Linear Phase FIR Filters -- Type 1\ldots Type
2\ldots''). Explain why `modulating' a Type-3 filter results in a Type-3
filter. Illustrate the statement ``BP\textless-\textgreater BP'' by
sketching the filter frequency responses. Are the two `BPs' exactly the
same?\\
I think you interpret a modulation of -1,1,-1,1\ldots{} as a shift of
pi. So it makes sense that LP \textless-\textgreater{} HP but BP
\textless-\textgreater{} BP.\\
\url{https://tomroelandts.com/articles/spectral-reversal-to-create-a-high-pass-filter}

3. Chapter-5 p.18 (``Repeated application results in `lossless lattice'
\ldots''): Explain the appearance of the multiplication by `cosq4' and
`sinq4' in this lossless lattice realization.\\
The rotation angles preserve the norm of the vectors and the power of
the input signal becuase of their orthogonality⇒ lossless. The
cos(theta4) and sin(theta4) appear because the system is a 4th order
system (order = \#delay elements)

4. Chapter-5 p.30 (``Derivation similar to p.22 (overlap-save, similar
for overlap-add) \ldots''): Explain why the highlighted element in the
first formula has to be a zero (unlike in p.21 and p.22).\\
Same as in Jan 12 exam

5. Chapter-6 p.32 (``Statistical analysis is based on the following
assumptions\ldots''): For the given example, compute the noise transfer
functions for e1 and e2. Can e1 and e2 be lumped into an equivalent
noise source? Why (not)?\\
They can be i think, like the transposed form

Question 2\\
1. Chapter-7 p.8 (``example: channel identification'') and p.14
(``example: channel equalization (training mode)''): Explain and compare
these two example applications.\\
Channel identification = to provide the mathematical model for signal
propagation\\
Channel equalization = the same but multipath propagation?

Channel identification uses system identification (i.e.~provides a model
for the channel) whereas channel equalization is a way of doing inverse
modeling (the plant which is the mobile reciever outputs to the adaptive
filter and the desired response is the initial training sequence)\\
2. Chapter-8 p.21 (``LMS analysis in a nutshell: Noisy
gradients\ldots.''): Explain the `noisy\\
gradients' effect in an acoustic echo cancellation set-up, with a
near-end speaker that is\\
sometimes active and sometimes not active.\\
Expectation implies statistical information which in practice is not
given. So you use estimations from the observed signal. These
estimations of course are not perfect so what happens is that the
gradient has an error and therefore will not be 0 at the WF solution.
Wont be zero at the instantaneous value

3. Chapter-9 p.35 (``4-by-4 example''): Give an intuitive explanation
for the fact that\\
exponential weighting can be implemented here only by adding
multiplications with l after\\
the delay operations.\\
Exponential weighting = giving less weight to older samples.\\
I would assume the weighting has to come after the process is done so
that it would not change the values?\\
See slide 30: R{[}k-1{]} and z{[}k-1{]} is multiplied with lamba
=\textgreater{} every element of R{[}k-1{]} and z{[}k-1{]} is multiplied
with lambda for the calculation of the new R{[}k{]} and z{[}k{]}

4. Chapter-10 p.13 (``Preliminaries / LS residuals are not
changed\ldots{}''): Explain in your own\\
words how this property is exploited in p.19 (``Theorem'').\\
From slide 13, we know that after random fixed permutation the LS
residual remains unchanged ⇒ the order is not affected. Similarly, in
slide 19, this is shown when the same signal is produced in 2 different
ways, one at the addition of rotational elements and one at the source
signal u. This means that the system is redundant and has high
computational complexity

5. In Chapter-11 p.24 (``Relevant sub-problem is\ldots''): If the Kalman
Filter would also have to\\
compute xk-1\textbar k (next to xk\textbar k and xk+1\textbar k ) how
would the formulas on the slide have to be\\
Adapted?\\
I assume the matrix size wll have to change, adding one more row for
sure. But for the other matrices idk\\
Matrices are expanded to include the equations at time k-1 since this is
where our relevant subproblem will come from (check p.20). The inherited
upper triangular matrix (p.24) will be inherited from k-2 so the matrix
is effectively Pk-1\textbar k-2 since we are using samples up to k-2 to
get an estimate of xk-1.\\
\textbf{Question 3}\\
1. Chapter-12 p.4 (``What we have in mind is this\ldots''): In the given
block scheme (without any\\
upsampling/downsampling), what would be the condition for perfect
reconstruction? Would\\
a set of power complementary filters provide perfect reconstruction?\\
Perfect reconstruction ⇒ IN = OUT ⇒ y{[}k{]} = u{[}k-d{]} (general
condition)\\
Based on intuition, power complementary filters should do perfect
reconstruction since we don't have downsampling/upsampling, and in that
case we would have zero aliasing whereas if we downsample/upsample, we
will have some unwanted response which is cancelled out using the
synthesis filters usually.\\
2. Chapter-13 p.18 (``Design Procedure:\ldots''): Explain in your own
words how the stability\\
issue mentioned in the last sentence is overcome by having D\textless N
instead of D=N?\\
At the D = N case, the order of R(z) Lr has to be infinitely large so it
leads to an R(z) IIR where stability is not ensured.\\
At the D \textless{} N case, the Lr has to be sufficiently large for an
underdetermined set of equations can be solved so R(z) is always FIR
therefore, there are no stability issues, as we know FIR filters ensure
stability.\\
Regard this answer, makes more sense; in the D=N case we are using the
inverse of E matrix which always leads to IIR slide 19\\
3. Chapter-13 p.33 (``Example N=4\ldots''): Indicate how the formula
changes (number of\\
unknowns versus number of equations) if the order of all the synthesis
filters is increased\\
by one (=copy the formula and then highlight where in the matrices
additional entries\\
appear).\\
We know that the number of coefficients is equal to Lr + Le -1 and the
number of equations is equal to Lr + Le. From the inequality in slide
31, we can deduce that there cant be perfect reconstruction if the
\#unknowns \textgreater= \#equations.\\
+++\\
4. Chapter-14 p.27 (``Example-1: Define B(z4)\ldots''): Specify B(z) for
the case where N=4 and\\
D=3 and provide the corresponding proof that a 4-channel DFT-modulated
filter bank is\\
obtained with this B(z) (similar to the proof on p.27 for N=8 and
D=4).\\
N=4, D=3 ⇒ N' = NxD / gcd(N,D) = 4x3 / 1 = 12\\
Adapt the equation on slide 26 for H0(z) and En'(z)\\
Adapt B(z\^{}12) matrix on slide 27, matrix size DxN\\
Adapt the equation on slide 27 for proof of B matrix\\
5. Chapter-15 p.12 (``If maximally decimated\ldots''). Provide an
interpretation of the last formula\\
by comparison with the DTFT on p.4. (What are the basis functions now?
How many basis\\
Functions?

The STFT windows certain sections of the signal at time k while DTFT
analyses the whole signal with no regards to time localization. Here the
basis functions are the impulse responses of the synthesis filters and
their shifted versions over all possible values of kbar.

\subsection{January 12 2023}\label{january-12-2023}

\textbf{Question 1}\\
1. Chapter-4 (``Filter Design''): Explain how the filter phase response
is controlled in IIR filter design, and compare this to FIR filter
design.\\
In IIR, the phase response is influenced by the poles. In the z-plane,
the angle of the complex pole determines the phase contribution to the
overall phase response. The pole angle determines how phase changes as f
increases. Poles closer to the unit circle and smaller angles induce
smaller phase changes across f while poles near the centre of the unit
circle and large angles produce abrupt changes in phase =\textgreater{}
this is from chatGPT, so use at your own risk.

2. Chapter-5 p.10 (``Repeated application results in `lattice form'
\ldots''): Explain the appearance of the multiplication by `b0' in the
lattice realization.\\
My guess is that you can't get rid of the b0 because there is a limit to
the number of order reductions you can accomplish. In p.8 it wouldn't
make sense to get rid of the z\^{}-0 term because you would be
multiplying by 0 in that case.\\
If you look at the bottom of slide 8, there are equations provided for
the new coefficients you get from doing a single iteration of the
lattice `application'. b4 is substituted for a coefficient k0 that is
common to both y{[}k{]} and y\textasciitilde{[}k{]} . It states that b0`
= b0, i.e.~b0 is unchanged. So it stands that after repeated
application, b0 is still unchanged, and thus can be applied right at the
start of the realization. (This is really just something I noticed, my
explanation is probably gibberish though)

3. Chapter-5 p.28 (``Derivation similar to p.22 (overlap-save, similar
for overlap-add) \ldots''): Explain why the highlighted element in the
first formula has to be a zero (unlike in p.21 and p.22).\\
My guess is that since you have only 4 b terms (because for 4-fold
decomposition there are 4 expressions) compared to 5 in p.22 it makes
sense that there is a 0 left over.

4. Chapter-6 p.13 (``Coefficient quantization effect on pole locations /
Higher-order systems\ldots''): Explain in your own words the meaning of
the (approximate) equation, and how it leads to the given conclusions.\\
For the Lth root, the quantisation error for the root is the summation
of L with, in the denominator, a product of the distances of the Lth
root that you are analysing and all other roots that you have. So, in a
high order polynomial you are going to have many roots (and of course
they all have to fit inside the unit circle) so the distance between
some of these poles is going to be very small. So large error, and
therefore high-order is bad. They will be closer together since there
will be less space =\textgreater{} more sensitive\\
Also, the (ai\textasciitilde{} - ai) component implies that if the
summed difference between the intended coefficients and the quantised
coefficients (the total error in the coefficients due to quantisation)
is large, the difference between the intended root and the quantised
root you get is larger. This makes sense, because if your quantization
does a poor job of representing the filter coefficients, it's reasonable
to assume your root approximations will also suffer.

5. Chapter-6 p.35 (``PS: In a direct form realization\ldots''): Explain
in detail why it is that all quantization noises can be lumped into e1
and e2. What are the corresponding noise transfer functions?\\
The quantization noises can be lumped into equivalent sources because of
the linearity of the direct form realization. It is 2 equivalent sources
because of the structure of the input/output and the position of the
adders/multiplies. We know that after each adder/multiplier there is a
noise source.\\
Not really sure what on earth he means by TF of the noise sources, but
heres my honest guess. I think the TF of ei{[}k{]} is just the filter TF
since its introduced at the same point as u{[}k{]}. By this logic
e2{[}k{]} has a TF of 1 (e.g.~nothing happens to it at the ouput)
because it's introduced at the final adder of the realization (and isn't
fed back into the filter).

\textbf{Question 2}\\
1. Chapter-7 p.19 (``PS: Can generalize FIR filter to `multi-channel FIR
filter'\ldots''): For this specific (3-input) example, provide formulas
for the (input auto-)correlation matrix, the cross-correlation vector,
and for the Wiener filter.\\
Matrix for Xuu matrix slide 23. I assume the vector ukT will become a
3x3 matric because of the 3inputs as in slide 19 ⇒ the Xuu will triple
in size? The matrix will still be symmetrical, non negative and have the
toeplitz form\\
I agree that the u\_k becomes a 3x3 matrix: u = {[} u\_0{[}k{]}
u\_1{[}k{]} u\_1{[}k{]} ; u\_0{[}k-1{]} u\_1{[}k-1{]} u\_1{[}k-1{]} ;
u\_0{[}k-2{]} u\_1{[}k-2{]} u\_1{[}k-2{]} {]}. But that means that X\_uu
= E\{u u\} still stays a 3x3-matrix.

Maybe the 3x3-matrix isn't the right option for u, but make u a
9x1-vector. Then, the X\_uu matrix will become a 9x9-matrix that is
symmetric and Toeplitz.

2. Chapter-8 p.37 (``compare to p.32-33\ldots.''): Explain in your own
words the appearance of the summation in the first formula, and the
statement (in p.36) that ``D takes the place of L+1''.

3. Chapter-9 p.26 (``QRD for LS estimation\ldots''): Explain why the ` *
` in the second formula does not play a role (in the third formula).\\
It does not play a role because when you look at the right-hand
triangular matrix, on the bottom of that is 0, since this multiplies w,
that portion that the minus sign multiplies will be 0. There is no w for
the bottom row that will minimise the cost function because there is no
w for that part.

4. Chapter-10 p.20 (``The main trick\ldots''): Redraw (sketch) the
signal flow graph when the ``main trick'' is used to remove the column
with R12. Define the relevant epsilon-signals in the signal flow graph
(with subscripts \& superscripts).\\
We always keep the diagonal elements! In this case, from column 2 we
only remove element R12. The epsilon notation shows were the triangular
matrix starts and ends in a sense. Element D \(\epsilon\) k+1 k-1:k,
element E \(\epsilon\) k k-1:k, element before the delay \(\epsilon\) k
k-1:k+1, element aftr the delay \(\epsilon\) k-1 k:k

5. In Chapter-11 p.10 (``PPS: Note that if in p.9 matrix U has only 1
row\ldots''): Explain in your own words, based on the given formulas,
how the standard RLS algorithm formulas can be related to a specific
linear regression parameter estimation problem with a specific `initial
estimate' .

\textbf{Question 3}\\
1. Chapter-12 p.29 (``Now insert DFT-matrix\ldots''): Matrix F is said
to be a DFT matrix, but more generally can be any square invertible
matrix. In general, would it also be possible and meaningful to have a
non-square matrix F (with some alternative for the inversion)?\\
In theory no, since you can't compute the inverse of a non-square matrix
but the IDFT is different. If you do the inverse of a DFT you simply
need to do F\^{}H where H is conjugate and transposed, so by using this
you can use rectangular DFT matrices instead of just square. Look at
slide 26 of chapter 3, some formula to do with this.

2. Chapter-13 p.12 (``Necessary \& sufficient condition for PR is
then\ldots''): Based on the first formula, provide an expression for the
distortion function (function of delta and r).

3. Chapter-14 p.13 (``Conclusion: economy in\ldots``): Explain in your
own words the statement ``N filters for the price of 1''.\\
I guess this is about only having to design a prototype filter that
serves for multiple subbands. In slide 11 you just cycle through the n
values to get a new filter every time.\\
See slide 13: H\_0, H\_1, \ldots{} all use the results of the same E\_0,
E\_1, E\_2 and E\_3. In other words, you don't have to implement all the
polyphases of each H separately.

4. Chapter-14 p.27 (``Example-1: Define B(z4)\ldots''): Specify B(z) for
the case where N=8 and D=6 and provide the corresponding proof that an
8-channel DFT-modulated filter bank is obtained with this B(z) (similar
to the proof on p.27 for N=8 and D=4).\\
Not specific to this question but in general N defines the number of
rows and D defines the number of columns. The number of elements within
the matrix is given by N' (formula shown in some previous slide). You
always start at the top left and always go diagonally down. Whenever you
reach the rightmost column, you drop 1 row down and go to the leftmost
column. Whenever you reach the bottommost row, you shift 1 column to the
right and go to the topmost row. Whenever you do the shift from right to
left you multiply the element by z\^{}-D. This results in the matrix
B(z\^{}D). (as in slide 27). This is then converted to B(z) because you
use the nobel identities to switch the structure round (downsampling
before). You divide the exponent of z by the downsampling factor (Note:
after a lot of consulting reports I found online and a bit of trial and
error, I found that the exponent of z becomes N`/D = N/gcd(N,D) rather
than dividing it by D. I know this seems weird for this example since
the exponent becomes 4, but it's consistent with the other examples i've
seen, and just dividing it by D gives you a fraction = 4/3, which makes
even less sense).=\textgreater{} This is just what I observed, not any
formal rules, Moonen at one point in the audio says ``do not ask
questions, this is how we do it'' (for the general construction of
matrix).

5. Chapter-15 p.15 (``Example: N=4, d=2\ldots'\,''). Explain/derive the
``minimum norm solution'' formulas. What is the relevance of such a
``minimum norm solution''

\subsection{January 13 2022}\label{january-13-2022}

\textbf{Question 1}\\
1. Chapter-4 p.10 (``This is a `Quadratic Optimization'\ldots{}''):
Provide similar formulas for Type-2\\
linear phase FIR filter design, and then explain all the symbols used in
these formulas.\\
Maybe something like this: H()=e-jL/2cos(/2)Gd()\\
And in the long equation add cos(/2)Gd() where Gd is.\\
Gd is the amplitude factor of the desired signal

2. Chapter-5 p.6 (p7) (``Starting point is direct form\ldots{}''):\\
a. For the second realization (i.e.~after retiming) specify the noise
transfer function\\
(G(z)=\ldots) for every individual arithmetic operation.\\
b. Which of these noise transfer functions can be lumped, to simplify
further analysis?\\
Explain why such lumping would indeed be allowed.

3. Chapter-5 p.18 (slide16 i think) (``From (*) (page 14), it follows
that\ldots{}''):\\
a. Provide similar formulas for the next order reduction (i.e.~the
second order\\
reduction, following the first order reduction as specified on this
page).\\
To reduce the 3rd order, b0xb3+ \textasciitilde\textasciitilde b0 x
\textasciitilde\textasciitilde b3 = 0 continue with the orthogonal
vectors. From the matrix with the b coefficients we add the zero on the
first line in order to shift the bs and the zero on the second line of
the matrix to remove the coefficient b3 in this case. Continue with the
same process on the slide 16

\begin{enumerate}
\def\labelenumi{\alph{enumi}.}
\setcounter{enumi}{1}
\tightlist
\item
  After the last order reduction, explain why a specific 0-order system
  was obtained (see `Explain' on p.20).\\
  B0 remains after every order reduction but idk why specifically (slide
  10)
\end{enumerate}

Question 2\\
1. Chapter-7 p.24 (``MMSE cost function can be expanded as\ldots{}''):
How does the final Wiener\\
filter formulachange in the case of a multi-channel FIR filter problem
(as\\
on p.19)?\\
The uk will be a vector? (slide20)\\
As in the explanation of exercise 2.1 of jan 12 2023, I think u\_k
becomes a 3x3-matrix and that the w\_wf changes from a vector to a
3x3-matrix because X\_du is now a 3x3-matrix instead of a vector (not
sure of this anymore)

2. Chapter-9 p.37\&38 (slide19) (``2.3 Exponentially weighted
RLS\ldots{}''): Explain (intuitively) how the inclusion of an
exponential weighting factor ``lambda'' in the LS cost function leads to
the appearance of factors 1/(lambda)2 in the RLS formulas.\\
1/lambda\^{}2 appears because we are computing an inverse matrix\\
Intuitively: lambda is multiplied with the input of time k-1 to have
less emphasis on the older inputs. X\_uu en X\_du ((in slides N\_uu and
N\_du is used)) are by definition \textasciitilde{} transpose(u).u so
the X\_uu{[}k-1{]} and X\_du{[}k-1{]} are multiplied by lambda\^{}2.
Now, in the RLS formulas the inverse of X\_uu and X\_du are used
=\textgreater{} 1/ lambda\^{}2

3. Chapter-9 p.21 (slide 35?)(``4-by-4 example\ldots{}''): If the
adaptive filter input signal and the desired output signal have
different `dynamics' (for instance if the characteristics of one are
very stationary, while the characteristics of the other are very
non-stationary), would it be\\
possible/useful to apply two different exponential weighting factors (a
first one in the part\\
corresponding to the input signal and a second one in the part
corresponding to the\\
desired output signal of the signal flow graph)?\\
Slide 30, there can be the same exponential weight factor

4. Chapter-9 p.35 (chapter-10 slide20) (``The main trick\ldots{}''):
Redraw (sketch) the (relevant parts of the) signal flow graph when the
``main trick'' is used to remove the column with R15, R25, etc. Define
the relevant epsilon-signals in the signal flow graph (with subscripts
\& superscripts).\\
Remove elements R15, R21, R35, R45. Dont remove R55 because it belongs
in the diagonal of the triangular matrix A (slide19). I think the
epsilon connect where we want the triangular matrix to end. It will
connect to u{[}k-4{]} and u{[}k+1{]} for the inputs and the output to
what remains, at the R55 element. So they will be defined as
\(\epsilon\) k+1 k-1:k (D matrix element), \(\epsilon\) k-3 k-1:k (E
matrix element), \(\epsilon\) k-3:k-1:k+1 (before the delay element),
\(\epsilon\) k-4 k-3:k (after the delay element)

5. In Chapter-10 p.20(chapter-11 slide 24) (``relevant sub-problem
is\ldots{}''): Explain in detail why the Kalman filter problem can
indeed be reduced to the specified sub-problem.

In a Kalman filter, we can reduce the complexity by not computing the
smoothed values (x0\textbar k, x1\textbar k, \ldots) since the set of
equations at time k can be reduced to a triangular set which allows
triangular backsubstitution, meaning that xk+1\textbar k will be
computed first, hence solving the main problem of computing
xk+1\textbar k without using the smoothed values.

Question 3\\
1. Chapter-11 p.12 (chapter-12 slide14) (``Filter Bank Applications''):
Explain the statement ``PR guarantees distortion-free desired near-end
speech signal''.\\
Perfect reconstruction aims for IN = OUT signal by using appropriate
filter banks and adding gain +++

2. Chapter-12 p.10 (chapter-13 slide30) (``Design
Procedure:\ldots{}''):\\
a. Explain in detail the last sentence (``It will turn out\ldots{}'').
Why is D=N excluded?\\
For D=N there is no FIR for R(z), it will be IIR\\
b. Provide a ``contrived exception'' for D\textless N.\\
+++\\
3. Chapter-13 p.20 (``A filter bank representation\ldots{}''): i cant
find which slide it is referring to\\
a. If H(z)=I, then perfect reconstruction is achieved in that y{[}k{]}
will be a delayed\\
version of u{[}k{]}. In the general case where H(z) is N-by-N, how large
is the delay?\\
Provide an intuitive explanation for this.

\begin{enumerate}
\def\labelenumi{\alph{enumi}.}
\setcounter{enumi}{1}
\tightlist
\item
  Explain in your own words how polyphase decompositions are exploited
  to trade off delay (latency) against complexity in frequency domain
  filter realizations (chapter-5 slide26)\\
  For large sample blocks ⇒ large complexity reduction but large
  latency\\
  Have to derive intermediate realizations with smaller latency and the
  expense of smaller complexity reduction TRADE OFF
\end{enumerate}

\subsection{February 1 2022}\label{february-1-2022}

Question 1\\
1. Chapter-4 p.10 (``This is a `Quadratic Optimization'\ldots{}''):
Provide similar formulas for\\
general (i.e.~not restricted to linear phase) FIR filter design, and
explain all the symbols\\
used in these formulas.\\
Arent all FIR filters linear by default? I dont understand the q\\
(See slide 6 and 7) Phase is non-linear if impulse respons is
non-symmetric =\textgreater{} not possible to simplify to freq response
with cosines\\
=\textgreater{} c does not exist anymore of cosines but of exp()\\
=\textgreater{} x doesn't consist of d\_ks but h{[}k{]}s

2. Chapter-5:\\
a. Explain in your own words the relevance of the Schur-Cohn stability
test appearing\\
in the derivation of the FIR lattice realization.\\
We use the schur-cohn stability test to see if all the zeros of B are
withing the unit circle. We compute all bi's and from those all the
ki's. To be stable the condition \textbar ki\textbar{} \textless{} 1 has
to be satisfied.\\
b. Explain in your own words the relevance of the Schur-Cohn stability
test appearing\\
in the derivation of the IIR lattice-ladder realization.\\
For the same reason but this time we compute the ai's and from those all
the ki's which correspond to the rotation angles sin(theta)

3. Chapter-5 p.12 (``Repeated application results in\ldots{}''):\\
a. For the special case with k0=k1=k2=0, specify the noise transfer
functions (G(z)=\ldots) for every individual arithmetic operation.\\
b. Which of these noise transfer functions can be lumped, to simplify
further analysis? Explain why such lumping would indeed be allowed.

\textbf{Question 2}\\
1. Chapter-7 p.24 (``MMSE cost function can be expanded as\ldots{}''):
How does the Wiener filter\\
formula ( wWF=(Xuu)-1Xdu ) and/or its components (Xuu and Xdu) change in
the case of a\\
`linear combiner' problem (as on p.20)?\\
Same as 13/1/22 exam

2. Chapter-8: Explain in your own words how the characteristics of the
filter input signal (uk)\\
influence the behavior of the LMS algorithm? What is then the `ideal'
filter input signal in\\
this respect?\\
I have written down that the observant signal is related somehow to
auto-correlation, cross-correlation and that it has to be computed over
a time window, maybe something about stochastic characteristics?\\
See slide 12

3. Chapter-9 p.16 (``QR-updating for RLS estimation\ldots{}''): If the
adaptive filter is used in an\\
acoustic echo cancellation application, should the tuning of the
exponential weighting\\
factor (i.e.~setting the lambda to a large or small value) be based on
the characteristics\\
(dynamics) of the input signal, or on the characteristics (dynamics) of
the path, or\\
both?\\
Same as previous exam

4. Chapter-9 p.31 (``Preliminaries / LS residuals are not
changed\ldots{}''): Explain in your own\\
words how this property is exploited in p.36 (``Theorem'').\\
Same as previous exam

5. In Chapter-10 p.6 (chapter-11 slide7) (``The Mean Squared Error (MSE)
of the estimation is..''): In which\\
sense is the MSE defined here different from the MSE used in the MMSE
problem\\
formulation in Chapter-7?\\
Does not take into account the desired signal?

\textbf{Question 3}\\
1. Chapter-11 p.10(chapter-12 slide 13) (``Filter Bank
Applications/Subband Coding''): Explain the relevance of\\
perfect reconstruction in (lossless) subband audio coding.\\
The full band signals are split into subbands and then are downsamples.
The subband signals are separately encoded

2. Chapter-12 p.24(chapter-13 p.30) (``Design Procedure:\ldots{}''):
Explain in your own words how the stability\\
issue mentioned in the last sentence is overcome by having D\textless N
instead of D=N?\\
Same as previous exam

3. Chapter-12 p.30 (chapter-14 slide 27) (``Example-1: Define
B(z)..''):\\
a. Provide a formula for the B-matrix in the case where N=6, D=5.\\
N' = NxD / gcd(N,D) = 6x5 / 1 = 30\\
Same as previous exam

\begin{enumerate}
\def\labelenumi{\alph{enumi}.}
\setcounter{enumi}{1}
\tightlist
\item
  What exactly is demonstrated by the proof at the bottom of the page?\\
  6-channel DFT-modulated filter bank is obtained with this B(z)
\end{enumerate}

4. Chapter-13 (chapter-5 slide 26): Explain in your own words how
polyphase decompositions are exploited to trade off delay (latency)
against complexity in frequency domain filter realizations\\
same as previous exam

\subsection{}\label{section-25}

\subsection{January 7 2021}\label{january-7-2021}

Question 1\\
1. In Chapter-3 p.19, explain the meaning of the formula. What would be
a suitable solution strategy to solve the minimization problem? Idk
which slide it is referring to

2. In Chapter-4 p.12, what could be an alternative strategy to solve the
minimization problem, alternative to computing Q-1 .p? Illustrate by
means of one or a few formulas.\\
Use the minmax design?

3. Counting computational complexity of a filter realization as the
number of multiplications per sample, which FIR filter realization
offers the lowest computational complexity? Provide concise complexity
indications for the FR filter realizations that have been discussed.\\
The direct form has the lowest computational complexity because it uses
less adders and multipliers and u{[}k{]}, u{[}k-1{]} are produced at the
same time\\
4. In Chapter-5 p.14,(chapter-6 p20) why is a `scaling' needed? Is the
scaling unique? Illustrate by means of a figure.\\
Scaling is needed to ensure that the roots of R(z) is inside the unit
circle

5. In Chapter-6 p.34, explain in detail why it is that all noise
transfer functions are the same up to a delay. And then why is it that
the delay does not play a role, such that the noise sources can be
lumped?\\
Noise transfer functions are equal all together because of the fact that
the DC gain and the noise gain are equal thats why the noise TF can be
lumped into one equivalent noise source.\\
Idk about the ``up to a delay'' part

Question 2\\
1. Provide an intuitive explanation for the statement (and the condition
under which the statement holds) that in an acoustic echo cancellation
set-up, the irreducible error is equal to the variance of the near-end
signal, Chapter-7 p.38.

2. What is a `rank-1 update', as mentioned in Chapter-8 p.34 (chapter-9
slide 16), and how does it define the computational complexity of the
standard RLS algorithm?\\
Rank-1 update = inverse common correlation matrix\\
The complexity is reduced

3. Explain the relevance of `residual extraction' in an acoustic echo
cancellation setup, Chapter-9 p.26 (slide 40).\\
In general residual extraction can be used to compute least squares
residuals without explicitly using the least squares filter vector. It
models the acoustic path from the loudspeaker to the microphone ⇒models
the echo contribution in the microphone's signal which is the least
squares residual. The a priori and posteriori are being extracted
sufficiently and then subtracted ⇒ only thing that matters are the
residuals and not the filter coefficients in this case.

4. How is the property illustrated in Chapter-9 p.32 (slide 26) exactly
used in the derivation of the QRD-LSL (least-squares-lattice)
algorithm?\\
Orthogonal factorization and triangular substitution?

5. In Chapter-10 p.20(chapter-11, p.24), explain why the quantities
propagated from time k-1 to time k in the first formula can be
identified as indicated in the second formula.

Question 3\\
1. In Chapter-11 p.34 (chapter-12 slide38), explain why the E(zN) has
polyphase components ordered in a per-row fashion, whereas the R(zN) has
polyphase components ordered in a percolumn fashion.\\
Is it because of downsampling and upsampling?

2. In Chapter-12 p.11(chapter-13 slide 31), explain the statement that
there are D.D.(LE+LR+1) equations and indicate what these equations look
like.\\
(Le+Lr+1) = \#coefficients DxD = dimension of the identity matrix =
\#equations

3. In Chapter-12 p.25 (chapter-14 p.21), explain the statement that E(z)
is FIR \& unimodular if and only each En(z) is FIR \& unimodular.

4. In Chapter-13 p.7, rewrite the formulas such that the `selection
matrix' becomes {[}I4x4 04x4{]} instead of {[}04x4 I4x4{]}. How does
this change the definition of the Bi's? Explain.

5. In Chapter-14 p.24, explain the meaning of the reconstruction formula
(compared to the reconstruction formula of p.4). Why is there a separate
term with the xo's?

\subsection{January 26 2021}\label{january-26-2021}

\textbf{Question 1}\\
1. Chapter-3 p.26 (`To ensure the time-domain signal is real-valued,
have to choose\ldots{}'): Why is a real-valued time-domain signal
needed? How does this choice modify the receiver structure?

2. Chapter-4 p.12: Explain by means of a few formulas how the
minimization problem can be solved using QR-decomposition, instead
computing Q-1 .p.

3. Chapter-5: Consider a linear phase FIR filter. Is it possible to use
a (LPC) lattice realization for this filter? A lossless lattice
realization?

4. Chapter-5 p.13: Explain the relevance of the Schur-Cohn stability
test for the derivation of the lattice-ladder realization.

5. Chapter-6 p.35: Explain why it is that all quantization noises can be
lumped into e1 and e2. What are the corresponding noise transfer
functions?

\textbf{Question 2}\\
1. Chapter-7 p.40: Provide an intuitive explanation for the unrealizable
Wiener filter formula for the two considered cases.

2. Chapter-8 p.21: Explain the `noisy gradients' effect in an acoustic
echo cancellation set-up, with a near-end speaker that is sometimes
active and sometimes not active.

3. Chapter-9 p.26 (38): In this structure with `residual extraction',
where are the actual FIR filter coefficients, i.e.~the estimated model
for the echo path?\\
In the delay line\\
Nope, it is not present in this system, that is the beauty, because we
can calculate the residual without calculating explicitly the model

4. Chapter-9 p.35 (chapter-10): Redraw (sketch!) the (relevant parts of
the) signal flow graph when the `main trick' is used to remove the
column with R15,\ldots R45. Define the relevant epsilon-signals to the
signal flow graph (with subscripts \& superscripts).\\
Same as previous exam. Removing all the elements of those columns except
the diagonal elements

5. Chapter-10 p.19 (chapter-11 p.23): Explain the statement ``is seen to
require only the lowerright/lower part''.

\textbf{Question 3}\\
1. Chapter-11 p.39 (chapter-12 p.40): Explain why the R(zD) is a D-by-N
matrix (and not an N-by-N or D-by-D or N-by-D matrix).\\
Due to the perfect reconstruction condition for the case of D\textless N

2. Chapter-12 p.7(chapter-13 p.20): Explain (in +/- half a page) the
design procedure mentioned at the bottom of the slide.

3. Chapter-12 p.31(chapter-14?): Specify B(z) for the case where N=5 and
D=4.\\
Same as previous exam

4. Chapter-13 p.18: Explain the statement `let B(z) take the place of
distortion function T(z)'. Will there be any distortion in this case?

5. In Chapter-14 p.12, explain the meaning of the reconstruction formula
(at the bottom of the slide) and compared to the reconstruction formula
of p.4.

\subsection{August 10 2020}\label{august-10-2020}

\textbf{Question 1}\\
1. Explain (max 10 lines) how in `weighted least squares' (WLS) based
FIR filter\\
design, imposing a linear phase response reduces the design degrees of
freedom?

2. Can QR-decomposition be used anywhere in the WLS based FIR filter
design\\
process? If yes, where/how?

3. When in WLS based FIR filter design the desired FIR filter has (next
to a desired\\
amplitude response) a desired phase response that is not linear, how
would the\\
design procedure have to be adjusted? Provide formulas.

\textbf{Question 2}\\
1. For an overdetermined set of linear equations A.x=b, the least
squares solution is\\
given as . Manipulate this last formula to derive an equivalent\\
formula that justifies the `backsubstitution' step in the QRD-RLS
algorithm.

2. How is the property illustrated in Chapter-9 page-32 exactly used in
the derivation\\
of the QRD-LSL (least-squares-lattice) algorithm (max 1 page)?

3. If in Chapter-9 page-33 the set of filter input signals\\
{[} u{[}k{]} u{[}k-1{]} u{[}k-2{]} u{[}k-3{]} u{[}k-4{]} {]}\\
is extended to {[} u{[}k{]} u{[}k-1{]} u{[}k-2{]} u{[}k-3{]} u{[}k-4{]}
v{[}k{]} {]}\\
(where v{[}k{]} is a second signal, independent from u{[}k{]}) would it
still be possible to\\
derive/use a QRD-LSL algorithm? If yes, sketch a block scheme (omit
details).

\textbf{Question 3}\\
1. In a maximally decimated (MD) filter bank aliasing usually occurs
because of the\\
downsampling operation. Provide an intuitive explanation (max 10 lines)
why\\
perfect reconstruction (PR) is still possible in an analysis/synthesis
filter bank, in\\
spite of the downsampling and aliasing.\\
The analysis filters are anti-aliasing filters to prevent aliasing in
the subband signals after decimation

2. Compare maximal decimation with oversampling in an analysis/synthesis
filter\\
bank. What are advantages/disadvantages of one versus the other (max 1
page)?\\
Maximal decimation D=N\\
Disadvantages: stability concern + large\\
Advantages: power complementary property

Nothing in the lectures for oversampling

3. Consider a 4-channel DFT-modulated filter bank with 3-fold
downsampling.\\
How can the analysis bank and synthesis bank be realized efficiently?
And then\\
how can perfect reconstruction be obtained in this case? Provide
formulas

\subsection{January 15 2018}\label{january-15-2018}

\begin{enumerate}
\def\labelenumi{\arabic{enumi}.}
\tightlist
\item
  Same as previous exam\\
\item
  \textgreater\textgreater{}\\
\item
  \textgreater\textgreater{}\\
\item
  \textgreater\textgreater{}
\end{enumerate}

\strut \\
2. Only orthogonal transform because it defined the rotational angles in
order to reduce the power order. The orhogonality preserves the norm of
the vectors and the power of the input signals ⇒ lossless

\subsection{January 17 2017}\label{january-17-2017}

\strut \\
All in previous exam

\subsection{August 16 2017}\label{august-16-2017}

\begin{enumerate}
\def\labelenumi{\arabic{enumi}.}
\tightlist
\item
  Noise is apparent in every realization after an adder/multiplier.
  Quantization requires removing least significant bits which introduces
  quntization noise.\\
\item
  The quantization noise is analyzed in a statistical manner, it is non
  linear and deterministic which introduces oscillations\\
\item
  The oscillations can be apparent only if the filter has feedback ⇒
  linear filters FIR cannot have this non-linearity
\end{enumerate}

\begin{enumerate}
\def\labelenumi{\arabic{enumi}.}
\tightlist
\item
  Same as previous exam\\
\item
  A priori = the difference of the desired output at time k and the
  filter output of the previous time k-1 using its previous design
  coefficients. A posteriori = same but using the updated design
  coefficients
\end{enumerate}

\begin{enumerate}
\def\labelenumi{\arabic{enumi}.}
\tightlist
\item
  Same as previous exam\\
\item
  The epsilon superscript = time index of the right hand side signal.
  The epsilon subscript = time indices of the set of the left hand side
  signals\\
\item
  Forward prediction = the next input sample is predicted using the
  previous sample. The opposite for backward prediction
\end{enumerate}

\subsection{January 11 2016}\label{january-11-2016}

\subsection{January 12 2016}\label{january-12-2016}

\begin{enumerate}
\def\labelenumi{\arabic{enumi}.}
\tightlist
\item
  In the quantization of arithmetic operations. When there is zero input
  it is expected for the output to also be zero but because of the
  quantization process, a quantization noise is introduced that created
  non linearities = zero input limit cycle oscillations (due to non
  stochastic signals)\\
\item
  When the filter is linear = does not have feedback then there are no
  oscillations
\end{enumerate}

\begin{enumerate}
\def\labelenumi{\arabic{enumi}.}
\tightlist
\item
  Same as previous exam\\
\item
  \textgreater\textgreater{}\\
\item
  Residual extraction refers to the low matrix elements of the QRD
  algorithm. The residual extraction process creates a priori and
  posteriori residuals that are relevant with the product of the
  rotation angles in the SFG. The residuals can be used to compute least
  squares and in the filter design process instead of using the
  coefficients
\end{enumerate}

\subsection{January 28 2016}\label{january-28-2016}

\begin{enumerate}
\def\labelenumi{\arabic{enumi}.}
\tightlist
\item
  See chap4 slide 4\\
\item
  The two TF have to have the same order so scaling is needed\\
\item
  Same as previous exam
\end{enumerate}

\begin{enumerate}
\def\labelenumi{\arabic{enumi}.}
\tightlist
\item
  Optimal filters are based on the steepest descent iteration method =
  using the previous sample to compute the next one. The quantities used
  for the wiener solution are statistical. In contrast with the RLS
  where is uses expected quantities. The common thing they have is that
  the result is the same = LS solution
\end{enumerate}

3. With the kalman gain vector?

\subsection{January 13 2015}\label{january-13-2015}

\begin{enumerate}
\def\labelenumi{\arabic{enumi}.}
\setcounter{enumi}{1}
\tightlist
\item
  QRD is numerically better method to compute the LS solution
\end{enumerate}

\subsection{}\label{section-26}

\subsection{}\label{section-27}

\subsection{Various questions}\label{various-questions}

\subsubsection{Exam 1}\label{exam-1}

1. Derive lossless lattice for a 3-port. Show the mathematical
derivation as well as SFGs.

\begin{enumerate}
\def\labelenumi{\arabic{enumi}.}
\tightlist
\item
  Does this realization have a pareunitary transfer function?\\
\item
  Does this realization have a unimodular transfer function?
\end{enumerate}

2. Explain DFT modulated filter bank

\begin{enumerate}
\def\labelenumi{\arabic{enumi}.}
\tightlist
\item
  What are the advantages/disadvantages\\
\item
  Compare it to another modulated filter bank\\
\item
  Does the complexity compare with those other modulated filter bank
  compare to a 1-input 4-output lossless ladder FIR filter (in other
  words what is the complexity of cosine modulated FB and 1-input
  4-output lossless ladder)
\end{enumerate}

3. Explain how QR-RLS works

\begin{enumerate}
\def\labelenumi{\arabic{enumi}.}
\tightlist
\item
  Explain the SFG for QRLS\\
\item
  How is the `a posteri' residuals computed a posteriori = division of
  the residual extraction with the rotational angle product = the
  difference between the desired output at time k with the filter output
  at time k-1 using the updated coefficients
\end{enumerate}

\subsubsection{Exam 2}\label{exam-2}

1. Lattice ladder of IIR. How can you be sure that \textbar a4\textbar{}
\textless{} 1?

\begin{enumerate}
\def\labelenumi{\arabic{enumi}.}
\tightlist
\item
  What if all a-coeff are equal to zero. Is H\textasciitilde{} still an
  allpass? If all a=0 then all theta=0 ⇒ the all pass part of the
  realization becomes only a delay line
\end{enumerate}

2. Explain SFTF

\begin{enumerate}
\def\labelenumi{\arabic{enumi}.}
\tightlist
\item
  Make the link with DFT modded filter banks What is the meaning of the
  number of channels?\\
\item
  What if the window length is greater than the number of channels? What
  does the decimation factor mean? D\textgreater N\\
\item
  How does perfect reconstructability affect window function design?
\end{enumerate}

3. What is MMSE? Draw the cost function for a 2-step filter

\begin{enumerate}
\def\labelenumi{\arabic{enumi}.}
\tightlist
\item
  What does the cost function mean for acoustic echo cancellation?\\
\item
  How does the cost function change as the statistical properties of the
  far end signal change?\\
\item
  What influence does the near end signal have\\
\item
  In practice why is the adaptation sometimes turned off when the near
  end signal becomes active?
\end{enumerate}

\subsubsection{Exam 3}\label{exam-3}

1. Explain the use of minimum least square optimization in the design of
FIR filters.

\begin{enumerate}
\def\labelenumi{\arabic{enumi}.}
\tightlist
\item
  Explain the design based on window functions. Which windows are `good'
  windows that qualify for this? Why?\\
\item
  Explain how the windows-based design is linked to the design based on
  minimum least square optimization. Which windows? Which weight
  function?
\end{enumerate}

2. Explain how OFDM (with cyclic prefix) has a convenient mechanism for
channel equalization. For this to works the channel must meet several
conditions. Which ones?

\begin{enumerate}
\def\labelenumi{\arabic{enumi}.}
\tightlist
\item
  OFDM can also be represented as a transmiltiplexer. What are the
  analysis filters (formulas)? Number of channels? What are the
  upsample/downsample factors and what is their meaning?
\end{enumerate}

3. How is the QRD-LSL alsogorhtm derived from the QRD-RLS algorithm?

\begin{enumerate}
\def\labelenumi{\arabic{enumi}.}
\tightlist
\item
  What is residual extraction? How is this used in acoustic echo
  cancellation\\
\item
  How would you implement this for the case of 2 microphones and 1
  speaker?
\end{enumerate}

\subsubsection{Exam 4}\label{exam-4}

1. Explain SFG for QRD-RLS, and how is the QRD-LSL derived from this.

\begin{enumerate}
\def\labelenumi{\arabic{enumi}.}
\tightlist
\item
  Explain on p 12.9 the equality with regard to the cos-products (the
  ones that are therefore not used ecplicitly).\\
\item
  Can QRD-LSL be used for the ``linear combiner'' p 9.29?
\end{enumerate}

2. What is DFT-modulated FB, explain how to obtain an efficient
realizaton for analysis and synthesis bank

\begin{enumerate}
\def\labelenumi{\arabic{enumi}.}
\tightlist
\item
  How does this lead to a design procedure for PR? Possibilities and
  limitations of that procedure? How can FIR lattice filters (p3.13) be
  used here?
\end{enumerate}

3. Explain how lattice ladder realization for IIR is derived

\begin{enumerate}
\def\labelenumi{\arabic{enumi}.}
\tightlist
\item
  p.340: sin = a4, what if \textbar a4\textbar\textgreater1? Same as
  previous exam\\
\item
  if a4=0 then theta0=0. what if all theta = 0? is H\textasciitilde{}
  still all-pass? No its only a delay line
\end{enumerate}

\subsubsection{Exam 5}\label{exam-5}

1. Explain lossless lattice for FIR

\begin{enumerate}
\def\labelenumi{\arabic{enumi}.}
\tightlist
\item
  Calculate thetas for H(z)=1/sqrt(2)+1/sqrt(2)*z\^{}-1
\end{enumerate}

2. Explain: maximally decimated DFT filter bank with perfect
reconstruction

\begin{enumerate}
\def\labelenumi{\arabic{enumi}.}
\tightlist
\item
  Ei(z) = 1+c*z\textsuperscript{-1+d*z}-2 for i=0..3 determine c and d
  for an optimal filter (+ what is optimal) and what does the spectrum
  look like
\end{enumerate}

3. Explain QRD-RLS with residue extraction (+ signal flow in diagram).

\begin{enumerate}
\def\labelenumi{\arabic{enumi}.}
\tightlist
\item
  Draw an acoustic echo cancellation schematic with 2 micros and 2
  loudspeakers
\end{enumerate}

\subsubsection{}\label{section-28}

\subsubsection{Exam 6}\label{exam-6}

1. Discuss WLS in FIR and how to solve via quadratic equation

\begin{enumerate}
\def\labelenumi{\arabic{enumi}.}
\tightlist
\item
  Explain how obtained linear phase filter can be implemented via
  lattice/loseless lattice and what H\textasciitilde(z)/
  H\textasciitilde\textasciitilde(z) looks like (also linear phase or
  not)\\
\item
  Can you also obtain linear phase for IIR filter, if A(z) and B(z) are
  both linear phase filters? can this also be calculated as sum of
  quadratic equation?
\end{enumerate}

2. Describe how polyphase decomposition can be used to achieve perfect
reconstruction.

\begin{enumerate}
\def\labelenumi{\arabic{enumi}.}
\tightlist
\item
  then describe the conditions for alias-free, and for perfect
  reconstruction\\
\item
  now also derives this for transmultiplexers
\end{enumerate}

3. Discuss steepest descent method, and how LMS follows from this

\begin{enumerate}
\def\labelenumi{\arabic{enumi}.}
\tightlist
\item
  discuss influence of loudspeaker signal spectrum in echo cancellation
  on convergence of LMS\\
\item
  discuss influence of spectrum of near-end signal in echo cancellation
  on convergence of LMS
\end{enumerate}

\subsubsection{Exam 7}\label{exam-7}

1. Explain QRD-RLS with extraction residue, and explain what a priori
and postpriori residue are. Same in previous exam

\begin{enumerate}
\def\labelenumi{\arabic{enumi}.}
\tightlist
\item
  explain SFG for QRD-RLs\\
\item
  how can one use qrd-rls in echocancelation where sometimes the filter
  has to adjust adaptively but sometimes not adjust filter coef but
  still keep working.(about the question, basically Rls just works if
  normal but the new values of R and z are not adjusted, I think anyway
  )\\
\item
  what about qrd-lsl when it comes to point b and c
\end{enumerate}

2. What is the importance of `perfect reconstruction' (PR) in relation
to filter bank design? Show with an application and explain

\begin{enumerate}
\def\labelenumi{\arabic{enumi}.}
\tightlist
\item
  In a maximum-decimated (MD) filterbank, aliasing usually occurs due to
  the downsampling. operation. Explain why perfect reconstruction is
  still possible (intuition)? Previous exam\\
\item
  What is a general procedure for the design of MD-PR filter banks\\
\item
  What are `modulated' MD-PR filter banks.\\
\item
  -Explain the design procedure for modulated MD-PR filter banks
\end{enumerate}

3. Explain how lossless lattice ladder realization for FIR is derived.

\begin{enumerate}
\def\labelenumi{\arabic{enumi}.}
\tightlist
\item
  derive lossless lattice from above but with three outputs, so H forms
  a power complementary thing together with 2 other functions. Instead
  of 1 as above.
\end{enumerate}

\subsubsection{}\label{section-29}

\subsubsection{}\label{section-30}

\subsubsection{Exam 8}\label{exam-8}

1.Explain the principle of `weighted least squares' FIR filter design

\begin{enumerate}
\def\labelenumi{\arabic{enumi}.}
\tightlist
\item
  Explain how a quadratic optimization problem is obtained when the
  filter design is limited to a set of `sample' frequencies.\\
\item
  How can the procedure be adapted if the linear phase behavior is not
  structurally imposed? (and where the desired phase behavior may or may
  not be linear\\
\item
  Specify the quadratic optimization problem (formulas) in this case
  (again limiting the filter design to a set of `sample' frequencies
\end{enumerate}

2.What are oversampled DFT-modulated perfect-reconstructing filter
banks?What are the benefits of this?

\begin{enumerate}
\def\labelenumi{\arabic{enumi}.}
\tightlist
\item
  Consider a 6 channel DFT modulated filterbank with 3-way downsampling.
  Describe how such a filter bank is designed, and how a para-unit
  structure can be obtained. Draw the resulting block diagram and sketch
  a possible frequency response of the 6 analysis filters
\end{enumerate}

3.What is Residue Extraction? Explain how residue extraction is
organized in QRD-based RLS. Are a priori residuals larger or smaller
than a priori residuals?

\begin{enumerate}
\def\labelenumi{\arabic{enumi}.}
\tightlist
\item
  Explain how the QRD-LSL (least-square-lattice) algorithm is derived.
  Also explain the epsilon notation used
\end{enumerate}

\subsubsection{Exam 9}\label{exam-9}

1. Explain WLS to IIR. Do you also get a quadratic optimization problem
here?

\begin{enumerate}
\def\labelenumi{\arabic{enumi}.}
\tightlist
\item
  Explain Steigliz-McBride. Where appropriate? Formulas? What about
  sample frequencies?
\end{enumerate}

2. Overlap-save method as representation of an oversampled filter bank?
Explain + draw frequency characteristics of analysis and synthesis
filter.

\begin{enumerate}
\def\labelenumi{\arabic{enumi}.}
\tightlist
\item
  Find an alternative transformation of T(z) and show how this leads to
  overlap-add.
\end{enumerate}

3. What is Residue Extraction? How is this organized in QRD based RLS?
Is the a priori larger/ smaller than the a posteriori? Same as previous
exam

\begin{enumerate}
\def\labelenumi{\arabic{enumi}.}
\tightlist
\item
  How is QRD-LSL derived? Does epsilon notation mean? Is residue
  extraction possible here too? Same as previous exam
\end{enumerate}

\subsubsection{Exam 10}\label{exam-10}

1. How do you derive lossless lattice realization for FIR filters?

\begin{enumerate}
\def\labelenumi{\arabic{enumi}.}
\tightlist
\item
  Suppose you have two transfer functions that you want to realize
  together (1-input/2-output). Can you embed this in a lossless
  1-input/3-output system? How do you arrive at a lossless lattice
  realization? Also work out how to calculate the orthogonal
  transformations.
\end{enumerate}

2. What are unimodular and paraunitary matrices and what is their
importance in perfect reconstruction filter banks?

\begin{enumerate}
\def\labelenumi{\arabic{enumi}.}
\tightlist
\item
  ) What are DFT modulated filter banks? What are the pros and cons?
  Additional question: how can you remedy this disadvantage (not
  paraunitary unless for trivial choices)? c) How would you design a DFT
  modulated transmultiplexer?
\end{enumerate}

3. What is QRD-RLS with Residue Extraction? What are a priori and a
posteriori residues? Same as previous exam\\
Explain the SFG (signal flow graph) b) How can you use the above for
acoustic echo cancellation with 1 loudspeaker and 2 microphones?

\begin{enumerate}
\def\labelenumi{\arabic{enumi}.}
\tightlist
\item
  What if you have two speakers and 1 microphone
\end{enumerate}

\subsubsection{Exam 11}\label{exam-11}

1. Explain: Lossless Lattice FIR filter realization

\begin{enumerate}
\def\labelenumi{\arabic{enumi}.}
\tightlist
\item
  How is H\textasciitilde\textasciitilde{} determined? c) Determine
  H\textasciitilde\textasciitilde{} and draw the realization for: H(z) =
  1/sqrt(2)*cos(theta) + z\^{}(-1) * sin(theta)
\end{enumerate}

2. Explain MD-PR filter banks and what is the requirement for Perfect
reconstruction

\begin{enumerate}
\def\labelenumi{\arabic{enumi}.}
\tightlist
\item
  \begin{enumerate}
  \def\labelenumii{\alph{enumii})}
  \setcounter{enumii}{1}
  \tightlist
  \item
    Discuss in detail the condition for anti - aliasing Same as previous
    exam\\
  \end{enumerate}
\item
  Discuss in detail the condition for Perfect reconstruction Same as
  previous exam
\end{enumerate}

3. Discuss QRD-LSL b) What does the implementation of QRD-LSL for
acoustic cancellation look like with 1 loudspeaker and 1 microphone

\begin{enumerate}
\def\labelenumi{\arabic{enumi}.}
\tightlist
\item
  What does the implementation of QRD-LSL for acoustic consolation look
  like with 2 loudspeakers and 2 microphones
\end{enumerate}

\subsubsection{Exam 12}\label{exam-12}

1. Explain lossless lattice in FIR.

\begin{enumerate}
\def\labelenumi{\arabic{enumi}.}
\tightlist
\item
  Given are 4 FIR filters that are power complementary. Explain how you
  could turn this into a lossless lattice structure. (In other words,
  multiple output lossless lattice, but for 4 instead of 3 as in the
  course. Just explain conceptually, don't deduce in detail. He asked
  which exact elements in that matrix are nullified by the rotations)
\end{enumerate}

2. What are maximally decimated DFT modulated filter banks? What are the
benefits of this? (Additional question: in the general case those Ri's
are not stable if Ei's are FIR. Can you formulate conditions on the Ei's
so that Ri's are still stable?)

\begin{enumerate}
\def\labelenumi{\arabic{enumi}.}
\tightlist
\item
  A maximally decimated DFT filter bank has polyphase components Ei(z) =
  1 + ci z\^{}-1 + di z\^{}-2. for i=0..3. Describe a method to
  determine optimal ci and di (and explain what is optimal).
\end{enumerate}

3. How can you interpret wiener adaptive filters as steepest slope
filters? Explain how you arrive at conditions for the step size µ from
this. How do you make the transition to LMS?

\begin{enumerate}
\def\labelenumi{\arabic{enumi}.}
\tightlist
\item
  \begin{enumerate}
  \def\labelenumii{\alph{enumii}.}
  \setcounter{enumii}{1}
  \tightlist
  \item
    What is the effect of the eigenvalue spectrum of the input signal on
    the operation of the filter? (Tip that was not given on the exam,
    but that he asked about: if all lambdas are equal, that would
    apparently correspond to an input signal that is white noise\ldots)
  \end{enumerate}
\end{enumerate}

\subsubsection{Exam 13}\label{exam-13}

1. Explain how one arrives at the lattice ladder realization of an IIR
filter. What happens when \textbar a4\textbar{} \textgreater{} 1? When
a4 is equal to zero, Theta0 is also equal to zero, what if all a
coefficients are equal to zero, what does H\textasciitilde{} look like?
Is H\textasciitilde{} still an APF? Same as previous exam

\begin{enumerate}
\def\labelenumi{\arabic{enumi}.}
\tightlist
\item
  Additional questions: What kind of filter is H\textasciitilde? (APF)
  How do you see that? (coefficients in reverse order).When at the end
  the whole procedure has to be repeated on the remaining piece, is this
  just an option? (Structure is the same but what is also important is
  that the remaining part is also an APF)\\
\item
  What does H\textasciitilde{} mean?\\
\item
  When all a-coefficients are zero, what structure is left? Same as
  previous exam\\
\item
  What to do when \textbar a4\textbar{} \textgreater{} 1? Same as
  previous exam
\end{enumerate}

2. Explain what Short Time Fourier Transform (STFT) is.

\begin{enumerate}
\def\labelenumi{\arabic{enumi}.}
\tightlist
\item
  What is the relationship between STFT and DFT modulated filter
  banks?\\
\item
  What does the decimation factor mean?\\
\item
  What does the number of channels mean? What happens when the window
  length is greater than the number of channels?\\
\item
  Does the requirement to have PR impose restrictions on the choice of
  window function?
\end{enumerate}

3. Explain the MSE cost function for optimal/adaptive filtering. Sketch
the cost function for a filter with two coefficients.

\begin{enumerate}
\def\labelenumi{\arabic{enumi}.}
\tightlist
\item
  What is the significance of the minimum of the cost function?\\
\item
  For an echocancellation filter, what happens to the shape of the cost
  function when the statistical properties of the far-end signal
  change?\\
\item
  What happens to the shape when the near end signal becomes active?\\
\item
  Why is practically adaptive LMS disabled when no near-end signal is
  present?
\end{enumerate}

\newpage

\section{License}\label{license}

This work is licensed under a
\href{https://creativecommons.org/licenses/by-nc-sa/4.0/}{Creative
Commons Attribution-NonCommercial-ShareAlike 4.0 International License
(CC BY-NC-SA 4.0)}.

You are free to:

\begin{itemize}
\tightlist
\item
  \textbf{Share} --- copy and redistribute the material in any medium or
  format.
\item
  \textbf{Adapt} --- remix, transform, and build upon the material.
\end{itemize}

Under the following terms:

\begin{itemize}
\tightlist
\item
  \textbf{Attribution} --- You must give appropriate credit, provide a
  link to the license, and indicate if changes were made. You may do so
  in any reasonable manner, but not in any way that suggests the
  licensor endorses you or your use.
\item
  \textbf{NonCommercial} --- You may not use the material for commercial
  purposes.
\item
  \textbf{ShareAlike} --- If you remix, transform, or build upon the
  material, you must distribute your contributions under the same
  license as the original.
\item
  \textbf{No additional restrictions} --- You may not apply legal terms
  or technological measures that legally restrict others from doing
  anything the license permits.
\end{itemize}

For the full legal text of the license, please visit:
\url{https://creativecommons.org/licenses/by-nc-sa/4.0/legalcode}

\subsection{Modification}\label{modification}

To contribute to this work or any other one from this project please
find more information at the
\href{https://github.com/Tfloow/Q8_KUL}{Github repository}.

\begin{center}\rule{0.5\linewidth}{0.5pt}\end{center}

\textbf{© 2025 Authors of the Summary, Professors of the Course and
possible book's authors. Some Rights Reserved.}

\end{document}
